\documentclass[11pt,a4paper,twoside]{article}

% Deutsche Spracheinstellungen
\usepackage[english,english]{babel, varioref}
\usepackage[T1]{fontenc}
\usepackage[utf8]{inputenc}

%\usepackage{marvosym}

\usepackage{amsfonts}
\usepackage{amssymb}
\usepackage{amsmath}
\usepackage{amscd}
\usepackage{amstext}
\usepackage{float}
\usepackage{caption}
\usepackage{wrapfig}
\usepackage{setspace}
%\usepackage[onehalfspacing]{setspace}
\usepackage{threeparttable}
\usepackage{footnote}
\usepackage{feynmf}
\usepackage{bbm}
\usepackage{slashed}
\usepackage{textcomp}
\usepackage{multirow}
\usepackage{courier}
\usepackage{listings}
\usepackage{color}
%\usepackage{minipage}
 
 \definecolor{middlegray}{rgb}{0.5,0.5,0.5}
 \definecolor{lightgray}{rgb}{0.8,0.8,0.8}
 \definecolor{orange}{rgb}{0.8,0.3,0.3}
 \definecolor{yac}{rgb}{0.6,0.6,0.1}
 \definecolor{puple}{rgb}{0.62,0.12,0.94}
 \lstset{language=Python,
                basicstyle=\ttfamily,
                keywordstyle=\color{red}\ttfamily,
                stringstyle=\color{magenta}\ttfamily,
                commentstyle=\color{blue}\ttfamily,
                morecomment=[l][\color{blue}]{\#},
		%stepnumber=1,
		%numberstyle=\color{magenta}\ttfamily,
		%    numbers=left,
		%    numberstyle={},
		%    numberblanklines=false,
		%    stepnumber=1,
		%    numbersep=10pt,
		    xleftmargin=15pt,
 		moredelim=[is][\color{purple}]{|}{|}
}

\newfloat{formel}{htbp}{for}
\floatname{formel}{Formel}

\onehalfspacing
%\setstretch {1.433}

\usepackage{longtable}

%\usepackage{bibgerm}

\usepackage{footnpag}

\usepackage{ifthen}                 %%% package for conditionals in TeX
\usepackage[amssymb]{SIunits}
%Fr textumflossene Bilder und Tablellen
%\usepackage{floatflt} - veraltet

%Fr Testzwecke aktivieren, zeigt labels und refs im Text an.
%\usepackage{showkeys}

% Abstand zwischen zwei Abs�zen nach DIN (1,5 Zeilen)
% \setlength{\parskip}{1.5ex}

% Einrckung am Anfang eines neuen Absatzes nach DIN (keine)
%\setlength{\parindent}{0pt}

% R�der definieren
% \setlength{\oddsidemargin}{0.3cm}
% \setlength{\textwidth}{15.6cm}

% bessere Bildunterschriften
\usepackage{caption2}


% Probleml�ungen beim Umgang mit Gleitumgebungen
\usepackage{float}

% Nummeriert bis zur Strukturstufe 3 (also <section>, <subsection> und <subsubsection>)
%\setcounter{secnumdepth}{3}

% Fhrt das Inhaltsverzeichnis bis zur Strukturstufe 3
%\setcounter{tocdepth}{3}

\usepackage{exscale}

\newenvironment{dsm} {\begin{displaymath}} {\end{displaymath}}
\newenvironment{vars} {\begin{center}\scriptsize} {\normalsize \end{center}}


\newcommand {\en} {\varepsilon_0}               % Epsilon-Null aus der Elektrodynamik
\newcommand {\lap} {\; \mathbf{\Delta}}         % Laplace-Operator
\newcommand {\R} { \mathbb{R} }                 % Menge der reellen Zahlen
\newcommand {\e} { \ \mathbf{e} }               % Eulersche Zahl
\renewcommand {\i} { \mathbf{i} }               % komplexe Zahl i
\newcommand {\N} { \mathbb{N} }                 % Menge der nat. Zahlen
\newcommand {\C} { \mathbb{C} }                 % Menge der kompl. Zahlen
\newcommand {\Z} { \mathbb{Z} }                 % Menge der kompl. Zahlen
\newcommand {\limi}[1]{\lim_{#1 \rightarrow \infty}} % Limes unendlich
\newcommand {\sumi}[1]{\sum_{#1=0}^\infty}
\newcommand {\rot} {\; \mathrm{rot} \,}         % Rotation
\newcommand {\grad} {\; \mathrm{grad} \,}       % Gradient
\newcommand {\dive} {\; \mathrm{div} \,}        % Divergenz
\newcommand {\dx} {\; \mathrm{d} }              % Differential d
\newcommand {\cotanh} {\; \mathrm{cotanh} \,}   %Cotangenshyperbolicus
\newcommand {\asinh} {\; \mathrm{areasinh} \,}  %Area-Sinus-Hyp.
\newcommand {\acosh} {\; \mathrm{areacosh} \,}  %Area-Cosinus-H.
\newcommand {\atanh} {\; \mathrm{areatanh} \,}  %Area Tangens-H.
\newcommand {\acoth} {\; \mathrm{areacoth} \,}  % Area-cotangens
\newcommand {\Sp} {\; \mathrm{Sp} \,}
\newcommand {\mbe} {\stackrel{\text{!}}{=}}     %Must Be Equal
\newcommand{\qed} { \hfill $\square$\\}
\newcommand{\midtilde}{\raisebox{-0,25\baselineskip}{\textasciitilde}}
\renewcommand{\i} {\imath}
\def\captionsngerman{\def\figurename{\textbf{Abb.}}}

%%%%%%%%%%%%%%%%%%%%%%%%%%%%%%%%%%%%%%%%%%%%%%%%%%%%%%%%%%%%%%%%%%%%%%%%%%%%
% SWITCH FOR PDFLATEX or LATEX
%%%%%%%%%%%%%%%%%%%%%%%%%%%%%%%%%%%%%%%%%%%%%%%%%%%%%%%%%%%%%%%%%%%%%%%%%%%%
%%%
\ifx\pdfoutput\undefined %%%%%%%%%%%%%%%%%%%%%%%%%%%%%%%%%%%%%%%%% LATEX %%%
%%%
\usepackage[dvips]{graphicx}       %%% graphics for dvips
\DeclareGraphicsExtensions{.eps,.ps}   %%% standard extension for included graphics
\usepackage[ps2pdf]{thumbpdf}      %%% thumbnails for ps2pdf
\usepackage[ps2pdf,                %%% hyper-references for ps2pdf
bookmarks=true,%                   %%% generate bookmarks ...
bookmarksnumbered=true,%           %%% ... with numbers
hypertexnames=false,%              %%% needed for correct links to figures !!!
breaklinks=true,%                  %%% breaks lines, but links are very small
linkbordercolor={0 0 1},%          %%% blue frames around links
pdfborder={0 0 112.0}]{hyperref}%  %%% border-width of frames
%                                      will be multiplied with 0.009 by ps2pdf
%
\hypersetup{ pdfauthor   = {Dimitrios Skodras},
pdftitle    = {Fermionic Dark Matter and its Role on B Anomalies}, pdfsubject  = {masterthesis}, pdfkeywords = {dark matter},
pdfcreator  = {LaTeX with hyperref package}, pdfproducer = {dvips
+ ps2pdf} }
%%%
\else %%%%%%%%%%%%%%%%%%%%%%%%%%%%%%%%%%%%%%%%%%%%%%%%%%%%%%%%%% PDFLATEX %%%
%%%
\usepackage[pdftex]{graphicx}      %%% graphics for pdfLaTeX
\DeclareGraphicsExtensions{.pdf}   %%% standard extension for included graphics
\usepackage[pdftex]{thumbpdf}      %%% thumbnails for pdflatex
\usepackage[pdftex,                %%% hyper-references for pdflatex
bookmarks=true,%                   %%% generate bookmarks ...
bookmarksnumbered=true,%           %%% ... with numbers
hypertexnames=false,%              %%% needed for correct links to figures !!!
breaklinks=true,%                  %%% break links if exceeding a single line
linkbordercolor={0 0 1},
linktocpage]{hyperref} %%% blue frames around links
%                                  %%% pdfborder={0 0 1} is the default
\hypersetup{
pdftitle    = {Fermionic Dark Matter and its Role on B Anomalies}, %right place
pdfsubject  = {master thesis}, 
pdfkeywords = {V301, Innenwiderstand, Leistungsanpassung},
pdfsubject  = {Protokoll AP},
pdfkeywords = {V301, Innenwiderstand, Leistungsanpassung}}
%                                  %%% pdfcreator, pdfproducer,
%                                      and CreationDate are automatically set
%                                      by pdflatex !!!
\pdfadjustspacing=1                %%% force LaTeX-like character spacing
\usepackage{epstopdf}
%
\fi %%%%%%%%%%%%%%%%%%%%%%%%%%%%%%%%%%%%%%%%%%%%%%%%%%% END OF CONDITION %%%
%%%%%%%%%%%%%%%%%%%%%%%%%%%%%%%%%%%%%%%%%%%%%%%%%%%%%%%%%%%%%%%%%%%%%%%%%%%%
% seitliche Tabellen und Abbildungen
%\usepackage{rotating}
\usepackage{ae}
\usepackage{
  array,
  booktabs,
  dcolumn
}
\makeatletter 
  \renewenvironment{figure}[1][] {% 
    \ifthenelse{\equal{#1}{}}{% 
      \@float{figure} 
    }{% 
      \@float{figure}[#1]% 
    }% 
    \centering 
  }{% 
    \end@float 
  } 
  \makeatother 


  \makeatletter 
  \renewenvironment{table}[1][] {% 
    \ifthenelse{\equal{#1}{}}{% 
      \@float{table} 
    }{% 
      \@float{table}[#1]% 
    }% 
    \centering 
  }{% 
    \end@float 
  } 
  \makeatother 
%\usepackage{listings}
%\lstloadlanguages{[Visual]Basic}
%\allowdisplaybreaks[1]
%\usepackage{hycap}
%\usepackage{fancyunits}

\usepackage{xfrac}
\usepackage{xcolor}
\usepackage{setspace}\usepackage{threeparttable}


 %\setlength{\parskip}{1.5ex}
\begin{document}
\begin{spacing}{1,2}
\pagenumbering{Roman}

% Anmerkung: Die Seitenraender wurden asymmetrisch gewaehlt,
%            damit genug Platz fuer eine Klemmbindung da ist.
%            Da neue Kapitel auf der rechten Seite (ungerade
%            Seitennummer) beginnen sollten, muss ggf. am Ende
%            des vorhergehenden Kapitels eine Leerseite
%            eingefuegt werden:
%
%            \newpage
%            \thispagestyle{empty}
%            \ \\
%            \newpage
%
%            Die Seitenraender koennen aber auch in der Datei Tex/global.tex
%            veraendert werden.

% >>> Titelseite <<<

\newcommand{\thetitle}{Flavour Portal to Dark Matter}

\thispagestyle{empty}
\begin{center}

\Huge\textbf{\thetitle}
\vfill
% Note that the size is given in normal parentheses
% instead of curly brackets.
% Define external vertices from bottom to top
\vfill
\Large
Masterarbeit\\ zur Erlangung des akademischen Grades \\ Master of Science \\
\vspace{20pt}
\normalsize
vorgelegt von \\[5pt]
{\Large Dimitrios Skodras} \\[5pt]
geboren in Aschaffenburg \\
\vspace{20pt}
Lehrstuhl für Theoretische Physik IV \\ Fakultät Physik \\
Technische Universität Dortmund \\ 2014
\end{center}
\newpage
% % >>> Gutachterseite <<<
% \thispagestyle{empty}
% \newpage
% \cleardoublepage

\thispagestyle{empty}
\vspace*{\fill}
\begin{tabbing}
1. Gutachter : \=\kill
1. Gutachter : \>Prof. Dr. Gudrun Hiller \\[11pt]
2. Gutachter : \> \\[11pt]
\end{tabbing}
\vspace{11pt}
Datum des Einreichens der Arbeit: 31. Oktober, 2016
\newpage
\thispagestyle{empty}
\begin{flushright} 
\textit{\glqq Es gibt nichts Praktischeres, als eine gute Theorie.\grqq}\\
- \textit{Kant, Immanuel}\\
\vspace{2cm}
% \textit{``Was dürfen wir hoffen?''}\\
% - \textit{Kant, Immanuel}\\
% \vspace{2cm}
% \textit{``Kein Mensch ist so wichtig, wie er sich nimmt.''}\\
% - \textit{Kant, Immanuel}
\end{flushright}

\newpage


% >>> Kurzfassung/Abstract <<<

\thispagestyle{empty}
%Kuzfassung
 \section*{Kurzfassung}

 \section*{Abstract}

 \section*{Abrégé}
 \newpage

% >>> Hauptteil <<<

%\addcontentsline{toc}{chapter}{Inhaltsverzeichnis}
\end{spacing}
\begin{spacing}{1,1}
\tableofcontents\newpage

\end{spacing}
\begin{spacing}{1,2}

\thispagestyle{empty}
% \cleardoublepage

\setcounter{page}{0}
\pagenumbering{arabic}

\section{Introduction}
After the prediction (1964) and the experimental verification (2012) of the
now called Higgs-boson, the Standard Model (SM) could be considered self-
contained. The forecasts it makes for processes described therein are proven
remarkably well by various experiments. At least at the energy scales they
are currently operating. That said it already contains associations that the
contributions to this magnificent theory made by our scientific forefathers are
not the end of the story. It still lacks for example a quantum field theoretical
description of gravity or does not contain the experimentally shown masses
of the neutrinos. Besides, its dynamics depend on 19 parameters, including
the masses of the fermions, which are quite a lot and are wanted to decrease
in number by finding relations among one another. Furthermore cosmology
tells us that the SM is only covering about tenty per cent of the mass content
of the universe where the rest is made of the nowadays called dark matter
(DM). Since this thesis contains the words flavour and dark matter, we
should have a closer look on the way things stand at the moment.

\subsection{Flavour in the SM}
The Standard Model is a gauge quantum field theory whose internal symmetry is the unitary product group $SU(3)_C\times SU(2)_L\times U(1)_{Y_W}$ representing
the quantum chromodynamics (QCD) whose charge is called color $C$ and the electroweak theory (GSW-Theory) whose charges are the weak isospin $T$ 
(hold by left handed particles) and the 
weak hypercharge $Y_W$. These interactions are mediated by a total oftwelve gauge boson fields, eight gluon fields $G_a$ for strong interactions and four 
electroweak boson fields from which $W_1$, $W_2$, $W_3$ couple to the weak isospin and $B$ to the weak hypercharge. The field $\phi$ of the already mentioned 
Higgs holds a special role in the SM. It is known that the boson fields $W^\pm$ and $Z$, responsible for weak processes, have nonzero masses. But their
mass terms would break the gauge invariance of the lagrangian. So the Higgs-mechanism was considered wherein $\phi$ develops a non vanishin vacuum expectation
value $v$ (vev) which breaks the electroweak symmetry spontaniously down to the electromagnetic symmetry $U(1)_Q$ resulting in a still massless photon field 
$A$ and the three just named massive ones.
\subsubsection*{Yukawa interaction}
Not only the bosons get their masses from this mechanism but the fermions as well - at least the electrically charged ($Q$) ones - what is represented by the
Yukawa (scalar-fermion interaction) term in the SM-lagrangian
\begin{equation}
 \mathcal{L}_{Y} = - y^u_{ij}\, \bar Q^i_L \, \phi^c\, u^j_R - y^d_{ij}\, \bar Q^i_L\, \phi\, d^j_R + y^e_{ij}\, \bar L^i_L\, \phi\, e^j_R - \text{h.c.}.
 \label{eq_yukawaSM}
\end{equation}
\noindent
This equation needs some explanation. $y^u$, $y^d$ and $y^e$ are $3\times 3$ real, so called yukawa matrices, thinking of three generations of fermions, and represent the 
coupling of the fermions to the Higgs. ``h.c.'' stands for hermetian conjugate so that it holds for the antiparticles as well. 
$Q_L$, $L_L$ and $\phi$ are doublets of the $SU(2)_L$, since they are build out of two fermion fields each holding a 
weak isospin of $T=\sfrac12$ with the 3rd component $T_3 = \pm\sfrac12$, e.g. $L^2_L = (\nu_{\mu\, L}, \mu_L)^T_{\textbf{2}}$ with the left handed muon-neutrino and muon. 
Their counterparts $u_R$, $d_R$ and $e_R$ are singlets of the $SU(2)_L$ because they have no weak isospin, e.g. $e^2_R = (\mu_R)_{\textbf{1}}$. After 
the symmetry breaking and rotating the fermion fields in a basis where the yukawa matrices become diagonal, we can write down their mass terms 
\begin{equation}
 \mathcal{L}_m = -m^u_i \bar u'^i_L u'^i_R -m^d_i \bar d'^i_L d'^i_R -m^e_i \bar e'^i_L e'^i_R 
 \label{eq_massSM}
\end{equation}
\noindent
where $m^\alpha_i \sim y^\alpha_i \cdot v$ ($\alpha = u,d,e$). Here you can see that the SM does not distinguish between generations. 
It treats a left handed up quark the same way it treats a charm or a top. So if the eigenvalues of $m^\alpha$ would be degenerate, i.e. the masses would
be all the same, one would not have a method to differentiate them. 

\vspace{-0.3cm}
You could ask
what happened with a term like $\nu_R$ in \eqref{eq_yukawaSM}. This would imply a particle which has neither a color charge, nor electrical charge, nor 
weak isospin and hence ($Q = T_3 + \sfrac12 Y_W$) no hypercharge which means that the SM is totally blind to it. Furthermore it seemed obvious not to be
able to construct a neutrino mass since they were considered massless. But to answer the question we could add 
such a term somehow when it can be shown for example that righthanded neutrinos exist, which is likely due to the measurements of neutrino oscillations.
\vspace{1cm}
\subsubsection*{Weak interaction}
Neutrino oscillations suggest that \textit{flavour} is not conserved in nature. As this phenomenon is still rather young (2001) and not implemented in the 
SM we already have flavour violation (FV) therein, in the quark sector. There is a huge list of Mesons and Baryons (composites of one quark and one anti-quark,
or three quarks respectively) decaying in others with different quark content enabled by the $W$-bosons. Formerly started by Cabibbo and pursued by 
Kobayashi and Maskawa for CP-violation reasons (charge conjugation C, parity P), the CKM-matrix is an unitary $3\times 3$ matrix which can be thought of as a rotation
matrix, rotating the weak eigenstates of the down-type quarks $d^i$ (like in \eqref{eq_yukawaSM}) in the mass eigenstates $d'^i$ 
(like in \eqref{eq_massSM}) by three Euler angles, $\theta_{12} = \theta_C$ called Cabibbo angle as well as $\theta_{23}$ and $\theta_{13}$.
\begin{equation}
 \begin{pmatrix}
  d' \\ s' \\ b'
 \end{pmatrix} = V_\text{CKM}  \begin{pmatrix}
  d \\ s \\ b
 \end{pmatrix},
\end{equation}
\noindent
where $V_\text{CKM}$ can be parameterised by $\lambda = \sin(\theta_C) \approx 0.2$ at leading order (LO) as
\begin{equation}
 V_\text{CKM} \approx \begin{pmatrix}
  1 & \lambda & \lambda^3\\
  -\lambda & 1 & \lambda^2\\
  -\lambda^3 & -\lambda^2 & 1
 \end{pmatrix}.
\end{equation}
The up-type quarks have no differences between their weak and mass eigenstates, but we could have played the same game with them leading to the 
same mixing, since $V_\text{CKM}$ is unitary. Looking at this matrix one can see that it is almost diagonal and hierachical which
means that the farther you leave the main diagonal the smaller become the magnitudes of the entries. Furthermore it is symmetric in respect of the 
their moduli.

Now comes one of the basic questions motivating this thesis: Are these patterns random or is there rather an underlying broken family-symmetry which
could serve as a blindman's stick for the SM? We go with our guts and prefer the latter suggestion.

\subsection{DM in cosmology}
dfasdölkj\\d\\d\\d\\d\\d\\d\\d\\d\\d\\d\\d

\section{DM candidate coupling to light quarks in T' framework}
\subsection{Aspects of discrete group}
\subsection{Assigning particles to multiplets}
\subsection{Messenger $\xi''$ mediating SM and DM}
\subsubsection{DM annihilation into light mesons}
\subsubsection{Meson-Mixing}
\subsubsection{DM-nucleus scattering}


\addcontentsline{toc}{section}{List of figures}
\newpage\listoffigures\newpage
\addcontentsline{toc}{section}{List of tables}
\listoftables\newpage
\end{spacing}

\end{document}