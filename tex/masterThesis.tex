\documentclass[11pt,a4paper,twoside]{article}
\input{longheader.tex}
\usepackage{xfrac}
\usepackage{xcolor}
\usepackage{setspace}\usepackage{threeparttable}
\usepackage{fancyhdr}
\usepackage{graphicx}
\numberwithin{equation}{section}
%\usepackage{epstopdf}
\newcommand{\ti}{\text{i}}
\fancyfoot{}
\fancyhead[RO,LE]{\thepage}
\fancyhead[LO]{\leftmark}
\fancyhead[RE]{\rightmark}


 %\setlength{\parskip}{1.5ex}
\begin{document}
\begin{spacing}{1,2}
\pagenumbering{Roman}

% Anmerkung: Die Seitenraender wurden asymmetrisch gewaehlt,
%            damit genug Platz fuer eine Klemmbindung da ist.
%            Da neue Kapitel auf der rechten Seite (ungerade
%            Seitennummer) beginnen sollten, muss ggf. am Ende
%            des vorhergehenden Kapitels eine Leerseite
%            eingefuegt werden:
%
%            \newpage
%            \thispagestyle{empty}
%            \ \\
%            \newpage
%
%            Die Seitenraender koennen aber auch in der Datei Tex/global.tex
%            veraendert werden.

% >>> Titelseite <<<

\newcommand{\thetitle}{Fermionic Dark Matter and its Role in B-anomalies}

\thispagestyle{empty}
\begin{center}

\Huge\textbf{\thetitle}
\vfill
% Note that the size is given in normal parentheses
% instead of curly brackets.
% Define external vertices from bottom to top
\vfill
\Large
Masterarbeit\\ zur Erlangung des akademischen Grades \\ Master of Science \\
\vspace{20pt}
\normalsize
vorgelegt von \\[5pt]
{\Large Dimitrios Skodras} \\[5pt]
geboren in Aschaffenburg \\
\vspace{20pt}
Lehrstuhl für Theoretische Physik IV \\ Fakultät Physik \\
Technische Universität Dortmund \\ 2016
\end{center}
\newpage
% % >>> Gutachterseite <<<
% \thispagestyle{empty}
% \newpage
% \cleardoublepage

\thispagestyle{empty}
\vspace*{\fill}
\begin{tabbing}
1. Gutachter : \=\kill
1. Gutachter : \>Prof. Dr. Gudrun Hiller \\[11pt]
2. Gutachter : \> \\[11pt]
\end{tabbing}
\vspace{11pt}
Datum des Einreichens der Arbeit: 31. Oktober, 2016
\newpage
\thispagestyle{empty}
\begin{flushright} 
\textit{\grqq Es gibt nichts Praktischeres, als eine gute Theorie.\grqq}\\
- \textit{Kant, Immanuel}\\
\vspace{2cm}
% \textit{``Was dürfen wir hoffen?''}\\
% - \textit{Kant, Immanuel}\\
% \vspace{2cm}
% \textit{``Kein Mensch ist so wichtig, wie er sich nimmt.''}\\
% - \textit{Kant, Immanuel}
\end{flushright}

\newpage


% >>> Kurzfassung/Abstract <<<

\thispagestyle{empty}
%Kuzfassung
 \section*{Kurzfassung}

 \section*{Abstract}

 \section*{Abrégé}
 \newpage

% >>> Hauptteil <<<

%\addcontentsline{toc}{chapter}{Inhaltsverzeichnis}
\end{spacing}
\begin{spacing}{1,1}
\tableofcontents\newpage

\end{spacing}
\begin{spacing}{1,2}

\thispagestyle{empty}
 \cleardoublepage

\pagenumbering{arabic}
\setcounter{page}{1}
\pagestyle{fancy}

\section{Introduction}
After the prediction (1964) (CITE) and the experimental verification (2012) (CITE) of the
now called Higgs-boson, the Standard Model (SM) could be considered self-
contained. The forecasts it makes for processes described therein are proven
remarkably well by various experiments. At least at the energy scales they
are currently operating. That said, it already contains associations that the
contributions to this magnificent theory, made by our scientific forefathers, might not be
the end of the story. We might know the Higgs mechanism being source of mass of visible particles,
but the reason for their absolute values is still notional. The fermions thereof can be put into
small groups and are considered to be interacting equally with the gauge bosons but it appears
that some break ranks. Not only in the elementary particle physics sphere but also in cosmology
strange things happen. From observation it seems that there has to be a mass distribution of
invisible particles, say dark matter, that lead to the actual orbits of the stars in the galaxy.
There is of course even more, but the resolution of these three issues in one combined theory is
the task of this thesis.

\noindent To resolve a problem, one has to understand it first. So we start with some notes on 
related topics in the SM and cosmology.


% It still lacks for example a quantum field theoretical
% description for gravity or does not contain the experimentally shown masses
% of the neutrinos. Besides, its dynamics depend on 19 parameters, including
% the masses of the fermions, which are quite a lot and are wanted to decrease
% in number by finding relations among one another. Furthermore cosmology
% tells us that the SM is only covering about twenty per cent of the mass content
% of the universe where the rest is made of the nowadays called dark matter
% (DM). Since this thesis is about flavour and dark matter, we
% should have a closer look on the way things stand at the moment.


\section{A Guide to today's Physics}

\subsection{Flavour in the SM}
% The Standard Model is a theory that contains particle fields, ordinary matter is composed of and messenger fields mediating interactions between them. 
% These fields can be separated by their quantum numbers indicating their couplings to each other. At first we have the twelve gauge vector boson fields, 
% eight gluon fields $G_a$ for strong interactions with coloured ($C$) particles and four electroweak boson fields $W_1$, $W_2$, $W_3$ and $B$ which couple
% to particles with weak isospin $T$ and hypercharge $Y_W$ respectively. Now we have 24 fermion fields, six leptons and six quarks each carrying one of three colours.
% The leptons can be subdivided into three electrically ($Q=T+Y_W$) charged and three uncharged ones and the quarks into three up-type ($T=\sfrac12$) and
% three down-type ($T=-\sfrac12$). Last but not least, the already mentioned scalar Higgs-field $\phi$ which develops a non vanishing vacuum expectation value
% (vev) $v$. This is the source for the spontanious breaking of the electroweak symmetry into the electromagnetic 
% (NOT)
The Standard Model is a gauge quantum field theory whose internal symmetry is the unitary product group $SU(3)_C\times SU(2)_L\times U(1)_{Y_W}$ representing
the quantum chromodynamics (QCD) whose charge is called color $C$ and the electroweak theory (GSW-Theory) whose charges are the weak isospin $T$ 
(hold by left handed particles) and the weak hypercharge $Y_W$. These quantum numbers (QN) are carried by a set of particle fields, ordinary matter is composed
of and messenger fields mediating these interactions between them. In particular twelve gauge vector boson fields, eight gluon fields $G_a$ for strong 
interactions and four electroweak boson fields from which $W_1$, $W_2$, $W_3$ couple to the weak isospin and $B$ to the weak hypercharge. Now we have 12
fermion fields, six leptons $e$, $\mu$, $\tau$ as well as their respective neutrinos $\nu$ and six quarks $u$, $d$, $s$, $c$, $b$, and $t$. Actually there
are even more since they are defined by their QN. So for each fermion there is a distinction drawn between left handed $f_L$ ($T=\sfrac12$) and right 
handed $f_R$ ($T=0$), although $\nu_R$ are not considered in the SM. Furthermore there are three different colours for each quark. Well finally the field 
$\phi$ of the already mentioned scalar
Higgs-boson holds a special role in the SM. It is known that the boson fields $W^\pm$ and $Z$, responsible for weak processes, have nonzero masses. But their
mass terms would break the gauge invariance of the lagrangian. So the Higgs-mechanism was considered wherein $\phi$ develops a non vanishing vacuum expectation
value $v$ (vev) which breaks the electroweak symmetry spontaniously down to the electromagnetic symmetry $U(1)_Q$ resulting in a still massless photon field 
$A$ and the three just named massive ones.
\subsubsection{Yukawa Interaction}
Not only the bosons get their masses from this mechanism but the fermions as well - at least the electrically charged ($Q$) ones - which is represented by the
Yukawa (scalar-fermion interaction) term in the SM-lagrangian
\begin{equation}
 \mathcal{L}_{Y} = - y^u_{ij}\, \bar Q^i_L \, \phi^c\, u^j_R - y^d_{ij}\, \bar Q^i_L\, \phi\, d^j_R - y^e_{ij}\, \bar L^i_L\, \phi\, e^j_R + \text{h.c.}.
 \label{eq_yukawaSM}
\end{equation}
\noindent
$y^u$, $y^d$ and $y^e$ are $3\times 3$ (three generations) real, so called yukawa matrices and represent the 
coupling of the fermions to the Higgs. $\phi^c = \ti\sigma_2\phi^*$ is the charged conjugate Higgs. ``h.c.'' stands for hermetian conjugate so that it holds for the antiparticles as well. 
$Q_L$, $L_L$ and $\phi$ are doublets of the $SU(2)_L$, since they are built out of two fermion fields each holding a 
weak isospin of $T=\sfrac12$ with the 3rd component $T_3 = \pm\sfrac12$, e.g. $L^2_L = (\nu_{\mu\, L}, \mu_L)^T_{\textbf{2}}$ with the left handed muon-neutrino and the muon. 
Their counterparts $u_R$, $d_R$ and $e_R$ are singlets under the $SU(2)_L$ because they have no weak isospin, e.g. $e^2_R = (\mu_R)_{\textbf{1}}$, and hence don't take part
in the weak interaction mediated by the $W$-bosons. After 
the symmetry breaking and rotating the fermion fields in a basis where the yukawa matrices become diagonal, we can write down their mass terms 
\begin{equation}
 \mathcal{L}_m = -m^u_i \bar u'^i_L u'^i_R -m^d_i \bar d'^i_L d'^i_R -m^e_i \bar e'^i_L e'^i_R 
 \label{eq_massSM}
\end{equation}
\noindent
where $m^\alpha_i \sim y^\alpha_i \cdot v$ ($\alpha = u,d,e$). Here you can see that the SM does not distinguish between generations. 
It treats a left handed up quark the same way it treats a left handed charm or a left handed top. So if the eigenvalues of $m^\alpha$ would be degenerate, i.e. the masses would
be all the same, one would not have a method to differentiate them. 

% \vspace{-0.3cm}
You could ask
what happened with a term like $\nu_R$ in \eqref{eq_yukawaSM}. This would imply a particle which has neither a color charge, nor electrical charge, nor 
weak isospin and hence ($Q = T_3 + Y_W$) no hypercharge which means that the SM is totally blind to it. Furthermore it seemed obvious not to be
able to construct a neutrino mass since they were considered massless. But to answer the question we could add 
such a term somehow when it can be shown for example that righthanded neutrinos exist, which is likely due to the measurements of neutrino oscillations.



\subsubsection{Weak Interaction}
Neutrino oscillation is a process where a neutrino of one generation changes its flavour while propagating through space described by the PMNS matrix. 
This suggests that 
flavour is not a conserved quantum number in nature unlike the electrical charge for example. As this phenomenon is still rather young (2001) and not implemented in the 
SM, we already have flavour violation (FV) therein, in the quark sector. There is a huge list of Mesons and Baryons (composites of one quark and one anti-quark,
or three quarks, respectively) decaying into others with different quark content enabled by the $W$-bosons via charged currents. 
\\ \\ \textit{Flavour Changing Charged Currents}\\
\noindent So it was thought, that the down-type quark mass
eigenstates $d'^i$ are superpositions of their interaction eigenstates $d^i$. Formerly started by Cabibbo and pursued by 
Kobayashi and Maskawa (1973) for CP-violation reasons (charge conjugation C, parity P), the CKM-matrix as an unitary $3\times 3$ matrix was invented,
which can be thought of as a rotation
matrix, rotating the weak eigenstates of the down-type quarks in the mass eigenstates by three Euler angles, $\theta_{12} = \theta_C$ called Cabibbo angle 
as well as $\theta_{23}$ and $\theta_{13}$.
\begin{equation}
 \begin{pmatrix}
  d' \\ s' \\ b'
 \end{pmatrix} = V_\text{CKM}  \begin{pmatrix}
  d \\ s \\ b
 \end{pmatrix},
\end{equation}
\noindent
where $V_\text{CKM}$ can be parameterised by $\lambda = \sin(\theta_C) \approx 0.2$ at leading order (LO) as
\begin{equation}
 V_\text{CKM} \approx \begin{pmatrix}
  1 & \lambda & \lambda^3\\
  -\lambda & 1 & \lambda^2\\
  -\lambda^3 & -\lambda^2 & 1
 \end{pmatrix}.
\end{equation}
The up-type quarks have no differences between their weak and mass eigenstates, but we could have played the same game with them leading to the 
same mixing, since $V_\text{CKM}$ is unitary. Looking at this matrix one can see that it is almost diagonal and hierachical which
means that the farther you leave the main diagonal the smaller become the magnitudes of the entries. It can be asked 
\\ \\ \textit{Flavour Changing Neutral Currents}\\
\noindent TO REVISE!
One problem of Cabibbo's theory is that a process $d'\bar {d'} \rightarrow Z$ would lead to an FCNC at tree level but is in fact highly suppressed. This
could be explained by the GIM-mechanism 


Now comes one of the basic questions motivating this thesis: Are these patterns random or is there rather an underlying broken family-symmetry which
could serve as a blindman's stick for the SM? We go with our guts and prefer the latter suggestion.


\subsubsection{Flavour Anomalies}
\label{sec_flAnom}
There are several observables in the SM which differ from the experimentally calculated expressions which request contributions from beyond the 
SM. We anticipate such contributions for three processes and therefore show how the experimentally compiled anomalous entities are compared with
the model expressions from section \ref{sec_pheno}.
\\ \\\textit{Anomalous magnetic moment of the muon}\\
\noindent
Every charged particle
may have a spin $s$ with an associated magnetic moment $ \mu = g \frac{e}{2m}s$ \cite{anomMom}.
The factor $g$ is called the gyromagnetic factor and is equal for all elementary particles of a kind. For fermions it is $g=2$. An anomalous
magnetic moment refers to a deviation value, expressed as $a=(g-2)/2$. Taking all deviations within the SM via loop effects into account, there
is still a discrepancy between SM and experiment 
\begin{align}
 \Delta a_\mu = a_\mu^\text{exp} - a_\mu^\text{SM} = 287(80)\cdot 10^{-11}.
\end{align}
Contributions to $a$ can be evaluated in the following way. An ingoing muon with momentum $p_1$ radiates a photon $A_\mu (q)$, leaving the muon in the final 
state with $p_2$. The matrix element for this process can be written as
\begin{align}
 \langle \mu_2|J^\mu(0)|\mu_1\rangle = \bar u_2 \left[F_D(q^2)\gamma^\mu + F_P(q^2)\frac{\ti \sigma^{\mu\nu}q_\nu}{2m}\right] u_1
 \label{eq_gordon}
\end{align}
which is the most general parametrisation one can write due to Ward identity and parity conservation in QED. The electric current is $J^\mu$ 
and $F_D$, $F_P$ are form factors from which the latter is used to calculate the deviation for the magnetic moment. Ideally, the model makes up
for the whole deviation. Hence,
\begin{align}
 \Delta a_\mu = F_P^\text{mod}(0).
\end{align}
\\ \\ \textit{Semileptonic four-fermion operators}\\
When hadrons are involved the methods of effective field theories (EFT) are used \cite{BurasEFT}. At their energy scale of order $\mathcal{O}$(1-10 GeV) processes
are expressed by an effective Hamiltonian which is a series of point-like vertices represented by local operators $O_i$ multiplied by effective 
couplings (Wilson coefficients) $C_i$. For $b\rightarrow s\bar\mu\mu$ transitions it can be written down as \cite{1411.3161}
\begin{align}
 \mathcal{H}^\text{eff} = -\frac{4G_F}{\sqrt{2}} V_{tb} V_{ts}^*\frac{e^2}{16\pi^2}\sum\limits_i (C_i(\mu)O_i(\mu) + C_i'(\mu)O_i'(\mu)) + \text{h.c.}
\end{align}
The crucial property of this is that this operator product expansion (OPE) allows a seperation of scales for calculating the process. At energies 
above a separation scale $\mu$ the perturbative calculation is encoded in the $C_i(\mu)$. Below this scale the matrix elements $\langle O_i(\mu)\rangle$
have to be calculated with non-perturbative methods. From the set of six-dimensional operators we consider NP effects in
\begin{align}
 O_9^{(')} &= (\bar s\gamma_\mu P_{L(R)}b)(\bar l\gamma^\mu l)\\
 O_{10}^{(')} &= (\bar s\gamma_\mu P_{L(R)}b)(\bar l\gamma^\mu\gamma_5 l).
\end{align}
Their Wilson coefficients ($C_i^\text{NP} = C_i - C_i^\text{SM}$) can be obtained by constructing a $\chi^2(\vec C^\text{NP})$ which contains the theory predictions for the observables,
the experimental central values and the corresponding (assumed to be Gaussian) uncertainties \cite{150306199}. For a left chiral model wherein the
NP states only couple to left-handed fermions the Wilson coefficients are thus derived as \cite{1608.07832}
\begin{align}
 C_9^\text{NP} = -C_{10}^\text{NP} \in [-0.81,-0.51]\quad \text{(at 1}\sigma), \label{eq_mumuBound1s}\\
 C_9^\text{NP} = -C_{10}^\text{NP} \in [-0.97,-0.37]\quad \text{(at 2}\sigma).
\end{align}
\\ \\ \textit{Four quark operators}\\
The observable relevant for $B_q$-mixing is $\Delta m_q$. By comparing its experimental value 
$\Delta m_s^\text{exp} = 1.1688(14) \cdot 10^{-11}$ GeV \cite{PDG} and the one from the SM 
$\Delta m_s^\text{SM} = 1.332(213)\cdot 10^{-11}$ GeV \cite{0612167} it appears that NP effect should have a destructive impact. 
With the SM Wilson coefficient $C_{B\bar B}^\text{SM} \simeq 8.2\cdot 10^{-5}$ TeV$^{-2}$ \cite{1608.07832} the determination of the NP contribution 
is estimated with 
\begin{align}
 \frac{\Delta m_s^\text{exp}}{\Delta m_s^\text{SM}}-1 = \frac{C^\text{NP}_{B\bar B}}{C^\text{SM}_{B\bar B}}
\end{align}
to be
\begin{align}
 C^\text{NP}_{BB} \in [-2.0,0.3] \cdot 10^{-5} \text{TeV}^{-2}.%\quad \text{(at 2}\sigma).
 \label{eq_mixBound}
\end{align}







\subsubsection{Effective Field Theory}
\cite{BurasEFT} simple derivation of EFT in SM as 4-fermi. Extraction of Wilson coefficients.
To get such a
low energy theory, one uses the framework of the Operator Product Expansion (OPE)


\subsection{DM in the $\Lambda$-CDM}

Phenomena on the largest scales in the universe are described by the Big Bang cosmology, parametrised
by the $\Lambda$-CDM model. It \cite{LambdaCDM} gives an explanation for, among other things, the energy density of the 
universe. The model's name refers to the sources adjusting the lack of energy considering only luminous matter. The $\Lambda$ points to the 
cosmological constant in Einstein's field equation
The CDM stands for cold dark matter, speaking particles moving slowly with respect to the speed of light (cold) and interacting very weakly with 
luminous matter (dark). In the following two hints for its existence are presented.
% The reasons for its existence and possible candidates satisfying the ensuing properties will be reviewed as well as the possiblities to test them.\\
\\ \textit{Rotational Curves}\\
\noindent The motion of stars is mainly influenced by their gravitational interactions \cite{LectDMLis}. From Newton's gravitation law, their circular 
velocity is derived as
$v(r)\propto\sqrt{M/r}$ with the radial distance $r$ from the centre of the galaxy and the enclosed mass $M$. For $r$ larger than the galactic 
disc, $M$ should remain constant and the velocity should drop $\propto r^{-1/2}$. But %(RubinFord,RobertsWhitehurst) 
it approximates a constant. Besides modified Newton dynamics \cite{11015122}\cite{160607790} there is strong evidence pointing at
a mass distribution even beyond the galactic
disc with $M(r)\propto r$ consisting of particles, or rather objects, which are not interacting electromagnetically, i.e. they are not visible. The rotation curves suggest a 
spherically symmetric ($\rho_\text{DM}\propto 1/r^2$) halo that also implies that this dark matter does not interact strongly with itself as the visible 
baryonic matter which collapses to the observed disc. To serve for this issue, the existence of DM is coherent but with this observation, its total 
amount in the 
universe cannot be extracted. It should be stressed that the mass distribution does not have to be created by just one particle species. Baryonic dark matter
and the SM neutrinos (may) also contribute. \\
\\ \textit{Cosmic Microwave Background}\\
\noindent The CMB is a major affirmation for the Big Bang theory. It emerged due to an effect called recombination. 
To this point, radiation, electrons and nucleons made up a plasma and the photons were scattering a lot off the actual free electrons. At some point it was 
energetically favoured for the electrons and nucleons to build light elements and the density of free electrons decreased which lead eventually to the decoupling
of the photons, meaning that they could travel freely without interacting. This residual amount of radiation has a wavelength in the microwave scale 
and is very isotropic and follows the spectrum of a black body \cite{DM-EvCaDo}. The observed anisotropics analysed by several CMB experiments 
(WMAP \cite{1212.5226}, ACBAR \cite{0303515}, CBI \cite{0205388}) lead to a discrepancy
between the abundance of baryonic and matter in total. We will now review the computation of the relic abundance and afterwards the detection type
of nucleon scattering.



\subsubsection{WIMP Relic Density}


\subsubsection{Detection Methods}
\textit{Direct Detection} \\
\\ \textit{Indirect Detection} \\
\\ \textit{Collider Signatures}

\section{Group Theory and Flavour Symmetries}
\subsection{Group Invariance}
symmetry breaking,representations
\subsection{$U(1)_\text{FN}$ and Continuous Groups}
\label{sec_FNGT}
 \textit{Lie-Groups}\\
\\ \textit{Clebsch Gordan coefficients} \\
\\ \textit{Froggat Nielsen-formalism}

\subsection{$A4$ and Discrete Groups}
\label{sec_A4GT}
\textit{General Properties}\\
finite amount of representations; characters; 
 \begin{align}
  \sum m_n n^2 = N_G \quad groupelements\\
  \sum m_n = \#_{IRR} \quad C = Irred\,Reps\\
  \varphi:\, G\rightarrow \text{GL}(V)\\
  g\mapsto \varphi(g):\, V\rightarrow V\\
  \chi_D(g) = \text{tr}D(g)\\
  \sum_g \chi_\alpha(g)^*\chi_\beta(g) = N_G \delta_{\alpha\beta}\\
  \sum_\alpha \chi_\alpha(g)^*\chi_\alpha(h) = \frac{N_G}{n_g} \delta_{C_g C_h} \stackrel{\Lambda}{=} \langle \chi^\mu, \chi^\nu \rangle = \delta^{\mu\nu}\\
  \text{FS}(R) := \frac{1}{N_G} \sum_g \chi_R(g^2) =\begin{cases}
                                                     1, & \text{real}\\
                                                     0, & \text{complex}\\
                                                     -1, & \text{pseudoreal}\\
                                                    \end{cases}\\
  \mu(k) = \langle \chi_R \cdot \chi_{R'} , \chi_{R_k} \rangle \quad tensor product decomposition\\
 \end{align}
 \\ \\ \textit{$A_4\times Z_3$}\\
 tetrahedron; link to $T'$ just because its nice; smallest group with $\boldsymbol{3}$; PMNS matrix; $\theta_{13}$
 seperation of leptons and neutrinos

\section{Model Outline}
\textit{The Underlying Model}\\
% \noindent  Now that we know the actors on the stage, we can build a model trying to reveal their behaviour altogether. 
\noindent We follow the model constructed
in \cite{Grip}. A renormalisable theory is built which explains the flavour anomalies associated with the $b$ and the $\mu$ at loop level. To achieve this, 
they introduce two new scalar fields $\Phi_l$ and $\Phi_q$ and a single fermion field $\chi$ which couple to SM-fermions via Yukawa interactions. 
The conditions to reduce the amount of possible charge assignments are:
\begin{enumerate}
 \item The conservation of baryon and lepton number and the prevention of additional sources of flavour violation should be ensured.
 \item Scalar couplings involving the Higgs as $\Phi H H$ or $\Phi H H H$, which could modify the observed Higgs phenomenology, should be prevented.
 \item $\Phi_q$ should be a scalar since it has weaker bounds on its mass.
 \item Only the BSM fields sould be charged non-trivially under an additional $U(1)$ symmetry which would lead to loop supression for all NP 
 flavour violating processes and would leave the lightest NP state (LP) stable. This LP should further be uncolored and electrically neutral 
 \item The $SU(2)_L$ representations for the BSM fields should have a dimension less than five.
\end{enumerate}
With these constraints some isospin representations of $\chi$ and $\Phi_l$ are rejected which could lead to unwanted renormalisable interactions, 
but it is stated that the LP would not fit the relic density and/or the direct detection bounds through a large $Z$-coupling, although the anomalies 
are well explained. \\ \\
\noindent \textit{Charge assignments}\\ \noindent
The addition of a flavour group $\mathcal{G} = U(1)_\text{FN}\times A_4 \times Z_3$ \cite{FerA4}\cite{VarzTotMod} prohibits
these unwanted interactions by charging the respective fields. The isospin representations themselves are therefore again enabled. Besides this,
the mass and mixing patterns of the SM fermions and the correct DM phenomeology are obtained. 
The downside herein is the surrender of renormalisability. NP is expected to couple chirally and global fits prefer the 4 fermion operator 
$\bar b_L \gamma^\nu s_L \mu_L \gamma_\nu \mu_L$. $SU(2)_L$ representations of NP yielding isospin invariance are considered and where the LP in
$\chi$ has vanishing hypercharge in order to reduce its coupling to the $Z$. This can be achieved by a singlet and a triplet representation while
the scalars are doublets.
Besides the focus on heavy quarks, anomalies including the muon as its anomalous magnetic moment and the discrepancy in  
$b\rightarrow s\bar\mu\mu$-like processes demand a special treatment in order to explain them.
To achieve this, $\Phi_l$ has to only couple to muons which can be ensured by assigning a suitable $A_4$ representation to it. 
% Anomalous tauonic transitions are also occuring but cannot be explained with this framework.
\begin{table}[t]
 \begin{tabular}{c|c|c|c}
%$SU(3)_C\times SU(2)_L\times U(1)_{Y_W}$
  Field & $\mathcal{G}_\text{SM}$ & $A_4 \times U(1)_\text{FN} \times Z_3$ & $U(1)_{B'}\times U(1)_{L'}\times U(1)_\chi$\\
  \hline
  $Q^i_L$ & (3,2,$\frac16$) & (1,$\Upsilon_{Q_i}$,$\omega$) & ($\frac13$,0,0)\\
  $U^i_R$ & (3,1,$\frac23$) & (1,$\Upsilon_{U_i}$,$\omega^2$)& ($\frac13$,0,0)\\
  $D^i_R$ & (3,1,$-\frac13$) & (1,$\Upsilon_{D_i}$,$\omega^2$)& ($\frac13$,0,0)\\
  $L^i_L$ & (1,2,$-\frac12$) & (3,0,$\omega$)& (0,1,0)\\
  $E^i_R$ & (1,1,$-1$) & ($1 {^(} {'} {^,} '' {^)} $,$\Upsilon_{E_i}$,$\omega^2$)& (0,1,0)\\
  $H$ & (1,2,$\frac12$) & (1,0,1)& (0,0,0)\\
  \hline
  $\chi$ & (1,1,0) & (1,0,$\omega$)& (0,0,1)\\ %bar chi has omega**2 (?) ->Z3invariance
 & (1,3,0) & (1,0,$\omega$)&(0,0,1)\\
  $\Phi_l$ & (1,2,$\frac12$) & ($1'$,0,1)& (0,-1,1)\\
  $\Phi_q$ & (3,2,-$\frac16$) & ($1$,0,1)& ($-\frac13$,0,1)\\
%   \hline
%   $\Phi_T$ & (1,1,0) & ($3$,0,1)& (0,0,0)\\
%   $\theta$ & (1,1,0) & (1,-1,0) & (0,0,0)
 \end{tabular}
\caption{Transformation rules for the SM and BSM fields. $i=1,2,3$ denotes a family index. The two rows for $\chi$ denote a singlet and a triplet, respectively. For the charges under 
$U(1)_\text{FN}$ and the representations of $E_R$ under $A_4$ see \eqref{eq_fnchargesQ} and table \ref{tab_a4charges}.}
\label{tab_models}
\end{table}
The accidental symmetries in the last column in table \ref{tab_models}
enforce stability of the proton and prevent contributions to other baryon and/or lepton number violating
processes. Additionally, the $U(1)_\chi$ stabilises the LP which is a crucial premise for a DM candidate. The $Z_3$ charges of the BSM fields are
chosen such interactions only with the fermionic SM doublets are valid.
When a fermionic multiplet only has $SU(2)_L\times U(1)_Y$ gauge interactions in the SM and its components have an identical tree-level mass, the charged 
ones become heavier than the neutral one due to quantum loop corrections of $\mathcal{O}$(100 MeV) \cite{Hisano}\cite{minMatter}. Chosing 
the hypercharge of $\chi$ to be 0 for both representations the DM particle can be a Majorana fermion. Eventually, 
the interaction Lagrangian for the new particles reads as
\begin{align}
 \mathcal{L} = g_i^q \bar \chi_R Q_L^i \Phi_q + g_i^l \bar \chi_R L_L^i \Phi_l + \text{h.c.}.
 \label{eq_modelLagrangian}
\end{align}
% The kinetic terms of the new particles are not shown here since their structure is not of interest in this thesis. 
$CP$-violating terms are not considered. The couplings $g^{q,l}$ are estimated from \eqref{eq_quarkyukawa} and \eqref{eq_muonmass}.
b-anomalies with eft neads chiral coupling C9, C10

\section{Phenomenological Analysis}
Now we want to check the two models of our DM field $\chi$ being a singlet or a triplet under the $SU(2)_L$, respectively, on five processes. Three of them
are indirect ones where the NP states are only virtually involved. To fit their respective constraints on the contribution from NP, we get bounds on 
the dark matter mass and crosscheck the results with the two remaining ones, the annihilation and the nucleon scattering. In what follows the mass
of the $\chi$ is just $m$ and the parameter of interest, the masses of the scalars are $M_l$ or $M_q$, respectively. Furthermore we introduce the fractions
$x_l=\frac{M_l}{m}$ and $x_q = \frac{M_q}{m}$. 

\subsection{Anomalous Magnetic Moment of the Muon}
We start with the anomalous magnetic moment of the muon since its parameter set only consists of 
\begin{align}
 \left(m, M_l, g_2^l\right).
\end{align}
In general, every charged particle
may have a spin $\boldsymbol{s}$ with an associated magnetic moment \cite{anomMom}
\begin{align}
 \boldsymbol{\mu} = g \frac{e}{2m}\boldsymbol{s}.
\end{align}
The factor $g$ is called the gyromagnetic factor and is equal for all elementary particles of a kind. For fermions it is $g=2$. By saying, the magnetic 
moment is anomalous, one states a deviation from this value which is expressed as $a= (g-2)/2$. There are many contributions to this already within the SM
from QED or hadronic vacuum polarisation which cause most of the uncertainties (1606.06861) but all of them are quantum loop effects. Even when all of them are taken into account, there is still a discrepancy
between theory and experiment 
\begin{align}
 \Delta a_\mu = a_\mu^\text{ex} - a_\mu^\text{SM} = 287(80)\cdot 10^{-11}.
\end{align}
Contributions to $a$ can be evaluated in the following way. An ingoing muon with momentum $p_1$ radiates a photon $A_\mu (q)$, leaving the muon in the final 
state with $p_2$. To first order in the external field, the scattering amplitude is 
\begin{align}
 M = -\ti e \langle \mu_2|J^\mu(x=0)|\mu_1\rangle A_\mu(q),
\end{align}
with the electic current $J^\mu(x)$. The most general parametrisation one can write thanks to Ward identity and parity conservation in QED is for on-shell
($p_i^2 = m_i^2$) external muons
\begin{align}
 \langle \mu_2|J^\mu(0)|\mu_1\rangle = \bar u_2 \left[F_D(q^2)\gamma^\mu + F_P(q^2)\frac{\ti \sigma^{\mu\nu}q_\nu}{2m}\right] u_1.
\end{align}
In the non-relativistic limit, the relation between the magnetic moment and these form factors $F_D$ and $F_P$ can be derived to be
\begin{align}
 \mu = \frac{e}{2m}\left(F_D(0) + F_P(0)\right)
\end{align}
\noindent
with $F_D(0)=1$ but $F_P(0)\neq0$. \cite{Lavoura} provides handy formulae for general one loop electroweak decays as $f_1 \rightarrow f_2 \gamma$. Therein
is $F_P(0) =m_\mu\left( \sigma_L P_L + \sigma_R P_R\right)$ with the $\sigma$s being sums of loop functions. The key process is depicted in figure \ref{pic_g-2}.
\begin{figure}[t]
 \includegraphics[width=0.6\textwidth]{../pics/g-2.eps}
 \caption{One-loop contribution to $a_\mu$. Depending on the representations, there are multiple possibilites to attach the photon.}
 \label{pic_g-2}
\end{figure}
Since we have a Yukawa interaction coupling to left handed particles, there are only terms left with $\lambda=|g_2^l|^2$, so that we eventually have
\begin{align}
 \Delta a_\mu^\text{NP} = \frac{|{g_2^l}|^2}{16\pi^2}\frac{m_\mu^2}{M_l^2}\left(Q^i_F \frac{1}{x_l}I(x_l^{-1}) + Q^i_B  I(x_l)\right)
\end{align}
with (1503.01500)
\begin{align}
 I(x) = \frac{1}{12(x-1)^4}\left(2+3x-6x^2+x^3+6x\log(x) \right).
\end{align}
The sum over $i$ is understood so that $Q_\mu = Q_F-Q_B$ is ensured.

there is an i in the electric current and an i in lavoura -> -1


\subsection{$B_s\rightarrow \mu\mu$}
\begin{figure}[t]
 \includegraphics[width=0.6\textwidth]{../pics/bsmumu.eps}
 \caption{bs->mumu}
 \label{pic_Bsmumu}
\end{figure}
After we first focussed on a pure leptonic process, we add a quark pair. Box diagrams of the kind as in picture \ref{pic_Bsmumu} are usually not calculated 
all the way down to cross sections but one rather uses the formalism of effective field theory since the typical hadronic energy scale is of
$\mathcal{O}$(1 GeV) which is much lower than the weak scale $\mathcal{O}$(100 GeV) that is also the expected mass scale of our NP states. 
\\ \\ \noindent \textit{Matrix Element}\\
\noindent So we can use 
the OPE framework where they are integrated out and the external masses and momenta are set to zero. The amplitude in the full theory is
\begin{align}
 M =\alpha_i^{q*} \alpha_j^{q*} \alpha_m^l \alpha_n^l\int \frac{\dx^4 q}{(2\pi)^4} \frac{q^\rho}{q^2-m^2}\Delta_q \Delta_l \frac{q^\sigma}{q^2-m^2} \left(\bar L_L^n \gamma_\rho Q_L^j\right) \left(\bar Q_L^i \gamma_\sigma L_L^m\right)
 \label{eq_matElemBSmumu}
\end{align}
with $q$ as the internal momentum and the scalar propagators $\Delta = \frac{1}{q^2-M^2}$ and the $\ti\epsilon$ terms in the denominators are already omitted
due to Wick-rotation later on when the integral is performed. The reasons why $m$ in the fermion propagators is missing
and why there is a Lorentz structure in the couplings, are connected. Since we have a model which couples only to left handed particles $\psi_L = P_L \psi$,
the fermion currents look like $\bar \psi_L (\slashed{q}+m)\varphi_L = \bar \psi P_R (q^\mu \gamma_\mu+m) P_L\varphi$. Further
$\{P_{L,R},\gamma^\mu\} = \gamma^\mu P_{R,L}$, $P_{L,R}^2 = P_{L,R}$ and $P_R P_L = 0$, so we are left over with the four fermion expression in 
\eqref{eq_matElemBSmumu}. We can also reexpress $q^\rho q^\sigma = q^2 g^{\rho\sigma}/4$ under the integral. This results from spherical coordinates where
every combination $\rho\neq\sigma$ would include at least one uneven angular function. When the spherical integral is performed, they would lead to zero. The 
4 in the denominator is the spacetime dimension. The spherical integral $\dx \Omega$ itself just gives a $2\pi^2$ and we are left with the momentum integral
which is not problematic since it does not diverge
\begin{align}
 M \propto \frac{1}{32\pi^2} \int\limits_0^\infty \dx q \frac{q^5}{(q^2-m^2)^2(q^2-M_l^2)(q^2-M_q^2)} = \frac{K(x_q,x_l)}{64\pi^2 m^2}
\end{align}
with
\begin{align}
 K(x,y) &= \frac{K(x)-K(y)}{x-y}\\
 K(x)&=\frac{1-x+x^2\log(x)}{(x-1)^2}.
\end{align}
In the SM, the fermion currents do not combine quarks and leptons. To compare the NP contribution to this process, we have to rearrange the fermion currents,
so that we have the same structure. This will be done by the Fierz identities.
\\ \\ \noindent \textit{Fierz Identities}\\
\cite{Fierz}
\begin{align}
 e_S(1234) = 1\cdot e_V(3214)
\end{align}
gives only a factor 1 (chiral).
\\ \\ \noindent \textit{Additional $SU(2)_L$-invariant term}\\
\noindent The operator $O_\delta=\left(\bar Q_L^i \gamma_\rho Q_L^j\right)\left(\bar L_L^m \gamma^\rho L_L^n\right)$ is not the only one which is a singlet
in the isospin space but $O_\tau=\left(\bar Q_L^i\vec \tau \gamma_\rho Q_L^j\right)\left(\bar L_L^m\vec \tau \gamma^\rho L_L^n\right)$ as well which enables
charged currents. The eventual operator is a superposition of these two. There might be several ways of computing the model dependent, relative 
prefactor $X$ of $O_\tau$  but one possibility is to regard the particles at each vertex as isospin multiplets. Since we want the vertices to be 
$SU(2)_L$-invariant, we 
decompose the product and focus on the singlet. This will be done for all four vertices per diagram which are multiplied and then added up for every diagram,
one can write down using these operators. For model A, this calculus can be done equivalently but can be skipped, because no charged currents as 
$b \bar u \rightarrow \mu \bar \nu$ are not induced, thus the prefactor of $O_\tau$ has to be zero and therefore 
\begin{align}
  \mathcal{L}_\text{eff} \supset \frac{K(x_q,x_l)}{m^2}\frac{\alpha_i^{q*} \alpha_j^{q*} \alpha_m^l \alpha_n^l}{64\pi^2} \times O_\delta.
 \label{eq_LagBSmumuModA}
\end{align}
In the case of higher isospin multiplets, processes of the just mentioned kind occur and thus we have three distinct types, with respect to their prefactor
that can be checked by expanding $O_\delta + X O_\tau$, namely $\bar d d\rightarrow \bar l l$, $\bar d d \rightarrow \bar\nu \nu$ and 
$d \bar u\rightarrow l\bar\nu$. 


TODO: CG-coefficients
\begin{align}
 \mathcal{L}_\text{eff} \supset \frac{K(x_q,x_l)}{m^2}\frac{\alpha_i^{q*} \alpha_j^{q*} \alpha_m^l \alpha_n^l}{64\pi^2}\times\left(O_\delta + \frac23 O_\tau\right).
 \label{eq_LagBSmumuModB}
\end{align}




\subsection{$B_s$-Mixing}
$KK$ and $DD$ also possible but higher suppressed due to couplings
\begin{align}
 \mathcal{L}_\text{eff} \supset \frac{K'(x_q)}{m^2}\frac{\alpha_i^{q*} \alpha_j^{q*} \alpha_m^l \alpha_n^l}{128\pi^2}\times\left(O_\delta + \frac23 O_\tau\right).
 \label{eq_LagBSmixModB}
\end{align}


\subsection{DM Annihilation}
$VV$ annihilation (9207234), wino RD from other su2-states (0913-main)
\begin{align}
 \langle \sigma v \rangle^\text{D} \left(\bar \chi \chi \rightarrow \bar f f\right) = \frac{N_c g_f^4 m^2}{32\pi^2\left(M_l^2 + m^2\right)^2}
\end{align}
\begin{align}
 \langle \sigma v \rangle^\text{M} \left(\bar \chi \chi \rightarrow \bar f f\right) = \frac{N_c g_f^4 m^2 \left(M^4+m^4 \right)}{48\pi^2\left(M_l^2 + m^2\right)^4}
\end{align}


\subsection{Direct Detection}
fierzing->vector current ->vanishes for Majorana

\section{Results}
supersymmetry
\subsection{A - $T(\chi)=\boldsymbol{1}$}
binolike
\subsection{B - $T(\chi)=\boldsymbol{3}$}
winolike
\section{Conclusion and Prospects}

% dfasdölkj\\d\\d\\d\\d\\d\\d\\d\\d\\d\\d\\d

% \section{DM candidate coupling to light quarks in T' framework}
% \subsection{Model construction}
% % \subsubsection{Grouptheoretical properties}
% \begin{align}
%  \sum m_n n^2 = N_G \quad groupelements\\
%  \sum m_n = \#_{IRR} \quad C = Irred\,Reps\\
%  \varphi:\, G\rightarrow \text{GL}(V)\\
%  g\mapsto \varphi(g):\, V\rightarrow V\\
%  \chi_D(g) = \text{tr}D(g)\\
%  \sum_g \chi_\alpha(g)^*\chi_\beta(g) = N_G \delta_{\alpha\beta}\\
%  \sum_\alpha \chi_\alpha(g)^*\chi_\alpha(h) = \frac{N_G}{n_g} \delta_{C_g C_h} \stackrel{\Lambda}{=} \langle \chi^\mu, \chi^\nu \rangle = \delta^{\mu\nu}\\
%  \text{FS}(R) := \frac{1}{N_G} \sum_g \chi_R(g^2) =\begin{cases}
%                                                     1, & \text{real}\\
%                                                     0, & \text{complex}\\
%                                                     -1, & \text{pseudoreal}\\
%                                                    \end{cases}\\
%  \mu(k) = \langle \chi_R \cdot \chi_{R'} , \chi_{R_k} \rangle \quad tensor product decomposation\\
% \end{align}
% 
% \subsubsection{Assigning particles to multiplets}
% \begin{align}
%  \langle \xi'' \rangle = \delta u''\approx \epsilon^4 \Lambda\\
%  M_{\xi''} = k \delta u'' 
% \end{align}
% 
% \subsection{Messenger $\xi''$ mediating SM and DM}
% \subsubsection{DM annihilation into light mesons}
% \subsubsection*{DM stability}
% page 24 - check on T, Z3 reps and spin
% \subsubsection*{cross section}
% \begin{align}
%  \sigma(\chi\chi \rightarrow d \bar s) = \frac{\lambda_f^2\lambda_\chi^2}{8\pi(4m_\chi^2 - M_{\xi''}^2)^2}(4m_\chi^2 - (m_d+m_s)^2)\\
%  \langle \sigma_{\text{Ann}} v \rangle = \sigma(1+\frac18
% \end{align}
% 
% \subsubsection{Meson-Mixing}
% \begin{align}
%  \Xi_{dd'} = \begin{pmatrix}
%               0 & 1 & 0\\
%               1 & 0 & 0 \\
%               0 & 0 & 0
%              \end{pmatrix} \xrightarrow{massbasis} \begin{pmatrix}
% 						-\epsilon & 1 & -\epsilon^2\\
% 						1 & \epsilon & -\epsilon^3\\
% 						-\epsilon^2 & \-\epsilon^3 & \epsilon^5
% 						\end{pmatrix}
% \end{align}
% 
% \subsubsection{DM-nucleus scattering}
% 
% 
% \addcontentsline{toc}{section}{List of figures}
% \newpage\listoffigures\newpage
% \addcontentsline{toc}{section}{List of tables}
% \listoftables\newpage
\end{spacing}
\newpage

\begin{thebibliography}{xxx}
 \bibitem[1]{Grip}B. Gripaios et al. \textit{Linear flavour violation and anomalies in $B$ physics} \href{http://arxiv.org/abs/1509.05020v1}{arxiv.org/abs/1509.05020v1}
 \bibitem[2]{Peskin}M. Peskin et al. \textit{An Introduction To Quantum Field Theory} ISBN: 0-201-50397-2
 \bibitem[3]{Fierz}J.F. Nieves et al. \textit{Generalized Fierz identities} \href{http://http://arxiv.org/abs/hep-ph/0306087v1}{arxiv.org/abs/hep-ph/0306087v1}
 \bibitem[4]{FerA4}G. Altarelli et al. \textit{Tri-Bimaximal Neutrino Mixing and Discrete Flavour Symmetries} \href{https://arxiv.org/abs/1205.5133}{arxiv.org/abs/1205.5133}
 \bibitem[5]{VarzTotMod}I.d.M. Varzielas et al. \textit{Clues for flavour from rare lepton and quark decays} \href{https://arxiv.org/abs/1503.01084}{arxiv.org/abs/1503.01084}
 \bibitem[6]{Hisano}J. Hisano et al. \textit{Direct Detection of Electroweak-Interacting Dark Matter} \href{https://arxiv.org/abs/1104.0228v2}{arxiv.org/abs/1104.0228v2}
 \bibitem[7]{minMatter}E.D. Nobile et al. \textit{Minimal Matter at the Large Hadron Collider} \href{http://arxiv.org/abs/0908.1567}{arxiv.org/abs/0908.1567}
 \bibitem[8]{anomMom}K. Melnikov et al. \textit{Theory of the Muon Anomalous Magnetic Moment} ISBN-13 978-3-540-32806-3
 \bibitem[9]{Lavoura}L. Lavoura \textit{General formulae for $f_1\rightarrow f_2 \gamma$} \href{http://arxiv.org/abs/hep-ph/0302221}{arxiv.org/abs/hep-ph/0302221}
 \bibitem[10]{BurasEFT}A.J. Buras \textit{Weak Hamiltonian, CP Violation and Rare Decays} \href{http://arxiv.org/abs/hep-ph/9806471}{arxiv.org/abs/hep-ph/9806471}
\end{thebibliography}
\end{document}
