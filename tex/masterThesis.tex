\documentclass[11pt,a4paper,twoside]{article}

% Deutsche Spracheinstellungen
\usepackage[english,english]{babel, varioref}
\usepackage[T1]{fontenc}
\usepackage[utf8]{inputenc}

%\usepackage{marvosym}

\usepackage{amsfonts}
\usepackage{amssymb}
\usepackage{amsmath}
\usepackage{amscd}
\usepackage{amstext}
\usepackage{float}
\usepackage{caption}
\usepackage{wrapfig}
\usepackage{setspace}
%\usepackage[onehalfspacing]{setspace}
\usepackage{threeparttable}
\usepackage{footnote}
\usepackage{feynmf}
\usepackage{bbm}
\usepackage{slashed}
\usepackage{textcomp}
\usepackage{multirow}
\usepackage{courier}
\usepackage{listings}
\usepackage{color}
%\usepackage{minipage}
 
 \definecolor{middlegray}{rgb}{0.5,0.5,0.5}
 \definecolor{lightgray}{rgb}{0.8,0.8,0.8}
 \definecolor{orange}{rgb}{0.8,0.3,0.3}
 \definecolor{yac}{rgb}{0.6,0.6,0.1}
 \definecolor{puple}{rgb}{0.62,0.12,0.94}
 \lstset{language=Python,
                basicstyle=\ttfamily,
                keywordstyle=\color{red}\ttfamily,
                stringstyle=\color{magenta}\ttfamily,
                commentstyle=\color{blue}\ttfamily,
                morecomment=[l][\color{blue}]{\#},
		%stepnumber=1,
		%numberstyle=\color{magenta}\ttfamily,
		%    numbers=left,
		%    numberstyle={},
		%    numberblanklines=false,
		%    stepnumber=1,
		%    numbersep=10pt,
		    xleftmargin=15pt,
 		moredelim=[is][\color{purple}]{|}{|}
}

\newfloat{formel}{htbp}{for}
\floatname{formel}{Formel}

\onehalfspacing
%\setstretch {1.433}

\usepackage{longtable}

%\usepackage{bibgerm}

\usepackage{footnpag}

\usepackage{ifthen}                 %%% package for conditionals in TeX
\usepackage[amssymb]{SIunits}
%Fr textumflossene Bilder und Tablellen
%\usepackage{floatflt} - veraltet

%Fr Testzwecke aktivieren, zeigt labels und refs im Text an.
%\usepackage{showkeys}

% Abstand zwischen zwei Abs�zen nach DIN (1,5 Zeilen)
% \setlength{\parskip}{1.5ex}

% Einrckung am Anfang eines neuen Absatzes nach DIN (keine)
%\setlength{\parindent}{0pt}

% R�der definieren
% \setlength{\oddsidemargin}{0.3cm}
% \setlength{\textwidth}{15.6cm}

% bessere Bildunterschriften
\usepackage{caption2}


% Probleml�ungen beim Umgang mit Gleitumgebungen
\usepackage{float}

% Nummeriert bis zur Strukturstufe 3 (also <section>, <subsection> und <subsubsection>)
%\setcounter{secnumdepth}{3}

% Fhrt das Inhaltsverzeichnis bis zur Strukturstufe 3
%\setcounter{tocdepth}{3}

\usepackage{exscale}

\newenvironment{dsm} {\begin{displaymath}} {\end{displaymath}}
\newenvironment{vars} {\begin{center}\scriptsize} {\normalsize \end{center}}


\newcommand {\en} {\varepsilon_0}               % Epsilon-Null aus der Elektrodynamik
\newcommand {\lap} {\; \mathbf{\Delta}}         % Laplace-Operator
\newcommand {\R} { \mathbb{R} }                 % Menge der reellen Zahlen
\newcommand {\e} { \ \mathbf{e} }               % Eulersche Zahl
\renewcommand {\i} { \mathbf{i} }               % komplexe Zahl i
\newcommand {\N} { \mathbb{N} }                 % Menge der nat. Zahlen
\newcommand {\C} { \mathbb{C} }                 % Menge der kompl. Zahlen
\newcommand {\Z} { \mathbb{Z} }                 % Menge der kompl. Zahlen
\newcommand {\limi}[1]{\lim_{#1 \rightarrow \infty}} % Limes unendlich
\newcommand {\sumi}[1]{\sum_{#1=0}^\infty}
\newcommand {\rot} {\; \mathrm{rot} \,}         % Rotation
\newcommand {\grad} {\; \mathrm{grad} \,}       % Gradient
\newcommand {\dive} {\; \mathrm{div} \,}        % Divergenz
\newcommand {\dx} {\; \mathrm{d} }              % Differential d
\newcommand {\cotanh} {\; \mathrm{cotanh} \,}   %Cotangenshyperbolicus
\newcommand {\asinh} {\; \mathrm{areasinh} \,}  %Area-Sinus-Hyp.
\newcommand {\acosh} {\; \mathrm{areacosh} \,}  %Area-Cosinus-H.
\newcommand {\atanh} {\; \mathrm{areatanh} \,}  %Area Tangens-H.
\newcommand {\acoth} {\; \mathrm{areacoth} \,}  % Area-cotangens
\newcommand {\Sp} {\; \mathrm{Sp} \,}
\newcommand {\mbe} {\stackrel{\text{!}}{=}}     %Must Be Equal
\newcommand{\qed} { \hfill $\square$\\}
\newcommand{\midtilde}{\raisebox{-0,25\baselineskip}{\textasciitilde}}
\renewcommand{\i} {\imath}
\def\captionsngerman{\def\figurename{\textbf{Abb.}}}

%%%%%%%%%%%%%%%%%%%%%%%%%%%%%%%%%%%%%%%%%%%%%%%%%%%%%%%%%%%%%%%%%%%%%%%%%%%%
% SWITCH FOR PDFLATEX or LATEX
%%%%%%%%%%%%%%%%%%%%%%%%%%%%%%%%%%%%%%%%%%%%%%%%%%%%%%%%%%%%%%%%%%%%%%%%%%%%
%%%
\ifx\pdfoutput\undefined %%%%%%%%%%%%%%%%%%%%%%%%%%%%%%%%%%%%%%%%% LATEX %%%
%%%
\usepackage[dvips]{graphicx}       %%% graphics for dvips
\DeclareGraphicsExtensions{.eps,.ps}   %%% standard extension for included graphics
\usepackage[ps2pdf]{thumbpdf}      %%% thumbnails for ps2pdf
\usepackage[ps2pdf,                %%% hyper-references for ps2pdf
bookmarks=true,%                   %%% generate bookmarks ...
bookmarksnumbered=true,%           %%% ... with numbers
hypertexnames=false,%              %%% needed for correct links to figures !!!
breaklinks=true,%                  %%% breaks lines, but links are very small
linkbordercolor={0 0 1},%          %%% blue frames around links
pdfborder={0 0 112.0}]{hyperref}%  %%% border-width of frames
%                                      will be multiplied with 0.009 by ps2pdf
%
\hypersetup{ pdfauthor   = {Dimitrios Skodras},
pdftitle    = {Fermionic Dark Matter and its Role on B Anomalies}, pdfsubject  = {masterthesis}, pdfkeywords = {dark matter},
pdfcreator  = {LaTeX with hyperref package}, pdfproducer = {dvips
+ ps2pdf} }
%%%
\else %%%%%%%%%%%%%%%%%%%%%%%%%%%%%%%%%%%%%%%%%%%%%%%%%%%%%%%%%% PDFLATEX %%%
%%%
\usepackage[pdftex]{graphicx}      %%% graphics for pdfLaTeX
\DeclareGraphicsExtensions{.pdf}   %%% standard extension for included graphics
\usepackage[pdftex]{thumbpdf}      %%% thumbnails for pdflatex
\usepackage[pdftex,                %%% hyper-references for pdflatex
bookmarks=true,%                   %%% generate bookmarks ...
bookmarksnumbered=true,%           %%% ... with numbers
hypertexnames=false,%              %%% needed for correct links to figures !!!
breaklinks=true,%                  %%% break links if exceeding a single line
linkbordercolor={0 0 1},
linktocpage]{hyperref} %%% blue frames around links
%                                  %%% pdfborder={0 0 1} is the default
\hypersetup{
pdftitle    = {Fermionic Dark Matter and its Role on B Anomalies}, %right place
pdfsubject  = {master thesis}, 
pdfkeywords = {V301, Innenwiderstand, Leistungsanpassung},
pdfsubject  = {Protokoll AP},
pdfkeywords = {V301, Innenwiderstand, Leistungsanpassung}}
%                                  %%% pdfcreator, pdfproducer,
%                                      and CreationDate are automatically set
%                                      by pdflatex !!!
\pdfadjustspacing=1                %%% force LaTeX-like character spacing
\usepackage{epstopdf}
%
\fi %%%%%%%%%%%%%%%%%%%%%%%%%%%%%%%%%%%%%%%%%%%%%%%%%%% END OF CONDITION %%%
%%%%%%%%%%%%%%%%%%%%%%%%%%%%%%%%%%%%%%%%%%%%%%%%%%%%%%%%%%%%%%%%%%%%%%%%%%%%
% seitliche Tabellen und Abbildungen
%\usepackage{rotating}
\usepackage{ae}
\usepackage{
  array,
  booktabs,
  dcolumn
}
\makeatletter 
  \renewenvironment{figure}[1][] {% 
    \ifthenelse{\equal{#1}{}}{% 
      \@float{figure} 
    }{% 
      \@float{figure}[#1]% 
    }% 
    \centering 
  }{% 
    \end@float 
  } 
  \makeatother 


  \makeatletter 
  \renewenvironment{table}[1][] {% 
    \ifthenelse{\equal{#1}{}}{% 
      \@float{table} 
    }{% 
      \@float{table}[#1]% 
    }% 
    \centering 
  }{% 
    \end@float 
  } 
  \makeatother 
%\usepackage{listings}
%\lstloadlanguages{[Visual]Basic}
%\allowdisplaybreaks[1]
%\usepackage{hycap}
%\usepackage{fancyunits}

\usepackage{xfrac}
\usepackage{xcolor}
\usepackage{setspace}\usepackage{threeparttable}
\usepackage{fancyhdr}
\newcommand{\ti}{\text{i}}
\fancyfoot{}
\fancyhead[RO,LE]{\thepage}
\fancyhead[LO]{\leftmark}
\fancyhead[RE]{\rightmark}


 %\setlength{\parskip}{1.5ex}
\begin{document}
\begin{spacing}{1,2}
\pagenumbering{Roman}

% Anmerkung: Die Seitenraender wurden asymmetrisch gewaehlt,
%            damit genug Platz fuer eine Klemmbindung da ist.
%            Da neue Kapitel auf der rechten Seite (ungerade
%            Seitennummer) beginnen sollten, muss ggf. am Ende
%            des vorhergehenden Kapitels eine Leerseite
%            eingefuegt werden:
%
%            \newpage
%            \thispagestyle{empty}
%            \ \\
%            \newpage
%
%            Die Seitenraender koennen aber auch in der Datei Tex/global.tex
%            veraendert werden.

% >>> Titelseite <<<

\newcommand{\thetitle}{Fermionic Dark Matter and its Role in B-anomalies}

\thispagestyle{empty}
\begin{center}

\Huge\textbf{\thetitle}
\vfill
% Note that the size is given in normal parentheses
% instead of curly brackets.
% Define external vertices from bottom to top
\vfill
\Large
Masterarbeit\\ zur Erlangung des akademischen Grades \\ Master of Science \\
\vspace{20pt}
\normalsize
vorgelegt von \\[5pt]
{\Large Dimitrios Skodras} \\[5pt]
geboren in Aschaffenburg \\
\vspace{20pt}
Lehrstuhl für Theoretische Physik IV \\ Fakultät Physik \\
Technische Universität Dortmund \\ 2016
\end{center}
\newpage
% % >>> Gutachterseite <<<
% \thispagestyle{empty}
% \newpage
% \cleardoublepage

\thispagestyle{empty}
\vspace*{\fill}
\begin{tabbing}
1. Gutachter : \=\kill
1. Gutachter : \>Prof. Dr. Gudrun Hiller \\[11pt]
2. Gutachter : \> \\[11pt]
\end{tabbing}
\vspace{11pt}
Datum des Einreichens der Arbeit: 31. Oktober, 2016
\newpage
\thispagestyle{empty}
\begin{flushright} 
\textit{\grqq Es gibt nichts Praktischeres, als eine gute Theorie.\grqq}\\
- \textit{Kant, Immanuel}\\
\vspace{2cm}
% \textit{``Was dürfen wir hoffen?''}\\
% - \textit{Kant, Immanuel}\\
% \vspace{2cm}
% \textit{``Kein Mensch ist so wichtig, wie er sich nimmt.''}\\
% - \textit{Kant, Immanuel}
\end{flushright}

\newpage


% >>> Kurzfassung/Abstract <<<

\thispagestyle{empty}
%Kuzfassung
 \section*{Kurzfassung}

 \section*{Abstract}

 \section*{Abrégé}
 \newpage

% >>> Hauptteil <<<

%\addcontentsline{toc}{chapter}{Inhaltsverzeichnis}
\end{spacing}
\begin{spacing}{1,1}
\tableofcontents\newpage

\end{spacing}
\begin{spacing}{1,2}

\thispagestyle{empty}
 \cleardoublepage

\pagenumbering{arabic}
\setcounter{page}{1}
\pagestyle{fancy}

\section{Introduction}
After the prediction (1964) (CITE) and the experimental verification (2012) (CITE) of the
now called Higgs-boson, the Standard Model (SM) could be considered self-
contained. The forecasts it makes for processes described therein are proven
remarkably well by various experiments. At least at the energy scales they
are currently operating. That said, it already contains associations that the
contributions to this magnificent theory, made by our scientific forefathers, might not be
the end of the story. We might know the Higgs mechanism being source of mass of visible particles,
but the reason for their absolute values is still notional. The fermions thereof can be put into
small groups and are considered to be interacting equally with the gauge bosons but it appears
that some break ranks. Not only in the elementary particle physics sphere but also in cosmology
strange things happen. From observation it seems that there has to be a mass distribution of
invisible particles, say dark matter, that lead to the actual orbits of the stars in the galaxy.
There is of course even more, but the resolution of these three issues in one combined theory is
the task of this thesis.

\noindent To resolve a problem, one has to understand it first. So we start with some notes on 
related topics in the SM and cosmology.


% It still lacks for example a quantum field theoretical
% description for gravity or does not contain the experimentally shown masses
% of the neutrinos. Besides, its dynamics depend on 19 parameters, including
% the masses of the fermions, which are quite a lot and are wanted to decrease
% in number by finding relations among one another. Furthermore cosmology
% tells us that the SM is only covering about twenty per cent of the mass content
% of the universe where the rest is made of the nowadays called dark matter
% (DM). Since this thesis is about flavour and dark matter, we
% should have a closer look on the way things stand at the moment.


\section{The Guide to present physics}


\subsection{Flavour in the SM}
% The Standard Model is a theory that contains particle fields, ordinary matter is composed of and messenger fields mediating interactions between them. 
% These fields can be separated by their quantum numbers indicating their couplings to each other. At first we have the twelve gauge vector boson fields, 
% eight gluon fields $G_a$ for strong interactions with coloured ($C$) particles and four electroweak boson fields $W_1$, $W_2$, $W_3$ and $B$ which couple
% to particles with weak isospin $T$ and hypercharge $Y_W$ respectively. Now we have 24 fermion fields, six leptons and six quarks each carrying one of three colours.
% The leptons can be subdivided into three electrically ($Q=T+Y_W$) charged and three uncharged ones and the quarks into three up-type ($T=\sfrac12$) and
% three down-type ($T=-\sfrac12$). Last but not least, the already mentioned scalar Higgs-field $\phi$ which develops a non vanishing vacuum expectation value
% (vev) $v$. This is the source for the spontanious breaking of the electroweak symmetry into the electromagnetic 
% (NOT)
The Standard Model is a gauge quantum field theory whose internal symmetry is the unitary product group $SU(3)_C\times SU(2)_L\times U(1)_{Y_W}$ representing
the quantum chromodynamics (QCD) whose charge is called color $C$ and the electroweak theory (GSW-Theory) whose charges are the weak isospin $T$ 
(hold by left handed particles) and the weak hypercharge $Y_W$. These quantum numbers (QN) are carried by a set of particle fields, ordinary matter is composed
of and messenger fields mediating these interactions between them. In particular twelve gauge vector boson fields, eight gluon fields $G_a$ for strong 
interactions and four electroweak boson fields from which $W_1$, $W_2$, $W_3$ couple to the weak isospin and $B$ to the weak hypercharge. Now we have 12
fermion fields, six leptons $e$, $\mu$, $\tau$ as well as their respective neutrinos $\nu$ and six quarks $u$, $d$, $s$, $c$, $b$, and $t$. Actually there
are even more since they are defined by their QN. So for each fermion there is a distinction drawn between left handed $f_L$ ($T=\sfrac12$) and right 
handed $f_R$ ($T=0$), although $\nu_R$ are not considered in the SM. Furthermore there are three different colours for each quark. Well finally the field 
$\phi$ of the already mentioned scalar
Higgs-boson holds a special role in the SM. It is known that the boson fields $W^\pm$ and $Z$, responsible for weak processes, have nonzero masses. But their
mass terms would break the gauge invariance of the lagrangian. So the Higgs-mechanism was considered wherein $\phi$ develops a non vanishing vacuum expectation
value $v$ (vev) which breaks the electroweak symmetry spontaniously down to the electromagnetic symmetry $U(1)_Q$ resulting in a still massless photon field 
$A$ and the three just named massive ones.
\subsubsection{Yukawa interaction}
Not only the bosons get their masses from this mechanism but the fermions as well - at least the electrically charged ($Q$) ones - which is represented by the
Yukawa (scalar-fermion interaction) term in the SM-lagrangian
\begin{equation}
 \mathcal{L}_{Y} = - y^u_{ij}\, \bar Q^i_L \, \phi^c\, u^j_R - y^d_{ij}\, \bar Q^i_L\, \phi\, d^j_R - y^e_{ij}\, \bar L^i_L\, \phi\, e^j_R + \text{h.c.}.
 \label{eq_yukawaSM}
\end{equation}
\noindent
$y^u$, $y^d$ and $y^e$ are $3\times 3$ (three generations) real, so called yukawa matrices and represent the 
coupling of the fermions to the Higgs. $\phi^c = \ti\sigma_2\phi^*$ is the charged conjugate Higgs. ``h.c.'' stands for hermetian conjugate so that it holds for the antiparticles as well. 
$Q_L$, $L_L$ and $\phi$ are doublets of the $SU(2)_L$, since they are built out of two fermion fields each holding a 
weak isospin of $T=\sfrac12$ with the 3rd component $T_3 = \pm\sfrac12$, e.g. $L^2_L = (\nu_{\mu\, L}, \mu_L)^T_{\textbf{2}}$ with the left handed muon-neutrino and the muon. 
Their counterparts $u_R$, $d_R$ and $e_R$ are singlets under the $SU(2)_L$ because they have no weak isospin, e.g. $e^2_R = (\mu_R)_{\textbf{1}}$, and hence don't take part
in the weak interaction mediated by the $W$-bosons. After 
the symmetry breaking and rotating the fermion fields in a basis where the yukawa matrices become diagonal, we can write down their mass terms 
\begin{equation}
 \mathcal{L}_m = -m^u_i \bar u'^i_L u'^i_R -m^d_i \bar d'^i_L d'^i_R -m^e_i \bar e'^i_L e'^i_R 
 \label{eq_massSM}
\end{equation}
\noindent
where $m^\alpha_i \sim y^\alpha_i \cdot v$ ($\alpha = u,d,e$). Here you can see that the SM does not distinguish between generations. 
It treats a left handed up quark the same way it treats a left handed charm or a left handed top. So if the eigenvalues of $m^\alpha$ would be degenerate, i.e. the masses would
be all the same, one would not have a method to differentiate them. 

% \vspace{-0.3cm}
You could ask
what happened with a term like $\nu_R$ in \eqref{eq_yukawaSM}. This would imply a particle which has neither a color charge, nor electrical charge, nor 
weak isospin and hence ($Q = T_3 + Y_W$) no hypercharge which means that the SM is totally blind to it. Furthermore it seemed obvious not to be
able to construct a neutrino mass since they were considered massless. But to answer the question we could add 
such a term somehow when it can be shown for example that righthanded neutrinos exist, which is likely due to the measurements of neutrino oscillations.

\subsubsection{Weak interaction}
Neutrino oscillation is a process where a neutrino of one generation changes its flavour while propagating through space described by the PMNS matrix. 
This suggests that 
flavour is not a conserved quantum number in nature unlike the electrical charge for example. As this phenomenon is still rather young (2001) and not implemented in the 
SM, we already have flavour violation (FV) therein, in the quark sector. There is a huge list of Mesons and Baryons (composites of one quark and one anti-quark,
or three quarks, respectively) decaying into others with different quark content enabled by the $W$-bosons via charged currents. 

\textit{Flavour Changing Charged Currents}\\
\noindent So it was thought, that the down-type quark mass
eigenstates $d'^i$ are superpositions of their interaction eigenstates $d^i$. Formerly started by Cabibbo and pursued by 
Kobayashi and Maskawa (1973) for CP-violation reasons (charge conjugation C, parity P), the CKM-matrix as an unitary $3\times 3$ matrix was invented,
which can be thought of as a rotation
matrix, rotating the weak eigenstates of the down-type quarks in the mass eigenstates by three Euler angles, $\theta_{12} = \theta_C$ called Cabibbo angle 
as well as $\theta_{23}$ and $\theta_{13}$.
\begin{equation}
 \begin{pmatrix}
  d' \\ s' \\ b'
 \end{pmatrix} = V_\text{CKM}  \begin{pmatrix}
  d \\ s \\ b
 \end{pmatrix},
\end{equation}
\noindent
where $V_\text{CKM}$ can be parameterised by $\lambda = \sin(\theta_C) \approx 0.2$ at leading order (LO) as
\begin{equation}
 V_\text{CKM} \approx \begin{pmatrix}
  1 & \lambda & \lambda^3\\
  -\lambda & 1 & \lambda^2\\
  -\lambda^3 & -\lambda^2 & 1
 \end{pmatrix}.
\end{equation}
The up-type quarks have no differences between their weak and mass eigenstates, but we could have played the same game with them leading to the 
same mixing, since $V_\text{CKM}$ is unitary. Looking at this matrix one can see that it is almost diagonal and hierachical which
means that the farther you leave the main diagonal the smaller become the magnitudes of the entries. It can be asked 

\textit{Flavour Changing Neutral Currents}\\
\noindent
One problem of Cabibbo's theory is that a process $d'\bar {d'} \rightarrow Z$ would lead to an FCNC at tree level but is in fact highly suppressed. This
could be explained by the GIM-mechanism 


Now comes one of the basic questions motivating this thesis: Are these patterns random or is there rather an underlying broken family-symmetry which
could serve as a blindman's stick for the SM? We go with our guts and prefer the latter suggestion.

\subsubsection{Effective Field Theory}
\cite{BurasEFT} simple derivation of EFT in SM as 4-fermi. Extraction of Wilson coefficients.
To get such a
low energy theory, one uses the framework of the Operator Product Expansion (OPE)

\subsection{DM in the $\Lambda$-CDM}
\subsubsection{WIMP Relic Density}
\subsubsection{Detection methods}
\section{Group Theory and Flavour Symmetries}
\subsection{Group Invariance}
symmetry breaking
\subsection{$U(1)_\text{FN}$ and Continuous Groups}
\label{sec_FNGT}
clebsch-gordan
\subsection{$A4$ and Discrete Groups}
\label{sec_A4GT}

\section{Model Outline}
\textit{The Underlying Model}\\
\noindent Now that we know the actors on the stage, we can build a model trying to reveal their behaviour altogether. We will follow basically the model constructed
in \cite{Grip}. They build a renormalisable theory which generates the anomalies associated with the $b$ and the $\mu$ at loop level. To do so, they 
introduce two new scalar fields $\Phi_l$ and $\Phi_q$ and a single fermion field $\chi$ which couple to SM-fermions via Yukawa interactions. 
Their conditions to reduce the amount of possible charge assignments for them are 
\begin{enumerate}
 \item the preservation of the accidental symmetries as baryon and lepton number and the prevention of other sources of flavour violation.
 \item scalar couplings with the Higgs as $\Phi H H$ or $\Phi H H H$ which could modify the observed Higgs phenomenology should be prevented.
 \item the coloured particle, required by quark interaction, should be a scalar since it will have weaker bounds on its mass.
 \item a $U(1)$ symmetry only interacting non trivially on the BSM fields has the advantage that all NP flavour-violating processes are loop suppressed. And
 moreover the lightest NP state (LP) would be stable that should be uncolored and electrically neutral. 
 \item the dimension of $SU(2)_L$ of the NP irreducibles is less then five.
\end{enumerate}
With these constraints they reject some representations of $\chi$ and $\Phi_l$ which could lead to unwanted renormalisable interactions, but they state 
that their LP would not fit the relic density and/or the direct detection bounds through the Z-coupling although the anomalies are 
well explained by the model. \\ \\
\noindent \textit{Charge assignment}\\ \noindent
The main feature of this thesis is the adding of a flavour group $\mathcal{G} = U(1)_\text{FN}\times A_4 \times Z_3$ \cite{FerA4}\cite{VarzTotMod}. 
It shall prohibit these dangerous interactions by charging the respective fields so that the representations themselves are again enabled. Besides this,
the flavour patterns and the correct DM phenomeology are obtained. 
The downside herein is the surrender of renormalisability. Since we expect that NP couples chirally and that the fits say that the 4 fermion operator 
$\bar b_L \gamma^\nu s_L \mu_L \gamma_\nu \mu_L$ is prefered, we consider $SU(2)_L$ representations of $\chi$ that yield invariance and wherein the lightest
state has a small or vanishing $T_3$ in order to reduce its coupling to the $Z$. This can be achieved in many ways depending on what you want to get in the end. Besides the focus on heavy quarks,
our interest is a special role for the muon in order to explain its anomalous magnetic moment and the discrepancy in $B_s\rightarrow \mu\mu$. 
To do so, we want $\Phi_l$ to only couple to muons which can be ensured by representing it suitably under $A_4$. \section{Model Outline}
\textit{The Underlying Model}\\
\noindent Now that we know the actors on the stage, we can build a model trying to reveal their behaviour altogether. We will follow basically the model constructed
in \cite{Grip}. They build a renormalisable theory which generates the anomalies associated with the $b$ and the $\mu$ at loop level. To do so, they 
introduce two new scalar fields $\Phi_l$ and $\Phi_q$ and a single fermion field $\chi$ which couple to SM-fermions via Yukawa interactions. 
Their conditions to reduce the amount of possible charge assignments for them are 
\begin{enumerate}
 \item the preservation of the accidental symmetries as baryon and lepton number and the prevention of other sources of flavour violation.
 \item scalar couplings with the Higgs as $\Phi H H$ or $\Phi H H H$ which could modify the observed Higgs phenomenology should be prevented.
 \item the coloured particle, required by quark interaction, should be a scalar since it will have weaker bounds on its mass.
 \item a $U(1)$ symmetry only interacting non trivially on the BSM fields has the advantage that all NP flavour-violating processes are loop suppressed. And
 moreover the lightest NP state (LP) would be stable that should be uncolored and electrically neutral. 
 \item the dimension of $SU(2)_L$ of the NP irreducibles is less then five.
\end{enumerate}
With these constraints they reject some representations of $\chi$ and $\Phi_l$ which could lead to unwanted renormalisable interactions, but they state 
that their LP would not fit the relic density and/or the direct detection bounds through the Z-coupling, although the anomalies are 
well explained by the model. \\ \\
\noindent \textit{Charge assignment}\\ \noindent
The main feature of this thesis is the adding of a flavour group $\mathcal{G} = U(1)_\text{FN}\times A_4 \times Z_3$ \cite{FerA4}\cite{VarzTotMod}. 
It shall prohibit these dangerous interactions by charging the respective fields so that the representations themselves are again enabled. Besides this,
the flavour patterns and the correct DM phenomeology are obtained. 
The downside herein is the surrender of renormalisability. Since we expect that NP couples chirally and that the fits say that the 4 fermion operator 
$\bar b_L \gamma^\nu s_L \mu_L \gamma_\nu \mu_L$ is prefered, we consider $SU(2)_L$ representations of $\chi$ that yield invariance and wherein the lightest
state has a small or vanishing $T_3$ in order to reduce its coupling to the $Z$. This can be achieved in many ways depending on what you want to get in the end. Besides the focus on heavy quarks,
our interest is a special role for the muon in order to explain its anomalous magnetic moment and the discrepancy in $B_s\rightarrow \mu\mu$. 
To do so, we want $\Phi_l$ to only couple to muons which can be ensured by representing it suitably under $A_4$. 
\begin{table}[t]
 \begin{tabular}{c|c|c|c}
%$SU(3)_C\times SU(2)_L\times U(1)_{Y_W}$
  Field & $\mathcal{G}_\text{SM}$ & $A_4 \times U(1)_\text{FN} \times Z_3$ & $U(1)_{B'}\times U(1)_{L'}\times U(1)_\chi$\\
  \hline
  $Q^i_L$ & (3,2,$\frac16$) & (1,$a^i$,$\omega$) & ($\frac13$,0,0)\\
  $U^i_R$ & (3,1,$\frac23$) & (1,$b^i$,$\omega^2$)& ($\frac13$,0,0)\\
  $D^i_R$ & (3,1,$-\frac13$) & (1,$c^i$,$\omega^2$)& ($\frac13$,0,0)\\
  $L^i_L$ & (1,2,$-\frac12$) & (3,0,$\omega$)& (0,1,0)\\
  $E^i_R$ & (1,1,$-1$) & ($1 {^(} {'} {^,} '' {^)} $,$d^i$,$\omega^2$)& (0,1,0)\\
  $H$ & (1,2,$\frac12$) & (1,0,1)& (0,0,0)\\
  \hline
  $\chi$ & (1,1,0) & (1,0,$\omega$)& (0,0,1)\\ %bar chi has omega**2 (?) ->Z3invariance
 & (1,3,0) & (1,0,$\omega$)&(0,0,1)\\
  $\Phi_l$ & (1,2,-$\frac12$) & ($1''$,0,1)& (0,-1,1)\\
  $\Phi_q$ & (3,2,-$\frac16$) & ($1$,0,1)& ($-\frac13$,0,1)\\
  \hline
  $\Phi_T$ & (1,1,0) & ($3$,0,1)& (0,0,0)\\
 \end{tabular}
\caption{Transformation rules for the SM and BSM fields. The two rows for $\chi$ denote "model A" and "model B", respectively. For the charges under $U(1)_\text{FN}$ and the representations for $E_R$ under $A_4$ 
see sections \ref{sec_FNGT} and \ref{sec_A4GT}.}
\label{tab_models}
\end{table}
The accidental symmetries in the last column in table \ref{tab_models}
enforce stability of the proton and prevents contributions to other baryon and/or lepton number violating
processes. Additionally, the $U(1)_\chi$ stabilises the LP which is a crucial premise for a DM candidate. 
When a fermionic multiplet only has $SU(2)_L\times U(1)_Y$ gauge interactions in the SM and its components have a common tree-level mass, the charged 
ones become havier than the neutral one due to quantum loop corrections of $\mathcal{O}$(100 MeV) \cite{Hisano}\cite{minMatter}. This implies that we 
can chose the hypercharge of $\chi$ to be 0 for both models. So eventually the interaction lagrangian for them reads
\begin{align}
 \mathcal{L} = g_i^q \bar \chi_R Q_L^i \Phi_q + g_i^l \bar \chi_R L_L^i \Phi_l + \text{h.c.}.
 \label{eq_modelLagrangian}
\end{align}
 
The accidental symmetries in the last column in table \ref{tab_models}
enforce stability of the proton and prevents contributions to other baryon and/or lepton number violating
processes. Additionally, the $U(1)_\chi$ stabilises the LP which is a crucial premise for a DM candidate. Its hypercharge for model B is chosen to be $+1$ 
so that the particle masses within the triplet have the right order which leaves the lightest particle electrically neutral. The case for $-1$ and applicable
hypercharges for the scalars would have similar implications. So the interaction lagrangian for both models reads
\begin{align}
 \mathcal{L} = g_i^q \bar \chi_R Q_L^i \Phi_q + g_i^l \bar \chi_R L_L^i \Phi_l + \text{h.c.}.
\end{align}




\section{Phenomenological Analysis}
Now we want to check the two models of our DM field $\chi$ being a singlet or a triplet under the $SU(2)_L$, respectively, on five processes. Three of them
are indirect ones where the NP states are only virtually involved. To fit their respective constraints on the contribution from NP, we get bounds on 
the dark matter mass and crosscheck the results with the two remaining ones, the annihilation and the nucleon scattering. In what follows the mass
of the $\chi$ is just $m$ and the parameter of interest, the masses of the scalars are $M_l$ or $M_q$, respectively. Furthermore we introduce the fractions
$x_l=\frac{M_l}{m}$ and $x_q = \frac{M_q}{m}$. 

\subsection{Anomalous Magnetic Moment of the Muon}
\begin{align}
 \Delta a_\mu^\text{NP} = \frac{{g_2^l}^2}{16\pi^2}\frac{m_\mu^2}{M_l^2}\left(Q^i_F \frac{1}{x_l}I(x_l^{-1}) + Q^i_B  I(x_l)\right)
\end{align}
\cite{Lavoura}

\subsection{$B_s\rightarrow \mu\mu$}
\begin{figure}[t]
 \includegraphics[width=0.7\textwidth]{../pics/boxBs-mumu.png}
 \caption{bs->mumu}
 \label{pic_Bsmumu}
\end{figure}
After we first focussed on a pure leptonic process, we add a quark pair. Box diagrams of the kind as in picture \ref{pic_Bsmumu} are usually calculated all
the way down to cross sections but rather one uses the formalism of effective field theory since the typical hadronic energy scale is of
$\mathcal{O}$(1 GeV) which is much lower than the weak scale $\mathcal{O}$(100 GeV) that is also the expected mass scale of our NP states. 
\\ \\ \noindent \textit{Matrix Element}\\
\noindent So we can use 
the OPE framework where they are integrated out and the external masses and momenta are set to zero. The amplitude in the full theory is
\begin{align}
 M =\alpha_i^{q*} \alpha_j^{q*} \alpha_m^l \alpha_n^l\int \frac{\dx^4 q}{(2\pi)^4} \frac{q^\rho}{q^2-m^2}\Delta_q \Delta_l \frac{q^\sigma}{q^2-m^2} \left(\bar L_L^n \gamma_\rho Q_L^j\right) \left(\bar Q_L^i \gamma_\sigma L_L^m\right)
 \label{eq_matElemBSmumu}
\end{align}
with $q$ as the internal momentum and the scalar propagators $\Delta = \frac{1}{q^2-M^2}$ and the $\ti\epsilon$ terms in the denominators are already omitted
due to Wick-rotation later on when the integral is performed. The reasons why $m$ in the fermion propagators is missing
and why there is a Lorentz structure in the couplings, are connected. Since we have a model which couples only to left handed particles $\psi_L = P_L \psi$,
the fermion currents look like $\bar \psi_L (\slashed{q}+m)\varphi_L = \bar \psi P_R (q^\mu \gamma_\mu+m) P_L\varphi$. Further
$\{P_{L,R},\gamma^\mu\} = \gamma^\mu P_{R,L}$, $P_{L,R}^2 = P_{L,R}$ and $P_R P_L = 0$, so we are left over with the four fermion expression in 
\eqref{eq_matElemBSmumu}. We can also reexpress $q^\rho q^\sigma = q^2 g^{\rho\sigma}/4$ under the integral. This results from spherical coordinates where
every combination $\rho\neq\sigma$ would include at least one uneven angular function. When the spherical integral is performed, they would lead to zero. The 
4 in the denominator is the spacetime dimension. The spherical integral $\dx \Omega$ itself just gives a $2\pi^2$ and we are left with the momentum integral
which is not problematic since it does not diverge
\begin{align}
 M \propto \frac{1}{32\pi^2} \int\limits_0^\infty \dx q \frac{q^5}{(q^2-m^2)^2(q^2-M_l^2)(q^2-M_q^2)} = \frac{K(x_q,x_l)}{64\pi^2 m^2}
\end{align}
with
\begin{align}
 K(x,y) &= \frac{K(x)-K(y)}{x-y}\\
 K(x)&=\frac{1-x+x^2\log(x)}{(x-1)^2}.
\end{align}
In the SM, the fermion currents do not combine quarks and leptons. To compare the NP contribution to this process, we have to rearrange the fermion currents,
so that we have the same structure. This will be done by the Fierz identities.
\\ \\ \noindent \textit{Fierz Identities}\\
\cite{Fierz}
\begin{align}
 e_S(1234) = 1\cdot e_V(3214)
\end{align}
gives only a factor 1.
\\ \\ \noindent \textit{Additional $SU(2)_L$-invariant term}\\
\noindent The operator $\left(\bar Q_L^i \gamma_\rho Q_L^j\right)\left(\bar L_L^m \gamma^\rho L_L^n\right)$ is not the only operator which is a singlet
in the isospin space but $\left(\bar Q_L^i\vec \tau \gamma_\rho Q_L^j\right)\left(\bar L_L^m\vec \tau \gamma^\rho L_L^n\right)$ as well which enables
charged currents. The eventual operator is a superposition of these two. There might be several ways of computing the relative prefactor of the 
second but one possibility is to regard the particles at each vertex as isospin multiplets. Since we want the vertices be $SU(2)_L$-invariant, we 
decompose the product and focus on the singlet. This will be done for all four vertices per diagram which are multiplied and added up for every diagram
one can write down with the operators. 
TODO: CG-coefficients
\begin{align}
 \nonumber
 \mathcal{L}_\text{eff} \supset \frac{K(x_q,x_l)}{m^2}\frac{\alpha_i^{q*} \alpha_j^{q*} \alpha_m^l \alpha_n^l}{64\pi^2}&\left[\left(\bar Q^i_L\gamma^\mu Q^j_L\right)\left(L^m_L\gamma_\mu L^n_L\right)\right.\\
 +&\left.\frac23\left(\bar Q^i_L\gamma^\mu \vec \tau Q^j_L\right)\left(L^m_L\gamma_\mu \vec \tau L^n_L\right)\right].
 \label{eq_LagBSmumuModB}
\end{align}
\begin{align}
  \mathcal{L}_\text{eff} \supset \frac{K(x_q,x_l)}{m^2}\frac{\alpha_i^{q*} \alpha_j^{q*} \alpha_m^l \alpha_n^l}{64\pi^2}&\left[\left(\bar Q^i_L\gamma^\mu Q^j_L\right)\left(L^m_L\gamma_\mu L^n_L\right)\right]
 \label{eq_LagBSmumuModA}
\end{align}




\subsection{$B_s$-Mixing}

\subsection{DM Annihilation}

\subsection{Direct Detection}

\section{Results}
\subsection{Model A}
\subsection{Model B}
\section{Conclusion and Prospects}

% dfasdölkj\\d\\d\\d\\d\\d\\d\\d\\d\\d\\d\\d

% \section{DM candidate coupling to light quarks in T' framework}
% \subsection{Model construction}
% % \subsubsection{Grouptheoretical properties}
% \begin{align}
%  \sum m_n n^2 = N_G \quad groupelements\\
%  \sum m_n = \#_{IRR} \quad C = Irred\,Reps\\
%  \varphi:\, G\rightarrow \text{GL}(V)\\
%  g\mapsto \varphi(g):\, V\rightarrow V\\
%  \chi_D(g) = \text{tr}D(g)\\
%  \sum_g \chi_\alpha(g)^*\chi_\beta(g) = N_G \delta_{\alpha\beta}\\
%  \sum_\alpha \chi_\alpha(g)^*\chi_\alpha(h) = \frac{N_G}{n_g} \delta_{C_g C_h} \stackrel{\Lambda}{=} \langle \chi^\mu, \chi^\nu \rangle = \delta^{\mu\nu}\\
%  \text{FS}(R) := \frac{1}{N_G} \sum_g \chi_R(g^2) =\begin{cases}
%                                                     1, & \text{real}\\
%                                                     0, & \text{complex}\\
%                                                     -1, & \text{pseudoreal}\\
%                                                    \end{cases}\\
%  \mu(k) = \langle \chi_R \cdot \chi_{R'} , \chi_{R_k} \rangle \quad tensor product decomposation\\
% \end{align}
% 
% \subsubsection{Assigning particles to multiplets}
% \begin{align}
%  \langle \xi'' \rangle = \delta u''\approx \epsilon^4 \Lambda\\
%  M_{\xi''} = k \delta u'' 
% \end{align}
% 
% \subsection{Messenger $\xi''$ mediating SM and DM}
% \subsubsection{DM annihilation into light mesons}
% \subsubsection*{DM stability}
% page 24 - check on T, Z3 reps and spin
% \subsubsection*{cross section}
% \begin{align}
%  \sigma(\chi\chi \rightarrow d \bar s) = \frac{\lambda_f^2\lambda_\chi^2}{8\pi(4m_\chi^2 - M_{\xi''}^2)^2}(4m_\chi^2 - (m_d+m_s)^2)\\
%  \langle \sigma_{\text{Ann}} v \rangle = \sigma(1+\frac18
% \end{align}
% 
% \subsubsection{Meson-Mixing}
% \begin{align}
%  \Xi_{dd'} = \begin{pmatrix}
%               0 & 1 & 0\\
%               1 & 0 & 0 \\
%               0 & 0 & 0
%              \end{pmatrix} \xrightarrow{massbasis} \begin{pmatrix}
% 						-\epsilon & 1 & -\epsilon^2\\
% 						1 & \epsilon & -\epsilon^3\\
% 						-\epsilon^2 & \-\epsilon^3 & \epsilon^5
% 						\end{pmatrix}
% \end{align}
% 
% \subsubsection{DM-nucleus scattering}
% 
% 
% \addcontentsline{toc}{section}{List of figures}
% \newpage\listoffigures\newpage
% \addcontentsline{toc}{section}{List of tables}
% \listoftables\newpage
\end{spacing}
\newpage

\begin{thebibliography}{xxx}
 \bibitem[1]{Grip}B. Gripaios et al. \textit{Linear flavour violation and anomalies in $B$ physics}\\ \href{http://arxiv.org/abs/1509.05020v1}{arxiv.org/abs/1509.05020v1}
 \bibitem[2]{Peskin}M. Peskin et al. \textit{An Introduction To Quantum Field Theory}\\ ISBN: 0-201-50397-2
 \bibitem[3]{Fierz}J.F. Nieves et al. \textit{Generalized Fierz identities}\\ \href{http://http://arxiv.org/abs/hep-ph/0306087v1}{arxiv.org/abs/hep-ph/0306087v1}
 \bibitem[4]{FerA4}G. Altarelli et al. \textit{Tri-Bimaximal Neutrino Mixing and Discrete Flavour Symmetries}\\ \href{https://arxiv.org/abs/1205.5133}{arxiv.org/abs/1205.5133}
 \bibitem[5]{VarzTotMod}I.d.M. Varzielas et al. \textit{Clues for flavour from rare lepton and quark decays}\\ \href{https://arxiv.org/abs/1503.01084}{arxiv.org/abs/1503.01084}
 \bibitem[6]{Lavoura}L. Lavoura \textit{General formulae for $f_1\rightarrow f_2 \gamma$}\\ \href{http://arxiv.org/abs/hep-ph/0302221}{arxiv.org/abs/hep-ph/0302221}
 \bibitem[7]{BurasEFT}A.J. Buras \textit{Weak Hamiltonian, CP Violation and Rare Decays}\\ \href{http://arxiv.org/abs/hep-ph/9806471}{arxiv.org/abs/hep-ph/9806471}
\end{thebibliography}
\end{document}
