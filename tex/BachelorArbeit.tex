\documentclass[11pt,a4paper,twoside]{report}
\input{longheader.tex}
\usepackage{xfrac}
\usepackage{xcolor}
\usepackage{setspace}\usepackage{threeparttable}


 %\setlength{\parskip}{1.5ex}
\begin{document}
\begin{spacing}{1,2}
\pagenumbering{Roman}

% Anmerkung: Die Seitenraender wurden asymmetrisch gewaehlt,
%            damit genug Platz fuer eine Klemmbindung da ist.
%            Da neue Kapitel auf der rechten Seite (ungerade
%            Seitennummer) beginnen sollten, muss ggf. am Ende
%            des vorhergehenden Kapitels eine Leerseite
%            eingefuegt werden:
%
%            \newpage
%            \thispagestyle{empty}
%            \ \\
%            \newpage
%
%            Die Seitenraender koennen aber auch in der Datei Tex/global.tex
%            veraendert werden.

% >>> Titelseite <<<

\newcommand{\thetitle}{Formfaktoren des semileptonischen $D \rightarrow  K l^+ \nu$ Zerfalls}

\thispagestyle{empty}
\begin{center}

\Huge\textbf{\thetitle}
\vfill
% Note that the size is given in normal parentheses
% instead of curly brackets.
% Define external vertices from bottom to top
\vfill
\Large
Bachelorarbeit \\ zur Erlangung des akademischen Grades \\ Bachelor of Science \\
\vspace{20pt}
\normalsize
vorgelegt von \\[5pt]
{\Large Dimitrios Skodras} \\[5pt]
geboren in Aschaffenburg \\
\vspace{20pt}
Lehrstuhl für Theoretische Physik IV \\ Fakultät Physik \\
Technische Universität Dortmund \\ 2014
\end{center}
\newpage
% % >>> Gutachterseite <<<
% \thispagestyle{empty}
% \newpage
% \cleardoublepage

\thispagestyle{empty}
\vspace*{\fill}
\begin{tabbing}
1. Gutachter : \=\kill
1. Gutachter : \>Prof. Dr. Gudrun Hiller \\[11pt]
2. Gutachter : \>Dr. Martin Jung\\[11pt]
\end{tabbing}
\vspace{11pt}
Datum des Einreichens der Arbeit: 14. Juli, 2014
\newpage
\thispagestyle{empty}
\begin{flushright} 
\textit{\glqq Es gibt nichts Praktischeres, als eine gute Theorie.\grqq}\\
- \textit{Kant, Immanuel}\\
\vspace{2cm}
% \textit{``Was dürfen wir hoffen?''}\\
% - \textit{Kant, Immanuel}\\
% \vspace{2cm}
% \textit{``Kein Mensch ist so wichtig, wie er sich nimmt.''}\\
% - \textit{Kant, Immanuel}
\end{flushright}

\newpage


% >>> Kurzfassung/Abstract <<<

\thispagestyle{empty}
%Kuzfassung
% \section*{Kurzfassung}
% Im Zuge dieser Arbeit sollen dimensionslose Formfaktoren ermittelt werden, die eine Darstellung eines zerfallrelevanten Matrixelements bilden. Sie werden
% durch eine $z$-Reihenentwicklung parametrisiert und anhand von Daten der CLEO Collaboration unter Minimierung einer $\chi^2$-Funktion bis zur 
% zweiten Ordnung gefittet. Die Fitparameter für $D^0\rightarrow  K^- l \nu$ sind $a_0$ = 0,726, $a_1$ = 0,140, $a_2$ = 9,548 und der Wert für den verbleibenden Formfaktor ist $|V_{cs}|f_+(q^2=0)$ = 0,762, wobei 
% $q^2$ der Impulsübertrag des Leptonpaares ist. Und für $D^+\rightarrow \bar K^0 l \nu$ sind $a_0$ = 0,708, $a_1$ = -1,263, $a_2$ = -10,95, sowie $|V_{cs}|f(0)$ = 0,708.
% 
% % Anhand des Flächeninhalts des Unitaritätsdreiecks, das aus Elementen der CKM-Matrix gebildet wird, kann man die Stärke der CP-Verletzung bestimmen. Ihre
% % Einträge beschreiben die Wahrscheinlichkeit eines Übergangs zwischen schwach zerfallenden Quarks. Da diese Quarks nicht ungebunden existieren, muss der
% % Einfluss der ``Spectatorquarks'' mitberücksichtigt werden, was durch den Ausdruck einheitenloser Formfaktoren geschieht in Abhängigkeit des Impulsübertrags.
% % Durch Minimierung eines $\chi^2$-Tests anhand von Daten der CLEO Collaboration werden geeignete Koeffizienten einer Reihenentwicklung mittels
% % eines Python-Scripts bestimmt.
% % Hieraus ergibt sich für den nicht verschwindenden Formfaktor $f_+$, multipliziert mit dem entsprechenden CKM-Element $V_{cs}$ bei nicht vorhandenem 
% % Impulsübertrag $q^2$ ein Wert von $f_+(0)|V_{cs}| = 0,707(10)$. Dies gilt für den angegebenen Zerfall. Übergänge mit anderen Spectatorquarks ergeben ähnliche
% % Ergebnisse mit einem Unterschied von bis zu 5\%.
% 
% \section*{Abstract}
% As part of this thesis form factors get ascertained, which build an exposition of a matrix element pertinent to the decay. They are parameterised by a
% series expansion of the $z$-expansion and fitted to the second order by data of the CLEO Collaboration under minimizing a $\chi^2$-function. The resulting
% fit parameters for $D^0\rightarrow  K^- l \nu$ are $a_0$ = 0,726, $a_1$ = 0,140, $a_2$ = 9,548 and the value for the form factor is $|V_{cs}|f_+(q^2=0)$ = 0,762, where $q^2$ is the momentum transfer carried by
% the lepton pair. And for $D^+\rightarrow \bar K^0 l \nu$ are $a_0$ = 0,708, $a_1$ = -1,263, $a_2$ = -10,95 and $|V_{cs}|f(0)$ = 0,708.

% It is possible to determine the magnitude of the CP-violation by calculating the area of the unitarity triangle, build by elements of the CKM-matrix. Her
% entries characterise the propability of a transition between weak decaying quarks. They are never found in unbound states, so the simple view on the converting
% quark is not sufficient, wherefore you need to consider the additional spectatorquark, that happens through form factors which depend on
% the momentum transfer. With a Python-Script, it is possible, to ascertain the coefficients of a series expansion for the form factor by minimizing a 
% $\chi^2$-test in reference to
% data from the CLEO Collaboration. It reveals for the remaining form factor $f_+$, multiplied by the corresponding CKM-element $V_{cs}$ at a vanishing
% momentum transfer $q^2$ in a value of $f_+(0)|V_{cs}| = 0,707(10)$. This counts for the given decay mode, but other transitions with other spectatorquarks
% state the same results within an accuracy of 5\%.
\newpage

% >>> Hauptteil <<<

%\addcontentsline{toc}{chapter}{Inhaltsverzeichnis}
\end{spacing}
\begin{spacing}{1,1}
\tableofcontents\newpage

\end{spacing}
\begin{spacing}{1,2}
\addcontentsline{toc}{chapter}{Abbildungsverzeichnis}
\listoffigures\newpage
\addcontentsline{toc}{chapter}{Tabellenverzeichnis}
\listoftables\newpage
\thispagestyle{empty}
% \cleardoublepage

\setcounter{page}{0}
\pagenumbering{arabic}

\end{spacing}

\end{document}