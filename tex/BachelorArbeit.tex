\documentclass[11pt,a4paper,twoside]{report}

% Deutsche Spracheinstellungen
\usepackage[english,english]{babel, varioref}
\usepackage[T1]{fontenc}
\usepackage[utf8]{inputenc}

%\usepackage{marvosym}

\usepackage{amsfonts}
\usepackage{amssymb}
\usepackage{amsmath}
\usepackage{amscd}
\usepackage{amstext}
\usepackage{float}
\usepackage{caption}
\usepackage{wrapfig}
\usepackage{setspace}
%\usepackage[onehalfspacing]{setspace}
\usepackage{threeparttable}
\usepackage{footnote}
\usepackage{feynmf}
\usepackage{bbm}
\usepackage{slashed}
\usepackage{textcomp}
\usepackage{multirow}
\usepackage{courier}
\usepackage{listings}
\usepackage{color}
%\usepackage{minipage}
 
 \definecolor{middlegray}{rgb}{0.5,0.5,0.5}
 \definecolor{lightgray}{rgb}{0.8,0.8,0.8}
 \definecolor{orange}{rgb}{0.8,0.3,0.3}
 \definecolor{yac}{rgb}{0.6,0.6,0.1}
 \definecolor{puple}{rgb}{0.62,0.12,0.94}
 \lstset{language=Python,
                basicstyle=\ttfamily,
                keywordstyle=\color{red}\ttfamily,
                stringstyle=\color{magenta}\ttfamily,
                commentstyle=\color{blue}\ttfamily,
                morecomment=[l][\color{blue}]{\#},
		%stepnumber=1,
		%numberstyle=\color{magenta}\ttfamily,
		%    numbers=left,
		%    numberstyle={},
		%    numberblanklines=false,
		%    stepnumber=1,
		%    numbersep=10pt,
		    xleftmargin=15pt,
 		moredelim=[is][\color{purple}]{|}{|}
}

\newfloat{formel}{htbp}{for}
\floatname{formel}{Formel}

\onehalfspacing
%\setstretch {1.433}

\usepackage{longtable}

%\usepackage{bibgerm}

\usepackage{footnpag}

\usepackage{ifthen}                 %%% package for conditionals in TeX
\usepackage[amssymb]{SIunits}
%Fr textumflossene Bilder und Tablellen
%\usepackage{floatflt} - veraltet

%Fr Testzwecke aktivieren, zeigt labels und refs im Text an.
%\usepackage{showkeys}

% Abstand zwischen zwei Abs�zen nach DIN (1,5 Zeilen)
% \setlength{\parskip}{1.5ex}

% Einrckung am Anfang eines neuen Absatzes nach DIN (keine)
%\setlength{\parindent}{0pt}

% R�der definieren
% \setlength{\oddsidemargin}{0.3cm}
% \setlength{\textwidth}{15.6cm}

% bessere Bildunterschriften
\usepackage{caption2}


% Probleml�ungen beim Umgang mit Gleitumgebungen
\usepackage{float}

% Nummeriert bis zur Strukturstufe 3 (also <section>, <subsection> und <subsubsection>)
%\setcounter{secnumdepth}{3}

% Fhrt das Inhaltsverzeichnis bis zur Strukturstufe 3
%\setcounter{tocdepth}{3}

\usepackage{exscale}

\newenvironment{dsm} {\begin{displaymath}} {\end{displaymath}}
\newenvironment{vars} {\begin{center}\scriptsize} {\normalsize \end{center}}


\newcommand {\en} {\varepsilon_0}               % Epsilon-Null aus der Elektrodynamik
\newcommand {\lap} {\; \mathbf{\Delta}}         % Laplace-Operator
\newcommand {\R} { \mathbb{R} }                 % Menge der reellen Zahlen
\newcommand {\e} { \ \mathbf{e} }               % Eulersche Zahl
\renewcommand {\i} { \mathbf{i} }               % komplexe Zahl i
\newcommand {\N} { \mathbb{N} }                 % Menge der nat. Zahlen
\newcommand {\C} { \mathbb{C} }                 % Menge der kompl. Zahlen
\newcommand {\Z} { \mathbb{Z} }                 % Menge der kompl. Zahlen
\newcommand {\limi}[1]{\lim_{#1 \rightarrow \infty}} % Limes unendlich
\newcommand {\sumi}[1]{\sum_{#1=0}^\infty}
\newcommand {\rot} {\; \mathrm{rot} \,}         % Rotation
\newcommand {\grad} {\; \mathrm{grad} \,}       % Gradient
\newcommand {\dive} {\; \mathrm{div} \,}        % Divergenz
\newcommand {\dx} {\; \mathrm{d} }              % Differential d
\newcommand {\cotanh} {\; \mathrm{cotanh} \,}   %Cotangenshyperbolicus
\newcommand {\asinh} {\; \mathrm{areasinh} \,}  %Area-Sinus-Hyp.
\newcommand {\acosh} {\; \mathrm{areacosh} \,}  %Area-Cosinus-H.
\newcommand {\atanh} {\; \mathrm{areatanh} \,}  %Area Tangens-H.
\newcommand {\acoth} {\; \mathrm{areacoth} \,}  % Area-cotangens
\newcommand {\Sp} {\; \mathrm{Sp} \,}
\newcommand {\mbe} {\stackrel{\text{!}}{=}}     %Must Be Equal
\newcommand{\qed} { \hfill $\square$\\}
\newcommand{\midtilde}{\raisebox{-0,25\baselineskip}{\textasciitilde}}
\renewcommand{\i} {\imath}
\def\captionsngerman{\def\figurename{\textbf{Abb.}}}

%%%%%%%%%%%%%%%%%%%%%%%%%%%%%%%%%%%%%%%%%%%%%%%%%%%%%%%%%%%%%%%%%%%%%%%%%%%%
% SWITCH FOR PDFLATEX or LATEX
%%%%%%%%%%%%%%%%%%%%%%%%%%%%%%%%%%%%%%%%%%%%%%%%%%%%%%%%%%%%%%%%%%%%%%%%%%%%
%%%
\ifx\pdfoutput\undefined %%%%%%%%%%%%%%%%%%%%%%%%%%%%%%%%%%%%%%%%% LATEX %%%
%%%
\usepackage[dvips]{graphicx}       %%% graphics for dvips
\DeclareGraphicsExtensions{.eps,.ps}   %%% standard extension for included graphics
\usepackage[ps2pdf]{thumbpdf}      %%% thumbnails for ps2pdf
\usepackage[ps2pdf,                %%% hyper-references for ps2pdf
bookmarks=true,%                   %%% generate bookmarks ...
bookmarksnumbered=true,%           %%% ... with numbers
hypertexnames=false,%              %%% needed for correct links to figures !!!
breaklinks=true,%                  %%% breaks lines, but links are very small
linkbordercolor={0 0 1},%          %%% blue frames around links
pdfborder={0 0 112.0}]{hyperref}%  %%% border-width of frames
%                                      will be multiplied with 0.009 by ps2pdf
%
\hypersetup{ pdfauthor   = {Dimitrios Skodras},
pdftitle    = {Fermionic Dark Matter and its Role on B Anomalies}, pdfsubject  = {masterthesis}, pdfkeywords = {dark matter},
pdfcreator  = {LaTeX with hyperref package}, pdfproducer = {dvips
+ ps2pdf} }
%%%
\else %%%%%%%%%%%%%%%%%%%%%%%%%%%%%%%%%%%%%%%%%%%%%%%%%%%%%%%%%% PDFLATEX %%%
%%%
\usepackage[pdftex]{graphicx}      %%% graphics for pdfLaTeX
\DeclareGraphicsExtensions{.pdf}   %%% standard extension for included graphics
\usepackage[pdftex]{thumbpdf}      %%% thumbnails for pdflatex
\usepackage[pdftex,                %%% hyper-references for pdflatex
bookmarks=true,%                   %%% generate bookmarks ...
bookmarksnumbered=true,%           %%% ... with numbers
hypertexnames=false,%              %%% needed for correct links to figures !!!
breaklinks=true,%                  %%% break links if exceeding a single line
linkbordercolor={0 0 1},
linktocpage]{hyperref} %%% blue frames around links
%                                  %%% pdfborder={0 0 1} is the default
\hypersetup{
pdftitle    = {Fermionic Dark Matter and its Role on B Anomalies}, %right place
pdfsubject  = {master thesis}, 
pdfkeywords = {V301, Innenwiderstand, Leistungsanpassung},
pdfsubject  = {Protokoll AP},
pdfkeywords = {V301, Innenwiderstand, Leistungsanpassung}}
%                                  %%% pdfcreator, pdfproducer,
%                                      and CreationDate are automatically set
%                                      by pdflatex !!!
\pdfadjustspacing=1                %%% force LaTeX-like character spacing
\usepackage{epstopdf}
%
\fi %%%%%%%%%%%%%%%%%%%%%%%%%%%%%%%%%%%%%%%%%%%%%%%%%%% END OF CONDITION %%%
%%%%%%%%%%%%%%%%%%%%%%%%%%%%%%%%%%%%%%%%%%%%%%%%%%%%%%%%%%%%%%%%%%%%%%%%%%%%
% seitliche Tabellen und Abbildungen
%\usepackage{rotating}
\usepackage{ae}
\usepackage{
  array,
  booktabs,
  dcolumn
}
\makeatletter 
  \renewenvironment{figure}[1][] {% 
    \ifthenelse{\equal{#1}{}}{% 
      \@float{figure} 
    }{% 
      \@float{figure}[#1]% 
    }% 
    \centering 
  }{% 
    \end@float 
  } 
  \makeatother 


  \makeatletter 
  \renewenvironment{table}[1][] {% 
    \ifthenelse{\equal{#1}{}}{% 
      \@float{table} 
    }{% 
      \@float{table}[#1]% 
    }% 
    \centering 
  }{% 
    \end@float 
  } 
  \makeatother 
%\usepackage{listings}
%\lstloadlanguages{[Visual]Basic}
%\allowdisplaybreaks[1]
%\usepackage{hycap}
%\usepackage{fancyunits}

\usepackage{xfrac}
\usepackage{xcolor}
\usepackage{setspace}\usepackage{threeparttable}


 %\setlength{\parskip}{1.5ex}
\begin{document}
\begin{spacing}{1,2}
\pagenumbering{Roman}

% Anmerkung: Die Seitenraender wurden asymmetrisch gewaehlt,
%            damit genug Platz fuer eine Klemmbindung da ist.
%            Da neue Kapitel auf der rechten Seite (ungerade
%            Seitennummer) beginnen sollten, muss ggf. am Ende
%            des vorhergehenden Kapitels eine Leerseite
%            eingefuegt werden:
%
%            \newpage
%            \thispagestyle{empty}
%            \ \\
%            \newpage
%
%            Die Seitenraender koennen aber auch in der Datei Tex/global.tex
%            veraendert werden.

% >>> Titelseite <<<

\newcommand{\thetitle}{Formfaktoren des semileptonischen $D \rightarrow  K l^+ \nu$ Zerfalls}

\thispagestyle{empty}
\begin{center}

\Huge\textbf{\thetitle}
\vfill
% Note that the size is given in normal parentheses
% instead of curly brackets.
% Define external vertices from bottom to top
\vfill
\Large
Bachelorarbeit \\ zur Erlangung des akademischen Grades \\ Bachelor of Science \\
\vspace{20pt}
\normalsize
vorgelegt von \\[5pt]
{\Large Dimitrios Skodras} \\[5pt]
geboren in Aschaffenburg \\
\vspace{20pt}
Lehrstuhl für Theoretische Physik IV \\ Fakultät Physik \\
Technische Universität Dortmund \\ 2014
\end{center}
\newpage
% % >>> Gutachterseite <<<
% \thispagestyle{empty}
% \newpage
% \cleardoublepage

\thispagestyle{empty}
\vspace*{\fill}
\begin{tabbing}
1. Gutachter : \=\kill
1. Gutachter : \>Prof. Dr. Gudrun Hiller \\[11pt]
2. Gutachter : \>Dr. Martin Jung\\[11pt]
\end{tabbing}
\vspace{11pt}
Datum des Einreichens der Arbeit: 14. Juli, 2014
\newpage
\thispagestyle{empty}
\begin{flushright} 
\textit{\glqq Es gibt nichts Praktischeres, als eine gute Theorie.\grqq}\\
- \textit{Kant, Immanuel}\\
\vspace{2cm}
% \textit{``Was dürfen wir hoffen?''}\\
% - \textit{Kant, Immanuel}\\
% \vspace{2cm}
% \textit{``Kein Mensch ist so wichtig, wie er sich nimmt.''}\\
% - \textit{Kant, Immanuel}
\end{flushright}

\newpage


% >>> Kurzfassung/Abstract <<<

\thispagestyle{empty}
%Kuzfassung
% \section*{Kurzfassung}
% Im Zuge dieser Arbeit sollen dimensionslose Formfaktoren ermittelt werden, die eine Darstellung eines zerfallrelevanten Matrixelements bilden. Sie werden
% durch eine $z$-Reihenentwicklung parametrisiert und anhand von Daten der CLEO Collaboration unter Minimierung einer $\chi^2$-Funktion bis zur 
% zweiten Ordnung gefittet. Die Fitparameter für $D^0\rightarrow  K^- l \nu$ sind $a_0$ = 0,726, $a_1$ = 0,140, $a_2$ = 9,548 und der Wert für den verbleibenden Formfaktor ist $|V_{cs}|f_+(q^2=0)$ = 0,762, wobei 
% $q^2$ der Impulsübertrag des Leptonpaares ist. Und für $D^+\rightarrow \bar K^0 l \nu$ sind $a_0$ = 0,708, $a_1$ = -1,263, $a_2$ = -10,95, sowie $|V_{cs}|f(0)$ = 0,708.
% 
% % Anhand des Flächeninhalts des Unitaritätsdreiecks, das aus Elementen der CKM-Matrix gebildet wird, kann man die Stärke der CP-Verletzung bestimmen. Ihre
% % Einträge beschreiben die Wahrscheinlichkeit eines Übergangs zwischen schwach zerfallenden Quarks. Da diese Quarks nicht ungebunden existieren, muss der
% % Einfluss der ``Spectatorquarks'' mitberücksichtigt werden, was durch den Ausdruck einheitenloser Formfaktoren geschieht in Abhängigkeit des Impulsübertrags.
% % Durch Minimierung eines $\chi^2$-Tests anhand von Daten der CLEO Collaboration werden geeignete Koeffizienten einer Reihenentwicklung mittels
% % eines Python-Scripts bestimmt.
% % Hieraus ergibt sich für den nicht verschwindenden Formfaktor $f_+$, multipliziert mit dem entsprechenden CKM-Element $V_{cs}$ bei nicht vorhandenem 
% % Impulsübertrag $q^2$ ein Wert von $f_+(0)|V_{cs}| = 0,707(10)$. Dies gilt für den angegebenen Zerfall. Übergänge mit anderen Spectatorquarks ergeben ähnliche
% % Ergebnisse mit einem Unterschied von bis zu 5\%.
% 
% \section*{Abstract}
% As part of this thesis form factors get ascertained, which build an exposition of a matrix element pertinent to the decay. They are parameterised by a
% series expansion of the $z$-expansion and fitted to the second order by data of the CLEO Collaboration under minimizing a $\chi^2$-function. The resulting
% fit parameters for $D^0\rightarrow  K^- l \nu$ are $a_0$ = 0,726, $a_1$ = 0,140, $a_2$ = 9,548 and the value for the form factor is $|V_{cs}|f_+(q^2=0)$ = 0,762, where $q^2$ is the momentum transfer carried by
% the lepton pair. And for $D^+\rightarrow \bar K^0 l \nu$ are $a_0$ = 0,708, $a_1$ = -1,263, $a_2$ = -10,95 and $|V_{cs}|f(0)$ = 0,708.

% It is possible to determine the magnitude of the CP-violation by calculating the area of the unitarity triangle, build by elements of the CKM-matrix. Her
% entries characterise the propability of a transition between weak decaying quarks. They are never found in unbound states, so the simple view on the converting
% quark is not sufficient, wherefore you need to consider the additional spectatorquark, that happens through form factors which depend on
% the momentum transfer. With a Python-Script, it is possible, to ascertain the coefficients of a series expansion for the form factor by minimizing a 
% $\chi^2$-test in reference to
% data from the CLEO Collaboration. It reveals for the remaining form factor $f_+$, multiplied by the corresponding CKM-element $V_{cs}$ at a vanishing
% momentum transfer $q^2$ in a value of $f_+(0)|V_{cs}| = 0,707(10)$. This counts for the given decay mode, but other transitions with other spectatorquarks
% state the same results within an accuracy of 5\%.
\newpage

% >>> Hauptteil <<<

%\addcontentsline{toc}{chapter}{Inhaltsverzeichnis}
\end{spacing}
\begin{spacing}{1,1}
\tableofcontents\newpage

\end{spacing}
\begin{spacing}{1,2}
\addcontentsline{toc}{chapter}{Abbildungsverzeichnis}
\listoffigures\newpage
\addcontentsline{toc}{chapter}{Tabellenverzeichnis}
\listoftables\newpage
\thispagestyle{empty}
% \cleardoublepage

\setcounter{page}{0}
\pagenumbering{arabic}

\end{spacing}

\end{document}