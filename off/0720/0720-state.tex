\documentclass[11pt,a4paper,twoside]{article}
\input{longheader.tex}
\usepackage{xfrac}
\usepackage{xcolor}
\usepackage{setspace}\usepackage{threeparttable}
\usepackage{fancyhdr}
\fancyfoot{}
\fancyhead[RO,LE]{\thepage}
\fancyhead[LO]{\leftmark}
\fancyhead[RE]{\rightmark}


 %\setlength{\parskip}{1.5ex}
\begin{document}
\begin{spacing}{1,2}
\pagenumbering{Roman}

% Anmerkung: Die Seitenraender wurden asymmetrisch gewaehlt,
%            damit genug Platz fuer eine Klemmbindung da ist.
%            Da neue Kapitel auf der rechten Seite (ungerade
%            Seitennummer) beginnen sollten, muss ggf. am Ende
%            des vorhergehenden Kapitels eine Leerseite
%            eingefuegt werden:
%
%            \newpage
%            \thispagestyle{empty}
%            \ \\
%            \newpage
%
%            Die Seitenraender koennen aber auch in der Datei Tex/global.tex
%            veraendert werden.

% >>> Titelseite <<<

\newcommand{\thetitle}{DM sheds light (up)on flavor anomalies}

\thispagestyle{empty}
\begin{center}

\Huge\textbf{\thetitle}
\vfill
% Note that the size is given in normal parentheses
% instead of curly brackets.
% Define external vertices from bottom to top
\vfill

\end{center}
\newpage
% % >>> Gutachterseite <<<
% \thispagestyle{empty}
% \newpage
% \cleardoublepage




% >>> Kurzfassung/Abstract <<<


% >>> Hauptteil <<<

%\addcontentsline{toc}{chapter}{Inhaltsverzeichnis}


\pagenumbering{arabic}
\setcounter{page}{1}

\section{Model outline}

\begin{table}
 \begin{tabular}{c|c|c|c}
  Field & $SU(3)_C\times SU(2)_L\times U(1)_{Y_W}$ & $A_4 \times U(1)_\text{FN} \times Z_3$ & $U(1)_\chi$\\
  \hline
  $Q^i_L$ & (3,2,$\frac16$) & (1,$a^i$,$\omega$) & 0\\
  $U^i_R$ & (3,1,$\frac23$) & (1,$b^i$,$\omega^2$)& 0\\
  $D^i_R$ & (3,1,$-\frac13$) & (1,$c^i$,$\omega^2$)& 0\\
  $L^i_L$ & (1,2,$-\frac12$) & (3,0,$\omega$)& 0\\
  $E^i_R$ & (1,1,$-1$) & ($1 {^(} {'} {^,} '' {^)} $,$d^i$,$\omega^2$)& 0\\
  $\Phi_H$ & (1,2,$\frac12$) & (1,0,1)& 0\\
  \hline
  $\psi$ & (1,1,0) & (1,0,$\omega^2$)& 1\\
  $\Phi_l$ & (1,2,-$\frac12$) & ($1''$,0,1)& 1\\
  $\Phi_q$ & (3,2,$\frac16$) & ($1$,0,1)& 1\\
  \hline
  $\Phi_T$ & (1,1,0) & ($3$,0,1)& 0\\

 \end{tabular}
\caption{Transformation rules for our SM and BSM fields. The index $i$ stands for the three generations which influence the charges under the 
$U(1)_\text{FN}$ $a-d$. The representation for $E_R$ also depends on the generation ($e^c \sim 1$, $\mu \sim 1''$, $\tau \sim 1'$).}
\label{tab_model}
\end{table}
\begin{align}
 \mathcal{L}_\text{yuk} = \alpha^q_i \bar \psi_R Q^i_L \Phi_q + \alpha^l_i\bar \psi_R L^i_L \Phi_l + \text{h.c.}
\end{align}




\section{Possible processes to look at}
Now we can check on some processes to give estimations for the masses of our BSM particles.
\subsection{anomalous magnetic momentum of the muon $\Delta a_\mu$}

\subsection{$B \rightarrow K\mu\mu$}
The goal for this section was to reproduce the effective Lagrangian for processes depicted in figure \ref{pic_boxqqll} in eq. 3.1 from \cite{Grip} which reads
\begin{align}
 \mathcal{L}_\text{eff} \supset \frac{K(x_q,x_l)}{m^2}\frac{\alpha_i^{q*} \alpha_j^{q*} \alpha_m^l \alpha_n^l}{64\pi^2}\left[\left(\bar Q^i_L\gamma^\mu Q^j_L\right)\left(L^m_L\gamma_\mu L^n_L\right)+\frac59\left(\bar Q^i_L\gamma^\mu \vec \tau Q^j_L\right)\left(L^m_L\gamma_\mu \vec \tau L^n_L\right)\right].
 \label{eq_effLag}
\end{align}
with $x_i = M_i^2/m^2$. Indeed I reproduced the loop function which gave me an additional 1/2 factor compared to the one stated in the paper. The normalisation of 1/$(2\pi)^4$, $2\pi^2$
from the $\Omega$-Integral a $1/4$ from reexpressing the internal momentum and the mentioned 1/2 from the $q$-integral add up to 1/$64\pi^2$. From the 
Fierz-transformation \cite{Fierz}
\begin{align}
 e_S(1234) = FSF e_V(4231)
\end{align}
we get just a factor 1, if we say we start with scalar couplings but want (axial-)vector ones. I want to sketch the main steps for \eqref{eq_effLag}. 
To me the effective Lagrangian expresses the 4-point-fermion-operatorwith an effective coupling which can be obtained by the full theory. 
So we start with the matrix element in the full theory
\begin{align}
 M &= \alpha_i^{q*} \alpha_j^{q*} \alpha_m^l \alpha_n^l \int \frac{\dx^4 q}{(2\pi)^4} \bar L_L^n \frac{\slashed{q}}{q^2-m^2+\text{i}\epsilon}Q_L^j \Delta_q \Delta_l \bar L_L^m \frac{\slashed{q}-\slashed{p_1}-\slashed{p_2}}{q^2-m^2+\text{i}\epsilon}Q_L^i\\
 &=\frac{\alpha^4}{(2\pi)^4} \int \dx^4 q \frac{q^\rho}{q^2-m^2}\Delta_q \Delta_l \frac{q^\sigma}{q^2-m^2} \left(\bar L_L^n \gamma_\rho Q_L^j\right)\left(\bar L_L^m \gamma_\sigma Q_L^i\right)
\end{align}
The $\Delta$ are the propagators of the scalars $\Phi_q$ and $\Phi_l$. The external momenta are neglectable. \textbf{Do the i$\epsilon$ terms disappear due to
Wick-rotation?} We can reexpress $q^\rho q^\sigma = q^2 g^{\rho\sigma}/4$ under the integral. After going into spherical coordinates every unequal combination
of $\rho$ and $\sigma$ would include an uneven function with respect to at least one angle and therefore lead to zero after the spherical integration. 
\textbf{For the moment I'm not sure, where the 4 is coming from.} Maybe when it is contracted with the $\gamma$ for example it gives another 4 from dimension.
From the $\Omega$-integration we get a $2\pi^2$. Another thing concerning about the Fierzing. I said that I transform from scalar to vector coupling whilst
swapping the first and fourth isospin-doublet. \textbf{But shouldn't it be a transformation from vector to vector when we look at the operator in the expression?}
Well they are still scalar couplings... 
\begin{align}
 M &= \frac{\alpha^4}{32\pi^2} \int\limits_0^\infty \dx q \frac{q^5}{(q^2-m^2)^2(q^2-M_l^2)(q^2-M_q^2)} \left(\bar Q_L^i \gamma_\rho Q_L^j\right)\left(\bar L_L^m \gamma_\rho L_L^n\right).
\end{align}
The integral, computed with \texttt{mathematica}, yields
\begin{align}
 \int \limits_0^\infty \dx q \frac{q^5}{D} = \frac{K(x_q,x_l)}{2m^2}
\end{align}
giving the correct prefactor. So, this was actually not the major thing I did the last three weeks but this will certainly be in my master thesis. 
One can already see that eq. 3.1 from \cite{Grip} is not totally reproduced. The second term with the Pauli matrices $\vec \tau$ is not there.
Now I have two problems concerning this term. First, why does it exist anyway and second, how do I get the prefactor of 5/9. In respect of the former, I think
we said that it is there because these two terms are the possible $SU(2)_L$ invariant ones we can build. But J. Brod told me that this term should drop out 
of the box calculus and should not be put in just by generality. 

\begin{figure}[t]
 \includegraphics[width=\textwidth]{boxBs-mumu.png}
 \caption{Box diagram for $qq\rightarrow ll$ processes. The numbers in the parentheses are the electric charges of the components of the respective $SU(2)_L$-multiplets.}
 \label{pic_boxqqll}
\end{figure}


\subsection{$\bar B_s B_s$ - mixing}
\subsection{DM-annihilation $\bar \psi\psi \rightarrow \bar f f$ }



\newpage
\end{spacing}
\newpage
\begin{thebibliography}{xxx}
 \bibitem[1]{Grip}B. Gripaios et al. \textit{Linear flavour violation and anomalies in $B$ physics}\\ \href{http://arxiv.org/abs/1509.05020v1}{http://arxiv.org/abs/1509.05020v1}
 \bibitem[2]{Peskin}M. Peskin et al. \textit{An Introduction To Quantum Field Theory}\\ ISBN: 0-201-50397-2
 \bibitem[3]{Fierz}J.F. Nieves et al. \textit{Generalized Fierz identities}
\end{thebibliography}
\end{document}
