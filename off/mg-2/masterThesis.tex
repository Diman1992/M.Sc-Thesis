\documentclass[11pt,a4paper,twoside]{article}
\input{longheader.tex}
\usepackage{xfrac}
\usepackage{xcolor}
\usepackage{setspace}\usepackage{threeparttable}
\usepackage{fancyhdr}
\fancyfoot{}
\fancyhead[RO,LE]{\thepage}
\fancyhead[LO]{\leftmark}
\fancyhead[RE]{\rightmark}


 %\setlength{\parskip}{1.5ex}
\begin{document}
\begin{spacing}{1,2}
\pagenumbering{Roman}

% Anmerkung: Die Seitenraender wurden asymmetrisch gewaehlt,
%            damit genug Platz fuer eine Klemmbindung da ist.
%            Da neue Kapitel auf der rechten Seite (ungerade
%            Seitennummer) beginnen sollten, muss ggf. am Ende
%            des vorhergehenden Kapitels eine Leerseite
%            eingefuegt werden:
%
%            \newpage
%            \thispagestyle{empty}
%            \ \\
%            \newpage
%
%            Die Seitenraender koennen aber auch in der Datei Tex/global.tex
%            veraendert werden.

% >>> Titelseite <<<

\newcommand{\thetitle}{DM sheds light (up)on flavor anomalies}

\thispagestyle{empty}
\begin{center}

\Huge\textbf{\thetitle}
\vfill
% Note that the size is given in normal parentheses
% instead of curly brackets.
% Define external vertices from bottom to top
\vfill

\end{center}
\newpage
% % >>> Gutachterseite <<<
% \thispagestyle{empty}
% \newpage
% \cleardoublepage




% >>> Kurzfassung/Abstract <<<


% >>> Hauptteil <<<

%\addcontentsline{toc}{chapter}{Inhaltsverzeichnis}


\pagenumbering{arabic}
\setcounter{page}{1}

\section{Model outline}
The SM-group $SU(3)_C\times SU(2)_L\times U(1)_{Y_W}$ gets extended by additional groups as $U(1)_{B'}\times U(1)_{L'}\times U(1)_{\chi}$ 
\cite{Grip} and even three more $G_F = A_4 \times U(1)_\text{FN} \times Z_3$ as flavour groups. Following \cite{Grip}, we introduce a new fermion $\psi$ which eventually serves
as a DM particle and two scalars $\Phi_l$ and $\Phi_q$ coupling to leptons or quarks, respectively. Some representations for $\Phi_l$ were just taken out
for several interactions coming up which would cause problems. So $G_F$ shall prohibit these interactions by charging the respective particles so that
the representations themselves are again enabled.\\
\noindent This can be achieved in many ways depending on what you want to get in the end. Our main interest here is a particular position for the muon in
order to explain anomalies in the $B$-sector, e.g. $R_K$. Therefore, we want $\Phi_l$ to only couple to muons which is achieved by a certain pattern 
of representing the particles which can be seen in table \ref{tab_model}.
\begin{table}
 \begin{tabular}{c|c|c|c}
  Field & $SU(3)_C\times SU(2)_L\times U(1)_{Y_W}$ & $A_4 \times U(1)_\text{FN} \times Z_3$ & $U(1)_\chi$\\
  \hline
  $Q^i_L$ & (3,2,$\frac16$) & (1,$a^i$,$\omega$) & 0\\
  $U^i_R$ & (3,1,$\frac23$) & (1,$b^i$,$\omega^2$)& 0\\
  $D^i_R$ & (3,1,$-\frac13$) & (1,$c^i$,$\omega^2$)& 0\\
  $L^i_L$ & (1,2,$-\frac12$) & (3,0,$\omega$)& 0\\
  $E^i_R$ & (1,1,$-1$) & ($1 {^(} {'} {^,} '' {^)} $,$d^i$,$\omega^2$)& 0\\
  $\Phi_H$ & (1,2,$\frac12$) & (1,0,1)& 0\\
  \hline
  $\psi$ & (1,1,0) & (1,0,$\omega^2$)& 1\\
  $\Phi_l$ & (1,2,-$\frac12$) & ($1''$,0,1)& 1\\
  $\Phi_q$ & (3,2,$\frac16$) & ($1$,0,1)& 1\\
  \hline
  $\Phi_T$ & (1,1,0) & ($3$,0,1)& 0\\

 \end{tabular}
\caption{Transformation rules for our SM and BSM fields. The index $i$ stands for the three generations which influence the charges under the 
$U(1)_\text{FN}$ $a-d$. The representation for $E_R$ also depends on the generation ($e^c \sim 1$, $\mu \sim 1''$, $\tau \sim 1'$).}
\label{tab_model}
\end{table}
The Yukawa part of our renormalizable lagrangian is given by
\begin{align}
 \mathcal{L}_\text{yuk} = \alpha^q_i \bar \psi_R Q^i_L \Phi_q + \alpha^l_i\bar \psi_R L^i_L \Phi_l \frac{\Phi_T}{\Lambda_A} + \text{h.c.}
\end{align}
with the couplings $\alpha$. The scalar $\Phi_T$ helps to make it $A_4$ invariant in a way that $\Phi_l$ only couples to muons. At the scale $\Lambda_A$ 
the $A_4$ group breaks spontaneously and gives $\Phi_T$ a vacuum expectation value $\langle \Phi_T \rangle = v_T = \epsilon \Lambda_A$ with a small parameter
$\epsilon \approx 0.2$.

\textbf{The lagrangian has additional FN-terms depending on their respective charge. In order to save free parameters I would guess that the breaking scales
of the groups might be equal (Feruglio et al. do this). The A4 ensures the interaction to muons only. I don't see how to give the b-mesons a special role
as well, so we do not observe something in the $K\rightarrow \pi \mu\mu / K \rightarrow \pi e e$ fraction. Direct DM decay $\psi\rightarrow fff$ appears to
be forbidden by at least one group in the end. DM annihilation $\psi\psi\rightarrow ff$ occurs at tree level.}


\section{Possible processes to look at}
Now we can check on some processes to give estimations for the masses of our BSM particles.
\subsection{anomalous magnetic momentum of the muon $\Delta a_\mu$}
This is probably the easiest process in this model since it occurs at one-loop level and has no hadronic contributions. The Feynman graph can be seen
in figure \ref{pic_feyn} and will be computed now and follows mainly \cite{Peskin}.
\begin{figure}
 \includegraphics[width=0.7\textwidth]{feynmg2.jpg}
 \centering
 \caption{Feynman graph contributing to the anomalous magnetic moment of the muon. Since $\psi$ is electrically neutral, this is the only diagram}
 \label{pic_feyn}
\end{figure}
The matrix element $M$ reads
\begin{align}
 M^\rho = \lambda_\mu^2 e \int \frac{\dx^4 k}{(2\pi)^4} \frac{(\slashed{k}+\slashed{p} + m)(2k-q)^\rho}{[(k+p)^2-m^2+\text{i}\epsilon][k^2-M^2+\text{i}\epsilon][(k-q)^2-M^2+\text{i}\epsilon]}
\end{align}
contracting with the polarization vector $\epsilon_\rho$ of the outgoing photon. The masses are $m=m_\psi$ and $M = M_{\Phi_l}$. Using Feynman 
parametrization, this can be written as
\begin{align}
 M^\rho = \lambda_\mu^2 e \int \frac{\dx^4 k}{(2\pi)^4} (\slashed{k}+\slashed{p} + m)(2k-q)^\rho \int\limits_0^1 \dx x \dx y \dx z \delta(x+y+z-1) \frac{2}{D^3}
\end{align}
with the denominator $D$ we want to facilitate now.
\begin{align}
 D =& x(k^2-M^2 + y((k-q)^2-M^2) + z((k+p)^2-m^2) + (x+y+z)\text{i}\epsilon\\
 =& k^2 + 2k(zp-yq) + yq^2 + zp^2 - M^2(x+y) - m^2z + \text{i}\epsilon
\end{align}
where we want to shift $k$ to $l=k-yq+zp$ which leaves $D$ being quadratic in $l$
\begin{align}
 D = l^2 - \Delta + \text{i}\epsilon,
\end{align}
with 
\begin{align}
 \Delta := -yq^2 -zp^2 + M^2(x+y)-m^2z.
\end{align}
The numerator gets shifted as well. After keeping only the terms quadratic in $l$ or not depending on $l$ at all, we are left with
\begin{align}
 N^\rho = 2\slashed{l} l^\rho + y^2\slashed{q}q^\rho + z^2\slashed{p}p^\rho + yz(\slashed{q}p^\rho+\slashed{p}q^\rho) + y\slashed{q}q^\rho + z\slashed{p}q^\rho + \slashed{p}q^\rho \\ + m(yq^\rho-q^\rho-zp^\rho)    .
\end{align}
Now we can perform the momentum integral. That for the term constant in $l$ is non-divergent and gives a finite expression 
\begin{align}
 \int \frac{\dx^4 l}{(2\pi)^4} \frac{1}{(l-\Delta)^m} =& \frac{\text{i}(-1)^m}{(4\pi)^2}\frac{1}{(m-1)(m-2)} \frac{1}{\Delta^{m-2}} \\
 \stackrel{m=3}{=}& \frac{-\text{i}}{(4\pi)^2}\frac{1}{2\Delta}.
\end{align}
If we want to compute the other one, we encounter a divergent integral. We use dimensional regularization and a Master formula from a problem sheet 
from G. Hiller
\begin{align}
 \int \frac{\dx^d l}{(2\pi)^d} \frac{l^\mu l^\nu}{(l^2-\Delta)^n} =& \frac{\text{i}g^{\mu\nu}}{32\pi^2} \frac{\Gamma(n-3+\epsilon)}{\Gamma(n)}\left(-\frac{1}{\Delta}\right)^{n-3}\left(\frac{4\pi \mu^2}{\Delta}\right)^\epsilon\\
 =& \frac{\text{i}g^{\mu\nu}}{64\pi^2} \left(\frac{1}{\epsilon}-\gamma_E + \ln\left(\frac{4\pi\mu^2}{\Delta} \right) +\mathcal{O}(\epsilon) \right)
\end{align}
with the generalized factorial $\Gamma(x)$, the dimension $d=4-2\epsilon$ and $\mu$ to ensure the action to be dimensionless. In the second step
$\Gamma(\epsilon)$ and the last fraction were expanded. \\
\textbf{This is where I got so far. I try to figure out how to deal with the divergence in $M$ and how to get the $\Delta a_\mu$ out of this calculus.
So far I'd like to know, if you see problems with the representations and maybe with the calculus (which I will rerun at least one more time.)}


\subsection{$B \rightarrow K\mu\mu$}
\subsection{$b\rightarrow s\gamma$}
\subsection{$\bar B_s B_s$ - mixing}
\subsection{DM-annihilation $\bar \psi\psi \rightarrow \bar f f$ }



\newpage
\end{spacing}
\newpage
\begin{thebibliography}{xxx}
 \bibitem[1]{Grip}B. Gripaios et al. \textit{Linear flavour violation and anomalies in $B$ physics}\\ \href{http://arxiv.org/abs/1509.05020v1}{http://arxiv.org/abs/1509.05020v1}
 \bibitem[2]{Peskin}M. Peskin et al. \textit{An Introduction To Quantum Field Theory}\\ ISBN: 0-201-50397-2
\end{thebibliography}
\end{document}
