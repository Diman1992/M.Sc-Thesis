
\documentclass[hyperref={pdfpagelabels=false}]{beamer}
% Die Hyperref Option hyperref={pdfpagelabels=false} verhindert die Warnung:
% Package hyperref Warning: Option `pdfpagelabels' is turned off
% (hyperref)                because \thepage is undefined. 
% Hyperref stopped early 
%
% 
% Deutsche Spracheinstellungen
\usepackage[english,english]{babel, varioref}
\usepackage[T1]{fontenc}
\usepackage[utf8]{inputenc}

%\usepackage{marvosym}

\usepackage{amsfonts}
\usepackage{amssymb}
\usepackage{amsmath}
\usepackage{amscd}
\usepackage{amstext}
\usepackage{float}
\usepackage{caption}
\usepackage{wrapfig}
\usepackage{setspace}
%\usepackage[onehalfspacing]{setspace}
\usepackage{threeparttable}
\usepackage{footnote}
\usepackage{feynmf}
\usepackage{bbm}
\usepackage{slashed}
\usepackage{textcomp}
\usepackage{multirow}
\usepackage{courier}
\usepackage{listings}
\usepackage{color}
%\usepackage{minipage}
 
 \definecolor{middlegray}{rgb}{0.5,0.5,0.5}
 \definecolor{lightgray}{rgb}{0.8,0.8,0.8}
 \definecolor{orange}{rgb}{0.8,0.3,0.3}
 \definecolor{yac}{rgb}{0.6,0.6,0.1}
 \definecolor{puple}{rgb}{0.62,0.12,0.94}
 \lstset{language=Python,
                basicstyle=\ttfamily,
                keywordstyle=\color{red}\ttfamily,
                stringstyle=\color{magenta}\ttfamily,
                commentstyle=\color{blue}\ttfamily,
                morecomment=[l][\color{blue}]{\#},
		%stepnumber=1,
		%numberstyle=\color{magenta}\ttfamily,
		%    numbers=left,
		%    numberstyle={},
		%    numberblanklines=false,
		%    stepnumber=1,
		%    numbersep=10pt,
		    xleftmargin=15pt,
 		moredelim=[is][\color{purple}]{|}{|}
}

\newfloat{formel}{htbp}{for}
\floatname{formel}{Formel}

\onehalfspacing
%\setstretch {1.433}

\usepackage{longtable}

%\usepackage{bibgerm}

\usepackage{footnpag}

\usepackage{ifthen}                 %%% package for conditionals in TeX
\usepackage[amssymb]{SIunits}
%Fr textumflossene Bilder und Tablellen
%\usepackage{floatflt} - veraltet

%Fr Testzwecke aktivieren, zeigt labels und refs im Text an.
%\usepackage{showkeys}

% Abstand zwischen zwei Abs�zen nach DIN (1,5 Zeilen)
% \setlength{\parskip}{1.5ex}

% Einrckung am Anfang eines neuen Absatzes nach DIN (keine)
%\setlength{\parindent}{0pt}

% R�der definieren
% \setlength{\oddsidemargin}{0.3cm}
% \setlength{\textwidth}{15.6cm}

% bessere Bildunterschriften
\usepackage{caption2}


% Probleml�ungen beim Umgang mit Gleitumgebungen
\usepackage{float}

% Nummeriert bis zur Strukturstufe 3 (also <section>, <subsection> und <subsubsection>)
%\setcounter{secnumdepth}{3}

% Fhrt das Inhaltsverzeichnis bis zur Strukturstufe 3
%\setcounter{tocdepth}{3}

\usepackage{exscale}

\newenvironment{dsm} {\begin{displaymath}} {\end{displaymath}}
\newenvironment{vars} {\begin{center}\scriptsize} {\normalsize \end{center}}


\newcommand {\en} {\varepsilon_0}               % Epsilon-Null aus der Elektrodynamik
\newcommand {\lap} {\; \mathbf{\Delta}}         % Laplace-Operator
\newcommand {\R} { \mathbb{R} }                 % Menge der reellen Zahlen
\newcommand {\e} { \ \mathbf{e} }               % Eulersche Zahl
\renewcommand {\i} { \mathbf{i} }               % komplexe Zahl i
\newcommand {\N} { \mathbb{N} }                 % Menge der nat. Zahlen
\newcommand {\C} { \mathbb{C} }                 % Menge der kompl. Zahlen
\newcommand {\Z} { \mathbb{Z} }                 % Menge der kompl. Zahlen
\newcommand {\limi}[1]{\lim_{#1 \rightarrow \infty}} % Limes unendlich
\newcommand {\sumi}[1]{\sum_{#1=0}^\infty}
\newcommand {\rot} {\; \mathrm{rot} \,}         % Rotation
\newcommand {\grad} {\; \mathrm{grad} \,}       % Gradient
\newcommand {\dive} {\; \mathrm{div} \,}        % Divergenz
\newcommand {\dx} {\; \mathrm{d} }              % Differential d
\newcommand {\cotanh} {\; \mathrm{cotanh} \,}   %Cotangenshyperbolicus
\newcommand {\asinh} {\; \mathrm{areasinh} \,}  %Area-Sinus-Hyp.
\newcommand {\acosh} {\; \mathrm{areacosh} \,}  %Area-Cosinus-H.
\newcommand {\atanh} {\; \mathrm{areatanh} \,}  %Area Tangens-H.
\newcommand {\acoth} {\; \mathrm{areacoth} \,}  % Area-cotangens
\newcommand {\Sp} {\; \mathrm{Sp} \,}
\newcommand {\mbe} {\stackrel{\text{!}}{=}}     %Must Be Equal
\newcommand{\qed} { \hfill $\square$\\}
\newcommand{\midtilde}{\raisebox{-0,25\baselineskip}{\textasciitilde}}
\renewcommand{\i} {\imath}
\def\captionsngerman{\def\figurename{\textbf{Abb.}}}

%%%%%%%%%%%%%%%%%%%%%%%%%%%%%%%%%%%%%%%%%%%%%%%%%%%%%%%%%%%%%%%%%%%%%%%%%%%%
% SWITCH FOR PDFLATEX or LATEX
%%%%%%%%%%%%%%%%%%%%%%%%%%%%%%%%%%%%%%%%%%%%%%%%%%%%%%%%%%%%%%%%%%%%%%%%%%%%
%%%
\ifx\pdfoutput\undefined %%%%%%%%%%%%%%%%%%%%%%%%%%%%%%%%%%%%%%%%% LATEX %%%
%%%
\usepackage[dvips]{graphicx}       %%% graphics for dvips
\DeclareGraphicsExtensions{.eps,.ps}   %%% standard extension for included graphics
\usepackage[ps2pdf]{thumbpdf}      %%% thumbnails for ps2pdf
\usepackage[ps2pdf,                %%% hyper-references for ps2pdf
bookmarks=true,%                   %%% generate bookmarks ...
bookmarksnumbered=true,%           %%% ... with numbers
hypertexnames=false,%              %%% needed for correct links to figures !!!
breaklinks=true,%                  %%% breaks lines, but links are very small
linkbordercolor={0 0 1},%          %%% blue frames around links
pdfborder={0 0 112.0}]{hyperref}%  %%% border-width of frames
%                                      will be multiplied with 0.009 by ps2pdf
%
\hypersetup{ pdfauthor   = {Dimitrios Skodras},
pdftitle    = {Fermionic Dark Matter and its Role on B Anomalies}, pdfsubject  = {masterthesis}, pdfkeywords = {dark matter},
pdfcreator  = {LaTeX with hyperref package}, pdfproducer = {dvips
+ ps2pdf} }
%%%
\else %%%%%%%%%%%%%%%%%%%%%%%%%%%%%%%%%%%%%%%%%%%%%%%%%%%%%%%%%% PDFLATEX %%%
%%%
\usepackage[pdftex]{graphicx}      %%% graphics for pdfLaTeX
\DeclareGraphicsExtensions{.pdf}   %%% standard extension for included graphics
\usepackage[pdftex]{thumbpdf}      %%% thumbnails for pdflatex
\usepackage[pdftex,                %%% hyper-references for pdflatex
bookmarks=true,%                   %%% generate bookmarks ...
bookmarksnumbered=true,%           %%% ... with numbers
hypertexnames=false,%              %%% needed for correct links to figures !!!
breaklinks=true,%                  %%% break links if exceeding a single line
linkbordercolor={0 0 1},
linktocpage]{hyperref} %%% blue frames around links
%                                  %%% pdfborder={0 0 1} is the default
\hypersetup{
pdftitle    = {Fermionic Dark Matter and its Role on B Anomalies}, %right place
pdfsubject  = {master thesis}, 
pdfkeywords = {V301, Innenwiderstand, Leistungsanpassung},
pdfsubject  = {Protokoll AP},
pdfkeywords = {V301, Innenwiderstand, Leistungsanpassung}}
%                                  %%% pdfcreator, pdfproducer,
%                                      and CreationDate are automatically set
%                                      by pdflatex !!!
\pdfadjustspacing=1                %%% force LaTeX-like character spacing
\usepackage{epstopdf}
%
\fi %%%%%%%%%%%%%%%%%%%%%%%%%%%%%%%%%%%%%%%%%%%%%%%%%%% END OF CONDITION %%%
%%%%%%%%%%%%%%%%%%%%%%%%%%%%%%%%%%%%%%%%%%%%%%%%%%%%%%%%%%%%%%%%%%%%%%%%%%%%
% seitliche Tabellen und Abbildungen
%\usepackage{rotating}
\usepackage{ae}
\usepackage{
  array,
  booktabs,
  dcolumn
}
\makeatletter 
  \renewenvironment{figure}[1][] {% 
    \ifthenelse{\equal{#1}{}}{% 
      \@float{figure} 
    }{% 
      \@float{figure}[#1]% 
    }% 
    \centering 
  }{% 
    \end@float 
  } 
  \makeatother 


  \makeatletter 
  \renewenvironment{table}[1][] {% 
    \ifthenelse{\equal{#1}{}}{% 
      \@float{table} 
    }{% 
      \@float{table}[#1]% 
    }% 
    \centering 
  }{% 
    \end@float 
  } 
  \makeatother 
%\usepackage{listings}
%\lstloadlanguages{[Visual]Basic}
%\allowdisplaybreaks[1]
%\usepackage{hycap}
%\usepackage{fancyunits}

% \input{global.tex}

\usepackage{lmodern}
% Das Paket lmodern erspart die folgenden Warnungen:
% LaTeX Font Warning: Font shape `OT1/cmss/m/n' in size <4> not available
% (Font)              size <5> substituted on input line 22.
% LaTeX Font Warning: Size substitutions with differences
% (Font)              up to 1.0pt have occurred.
%

% % % % % % % % % % % % % % % % % % % % % % % % % % % % % % % % % % % % % % % % % % % %
\usepackage{siunitx}
\sisetup{load-configurations=abbreviations}
\sisetup{
	%locale=DE,
	seperr=true,                    % Fehler anzeigen
	tightpm,                        % Abstand zwischen Fehler verringern
	tophrase={{\text{ bis }}},
	fraction=nice,
	per-mode=fraction,
	free-standing-units=true,
	space-before-unit=true,
	use-xspace=true,
	group-separator={{\text{~}}},
	list-final-separator={{\text{ und }}}
}
\usepackage{natbib}
\usepackage[labelformat=empty]{caption}
\usepackage{movie15}
\usepackage{xcolor,colortbl}
\usepackage{slashed}
\usepackage{amsfonts}
\usepackage{amssymb}
\usepackage{amsmath}
\usepackage{amscd}
\usepackage{amstext}
\usepackage[ngerman,german]{babel, varioref}
\usepackage[T1]{fontenc}
\usepackage[utf8]{inputenc}
\usepackage{xfrac}
\usepackage{booktabs}
\usepackage{dsfont}

\usepackage{pgf,tikz}
\usetikzlibrary{arrows}
% % % % % % % % % % % % % % % % % % % % % % % % % % % % % % % % % % % % % % % % % % % % % % % % %
% Wenn \titel{\ldots} \author{\ldots} erst nach \begin{document} kommen,
% kommt folgende Warnung:
% Package hyperref Warning: Option `pdfauthor' has already been used,
% (hyperref) ... 
% Daher steht es hier vor \begin{document}

\title{Handling Family Issues with Discretion}  
\institute{TU Dortmund \\\vspace{1cm} 2nd BCD International School of High Energy Physics}
\author{Dimitrios Skodras}
\date{14.04.2016 - Cargèse, Corsica} 

% zusaetzlich ist das usepackage{beamerthemeshadow} eingebunden 
\usepackage{beamerthemeshadow}


%  \beamersetuncovermixins{\opaqueness<1>{25}}{\opaqueness<2->{15}}
%  sorgt dafuer das die Elemente die erst noch (zukuenftig) kommen 
%  nur schwach angedeutet erscheinen 
\beamersetuncovermixins{\opaqueness<1>{25}}{\opaqueness<2->{15}}
% klappt auch bei Tabellen, wenn teTeX verwendet wird\ldots

\beamertemplatenavigationsymbolsempty

\begin{document}

\setbeamertemplate{footline}
{%
  \leavevmode%
 \begin{beamercolorbox}%
    [wd=.5\paperwidth,ht=2.5ex,dp=1.125ex,leftskip=.3cm,rightskip=.3cm]%
    {author in head/foot}%
    \usebeamerfont{author in head/foot}%
    \hfill\insertshortauthor
  \end{beamercolorbox}%
  \begin{beamercolorbox}%
    [wd=.5\paperwidth,ht=2.5ex,dp=1.125ex,leftskip=.3cm ,rightskip=.3cm]%
    {title in head/foot}%
    \usebeamerfont{title in head/foot}%
    \insertshorttitle\hfill\insertframenumber{}
  \end{beamercolorbox}%
}%

\setbeamertemplate{caption}{\raggedright\insertcaption\par}
\captionsetup[figure]{font=small,skip=0pt}


\begin{frame}
\titlepage
\end{frame} 

\begin{frame}
\frametitle{What's the Matter}
  \begin{itemize}
   \item masses of charged fermions are strongly hierarchical
   \begin{align*}
    &m_u : m_c : m_t \approx \lambda^8 : \lambda^4:1; \quad m_d : m_s : m_b \approx \lambda^4 : \lambda^2:1\\
    &m_e : m_\mu : m_\tau \approx \lambda^5 : \lambda^2:1; \quad \text{with}\quad \lambda = \sin\theta_C \approx 0.2\\
   \end{align*}
   \item mass hierachy of neutrinos are milder and the ordering so far unknown
  \end{itemize}
   \begin{figure}[H]
 \includegraphics[width = 0.9\textwidth]{pics/massPattern.jpg}

 \centering
 \end{figure}

\end{frame} 
\begin{frame}
\frametitle{What's the Mixing}
\begin{minipage}{0.49\textwidth}
3 small mixing angles,  \\ 1 large phase
\begin{align*}
 &\theta_{12} = 13.0(1)^\circ, \, \theta_{23} = 2.4(1)^\circ\\
 &\theta_{13} = 0.2(1)^\circ, \, \delta_{\text{CKM}} = 69(4)^\circ
\end{align*}
$V_\text{CKM}$ is almost unity
\begin{figure}
 \includegraphics[width = 0.5\textwidth]{pics/ckmPattern.jpg}
 \caption{CKM-Matrix}
 \centering 
\end{figure}
\end{minipage}
\begin{minipage}{0.49\textwidth}
%\vspace{-0.5cm}
2 large mixing angles,\\ 1 small one,
\begin{align*}
 &\theta_{12} = 33.4(8)^\circ, \, \theta_{23} = 38.5(13)^\circ\\
 &\theta_{13} = 8.7(4)^\circ 
\end{align*}

$V_\text{PMNS}$ is far away from unity

 \begin{figure}
 \includegraphics[width = 0.5\textwidth]{pics/pmnsPattern.jpg}
 \caption{PMNS-Matrix}
 \centering 
 \end{figure}
\end{minipage}
\end{frame}

\begin{frame}
 \frametitle{What could we do}
 \begin{minipage}{0.4\textwidth}
Nothing
 \end{minipage}
 \begin{minipage}{0.5\textwidth}
  \includegraphics[width=1\textwidth]{pics/sad_clown.jpg}
 \end{minipage}
 
Or something not that discouraging: \\Introduce a family symmetry $G_F$ which may explain these patterns


\end{frame}

% \begin{frame}
%  \frametitle{What are the options}
%  \begin{minipage}{0.45\textwidth}
%   \begin{itemize}
%    \item spontaneous symmetry breaking
%    \item[$\rightarrow$] Goldstone bosons?
%    \item 
%   \end{itemize}
% 
%  \end{minipage}
%  \begin{minipage}{0.45\textwidth}
%   \begin{itemize}
%    \item explicit symmetry breaking\\
%    \item[$\rightarrow$] why approximate and not wholly absent?
%   \end{itemize}
% 
%  \end{minipage}
% 
% 
% \end{frame}

\begin{frame}
\frametitle{What do we do}
 \begin{itemize}
  \item We use $T'$, which is a\\
 \textit{discrete} (avoid Goldstone-bosons)\\
 \textit{non-abelian} (unify two of three generations)\\
 group and plays the key role here.\\
 \item A $U(1)_\text{FN}$ to produce the necessary mass suppressions which $T'$ cannot induce.
 \item And a final $Z_3$ which seperates the contributions to neutrino and to charged fermion masses
 \end{itemize}
 Our full symmetry group is therefore $G_F = T'\times U(1)_\text{FN} \times Z_3$\\ \tiny{(Feruglio et al.: hep-ph:0702194v2)}
\end{frame}

\begin{frame}
 \frametitle{How is it built}
 $T'$ has 7 irreducible represenations $D$: 1, 1', 1'', 3, 2, 2', 2''.\\
 with its characters $\chi = \text{tr}(D)$ one can derive the multiplication rules $D\otimes D'$.
 \begin{table}[l]
  \begin{tabular}{c|ccccccccccc}
   Field&$l$&$e^c$&$\mu^c$&$\tau^c$&$D_q$&$D_u$&$D_d$&$q_3$&$t^c$&$b^c$&$h_{u,d}$\\
   $T'$& 3 & 1 & 1' & 1'' & 2'' & 2'' & 2'' & 1 & 1 & 1&1\\
   \hline
   Field&$\phi_T$&$\eta$&$\xi''$&$\phi_S$&$\xi,\tilde{\xi}$\\
   $T'$ & 3 & 2' & 1'' & 3 & 1
  \end{tabular}
  \caption{Transformation rules of the fermion und scalar fields}
 \end{table}
The fermion fields are thought of as
\begin{align*}
 &l = ((e,\nu_e),(\mu,\nu_\mu),(\tau,\nu_\tau))^t,\qquad D_q = ((u,d),(c,s))^t,\\ 
 &q_3 = (t,b)^t \qquad D_u = (u^c,c^c)^t, \qquad D_d = (d^c,s^c)^t
\end{align*}



\end{frame}

\begin{frame}
 \frametitle{How does it break}
 The scalars obtain a vacuum expectation value due to spontaneous symmetry breaking of $T'$ at a scale $\Lambda$ ($\gg \Lambda_\text{EW}$) as
 \begin{align*}
  &\langle\phi_T\rangle = (v_T,\delta v_{T2}, \delta v_{T3}),\qquad \langle\eta\rangle = (v_1,\delta v_2), \qquad \langle\xi''\rangle = \delta u'',\\
  &\langle\phi_S\rangle = (v_S, v_S, v_S), \qquad \langle\xi\rangle = u, \qquad \langle\tilde{\xi}\rangle = \delta u,
 \end{align*}
where $\frac{x}{\Lambda} = \lambda^2$ and $\frac{\delta x}{\Lambda} = \lambda^4$, with $x$ being any mentioned vev.
\end{frame}

\begin{frame}
 \frametitle{What do we get}
 $\phi_S$, $\xi$, $\tilde{\xi}$ only couple to neutrinos (due to $Z_3$ charge)\\
 $\phi_T$ couples to all charged fermions\\
 $\eta$ connects the 2nd and 3rd quark generation\\
 $\xi''$ mediates between 2nd and 1st quark generation\\
 \vspace{1cm}
 This yields directly a diagonal mass matrix for the charged leptons 
 \begin{align*}
m_l = \frac{v_Tv_d}{\sqrt{2}\Lambda}  \text{diag}(y_e,y_\mu,y_\tau)
 \end{align*}
 with the yukawa couplings $y_l$ getting their values from appropriate powers of $\left(\frac{\langle\theta\rangle}{\Lambda}\right)^n = \lambda^2$ and
 $\theta$ being the field breaking the Froggat-Nielsen symmetry
\end{frame}

\begin{frame}
 
\end{frame}




\end{document}
