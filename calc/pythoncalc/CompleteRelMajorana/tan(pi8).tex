\documentclass[11pt,a4paper,twoside]{article}
\input{longheader}
\usepackage{fancyhdr}
\pagestyle{fancy}
\fancyhf{}
%Fußzeile mittig (Seitennummer)
\fancyfoot[C]{$\int_{K\{0\}} \dx r \, a_s$}
%Linie unten


\begin{document}

\section*{Completeness Relation for Majorana Fermions}
Unlike Dirac fermions, Majorana fermions are their own anti-particles and therefore obey different Feynman-rules, e.q. when computing cross sections.
If two Dirac fermions meet, depicted by their spinors $v$ and $u$, they couple with a coupling constant $g$ to a mediator with a certain type of coupling
$\Gamma$. The part of the amplitude $M$ which contains the vertex part reads
\begin{equation}
 M_\text{vertex} = g\left[\bar v_1 \Gamma u_2\right].
\end{equation}
% 
% \begin{equation}
%  -\text{i}M(e^+e^-\rightarrow e'^+e'^-) = \left[(\bar v_{e^+} (\text{i}e\gamma_\mu) u_{e^-}\right]\frac{-\text{i}g^{\mu\nu}}{k^2+\text{i}\epsilon} \left[\bar u_{e'^-} (\text{i}e\gamma_\nu) v_{e'^+}\right].
% \end{equation}
During the following computation of the cross section one has to use Casimir's trick at one point in the unpolarised case, i.e. average of initial spins and
sum over final spins. This translates the summation over the final spins to tracing operations of Dirac-matrices $\gamma$. Therefore one needs the completeness 
relations which are
\begin{align}
 &\sum_{s=1,2} u_p^{(s)} \bar u_p^{(s)} = \slashed{p} + m \qquad \text{for particles} \\
 &\sum_{s=1,2} v_p^{(s)} \bar v_p^{(s)} = \slashed{p} - m \qquad \text{for anti-particles},
\end{align}
where $\bar a = a^\dagger \gamma^0$ and with momentum $p$, feynman-slash notation $\slashed{a}=a_\mu\gamma^\mu$, mass $m$ and the spinors as
\begin{align}
 &u^{(s)}(p) = \frac{\slashed{p}+m}{\sqrt{2(E+m)}} \begin{pmatrix}
                                             \xi_s \\ \xi_s
                                            \end{pmatrix}\quad \text{and}\quad  v^{(s)}(p) = \frac{-\slashed{p}+m}{\sqrt{2(E+m)}} \begin{pmatrix}
                                             \xi_s \\ -\xi_s
                                            \end{pmatrix}\\
 &\xi_1^T = (1\,\,\,0), \qquad \xi_2^T = (0\,\,\,1), \qquad 
 \label{eq_spinors}
\end{align}
But if we have two Majorana fermions coupling to each other we have a different vertex expression
\begin{equation}
 M_\text{vertex} = g\left[\bar v_1 \Gamma u_2 - \bar u_1 \Gamma v_2\right],
\end{equation}
where the minus sign follows from Fermi-Dirac statistics. There will also arise terms like $u\bar u$ during the computation with identical completeness
relations as for the Dirac case. But what is about terms as $v\bar u$ and $u\bar v$? Starting with the spinor representation in \eqref{eq_spinors} leads
to 
\begin{align}
 &\sum_{s=1,2} u_p^{(s)} \bar v_p^{(s)} = U(-\slashed{p} + m) \\
 &\sum_{s=1,2} v_p^{(s)} \bar u_p^{(s)} = U(+\slashed{p} + m), 
\end{align}
with a matrix $U$ which is build as
\begin{align}
 U = \gamma^0 \gamma_5 = \text{i} \gamma^1 \gamma^2 \gamma^3
\end{align}
% Some properties of $U$:
% \begin{itemize}
%  \item $U^2 = -I_4$
%  \item $\{U,\gamma^0\} = \{U,\gamma_5\} = 0$
%  \item $[U,\gamma^i] = 0$
%  \item $U\gamma^\mu U = \left(\gamma^\mu\right)^\dagger$
% \end{itemize}





\end{document}
