
\subsection{$A_4\times Z_3$ and Discrete Groups}
\label{sec_A4GT}
The democratic pattern of the PMNS matrix \eqref{eq_pmns} shows that an addition of just a single $\FN$ flavour symmetry is not viable. Some efforts
were made to explain the lepton mixing with discrete groups but to cover both, the lepton and the quark sector, one has to try quite large groups
\cite{1605.03581}. We will use the smallest group containing a triplet, namely $A_4$ for the leptons. The quarks are trivially charged.
As a remark, its double cover $T'$ exhibits doublet representations which could be used for quark sector phenomenology \cite{Tprime}. 
\\ \\ \textit{Properties of discrete groups}\\
\noindent  We will now derive the representations of a discrete group and how the tensor product with one another can be decomposed \cite{1003.3552}. 
Consider the operator $A^{\alpha\beta}_{jl}$ \cite{georgi}
\begin{align}
 A^{\alpha\beta}_{ij} := \sum\limits_{g\in G} D_\alpha(g^{-1})|\alpha,i\rangle\langle \beta,j|D_\beta(g)
 \label{eq_orthoOp}
\end{align}
where $D_{\alpha,\beta}$ are irreducible representations. Computation yields $D_\alpha(g_1)A^{\alpha\beta}_{ij} = A^{\alpha\beta}_{ij}D_\beta(g_1)$. Schur's lemma now implies that
 $A^{\alpha\beta}_{ij}=\delta^{\alpha\beta} \lambda_{ij} I$ with the idendity matrix $I$. Taking the trace and applying its cyclic property to \eqref{eq_orthoOp}
results in
\begin{align}
 \text{Tr}A^{\alpha\beta}_{ij} &= \delta^{\alpha\beta}\text{Tr}\left(\lambda_{ij} I\right) = \delta^{\alpha\beta}\lambda_{ij} \text{Tr}I = \delta^{\alpha\beta}\lambda_{ij}n_\alpha,\\
 \text{Tr}A^{\alpha\beta}_{ij} &= \delta^{\alpha\beta}\sum\limits_{g\in G} \langle \alpha,i\left| D_\alpha(g)D_\alpha(g^{-1}) \right|\alpha,j\rangle = N_G \delta^{\alpha\beta}\delta_{ij}
 \label{eq_orthoRelA}
\end{align}
with the dimension of the representation $n_\alpha$ and the number of elements (order) in the group $N_G$. We further define characters as the trace of
representations $\chi^{(\alpha)}(g):= \text{Tr}D_\alpha(g)$ and introduce the term conjugacy class $C$ which is a set of group elements $g$ and $a$ which 
fulfil $gag^{-1}=a$.  The number of irreducibles with a multiplicity $m$ is equal to the number of conjugacy classes $N_C$. From \eqref{eq_orthoRelA} the 
orthonormality relations for characters can be written as
\begin{align}
 &\sum\limits_{g\in G} \chi^{(\alpha)}(g^{-1})\chi^{(\beta)}(g) = N_G \delta^{\alpha\beta},\label{eq_orthoRelChi1}\\
 &\sum\limits_\alpha \chi^{(\alpha)}(g^{-1}_i)\chi^{(\alpha)}(g_j) = \frac{N_G}{N_i} \delta_{C_iC_j}.
 \label{eq_orthoRelChi2}
\end{align}
The number of elements in conjugacy class $C_i$ is denoted by $N_i$. One conjugacy class $C_1$ only consists of the idendity $e$ and its character 
is hence always the dimension of the respective representation. Using this fact and \eqref{eq_orthoRelChi2}, we can deduce the dimension and the
multiplicity of representations with 
\begin{align}
 &\sum\limits_i m_i = N_C,\label{eq_multi1}\\
 &\sum\limits_i m_i n_i ^2 = N_G.\label{eq_multi2}
\end{align}
There exists for each conjugacy class and each representation a character which can be put into a character table which is exemplarily shown
for $A_4$ in table \ref{tab_charactertable}. The scalar product of two characters is defined as 
$\langle \chi^{(\alpha)},\chi^{(\beta)}=\delta^{\alpha\beta}$ in resemblance to \eqref{eq_orthoRelChi1}. This can be used to determine 
the multiplicity of a distinct representation $D_\gamma$ in a given tensor product $D_\alpha\times D_\beta$ \cite{1273369} 
\begin{align}
 m_\gamma = \langle \chi_\alpha \chi_\beta,\chi_\gamma\rangle.
 \label{eq_multiplicity}
\end{align}
\\ \\ \textit{Flavour pattern}\\
\noindent Following \cite{0512103}, the flavour symmetry associated to $A_4$ 
(sec. \ref{sec_appendixA4}) may be broken into its subgroups by a vev of a triplet 
$\phi=(\phi_1,\phi_2,\phi_3)$ of scalar fields $\phi_i$. With a vacuum alignment as $\langle\phi_S\rangle = (1,1,1)v_S$ it breaks down to $G_S$
and with $\langle\phi_T\rangle=(1,0,0)v_T$ to $G_T$. Another scalar $\xi$ gets introduced also developing a vev $\langle\xi\rangle=u$. 
A discrete $Z_3$ is further needed to seperate the neutrino from the charged lepton sector and to prevent unwanted couplings.
With the charge assignment for the leptons and the scalars from table \ref{tab_a4charges} one can write down the following Lagrangian
\begin{table}
 \begin{tabular}{c|cccc|ccc}
  &$L$&$e_R$&$\mu_R$&$\tau_R$&$\phi_T$&$\phi_S$&$\xi$\\
  \hline
  $A_4$ & 3&1&$1''$&$1'$&3&3&1\\
  $Z_3$ & $\omega$&$\omega^2$&$\omega^2$&$\omega^2$&$1$&$\omega^2$&$\omega^2$\\
  $\FN$ & 0&4&2&0&0&0&0
 \end{tabular}
\caption{Charges of leptons and new scalars under additional flavour groups.}
\label{tab_a4charges}
\end{table}
\begin{align}
 \mathcal{L}_l = y_e e_R(\phi_T l)_{\textbf{1}} + y_\mu \mu (\phi_T l)_{\textbf{1}'} + y_\tau (\phi_Tl)_{\textbf{1}''} + x_a\xi(ll)_{\textbf{1}} + x_b \phi_S (ll)_{\textbf{3}} + \text{h.c.}+...
\end{align}
Operators of higher dimension (...), as well as the vevs $v_{u,d}$ of the supersymmetric Higgs-fields $h_{u,d}$ are omitted for compactness. 
For this leading order expression we already get a diagonal charged lepton mass matrix from which want to note the muonic entry
\begin{align}
 m_\mu = y_\mu v_d \frac{v_T}{\Lambda}
 \label{eq_muonmass}
\end{align}
with a cut-off scale $\Lambda\approx v_T/\epsilon$. The Yukawa coupling $y_\mu$ is evaluated with the FN-mechanism as for quarks. As experiments
stated back then when $A_4$ was initially considered, the neutrino mixing would have a tribimaximal (TBM) pattern 
\begin{align}
 U_\text{TBM} = \begin{pmatrix}
                 \sqrt{2/3}&\sqrt{1/3}&0\\
                 -\sqrt{1/6}&\sqrt{1/3}&-\sqrt{1/2}\\
                 -\sqrt{1/6}&\sqrt{1/3}&+\sqrt{1/2}
                \end{pmatrix}.
\end{align}
With this leading approximation the neutrino mass matrix gets diagonalised $\hat{m}_\nu=2v_u^2/\Lambda^2 \text{diag}(x_au+x_bv_S,x_au,-x_au+x_bv_S)$
and the PMNS matrix can be expressed as \cite{1211.5370}
\begin{align}
 V_\text{PMNS} = U_\text{TBM}\begin{pmatrix}
                              \cos(\varphi) & 0 &\sin(\varphi)\\
                              0&1&0\\
                              -\sin(\varphi) & 0 & \cos(\varphi)
                             \end{pmatrix}
\end{align}
and $\varphi = 0^\circ$ for TBM mixing.
As the non zero value of $\theta_{13}$ is nowadays well established \cite{1303.1289} one has to add higher order operators and new scalars and
also rearrange the vacuum alignment in order to obtain the right pattern \cite{1211.5370}\cite{Tprime}. 

%\textit{General Properties}\\


% finite amount of representations; characters; 
%  \begin{align}
%   \sum m_n n^2 = N_G \quad groupelements\\
%   \sum m_n = \#_{IRR} \quad C = Irred\,Reps\\
%   \varphi:\, G\rightarrow \text{GL}(V)\\
%   g\mapsto \varphi(g):\, V\rightarrow V\\
%   \chi_D(g) = \text{tr}D(g)\\
%   \sum_g \chi_\alpha(g)^*\chi_\beta(g) = N_G \delta_{\alpha\beta}\\
%   \sum_\alpha \chi_\alpha(g)^*\chi_\alpha(h) = \frac{N_G}{n_g} \delta_{C_g C_h} \stackrel{\Lambda}{=} \langle \chi^\mu, \chi^\nu \rangle = \delta^{\mu\nu}\\
%   \text{FS}(R) := \frac{1}{N_G} \sum_g \chi_R(g^2) =\begin{cases}
%                                                      1, & \text{real}\\
%                                                      0, & \text{complex}\\
%                                                      -1, & \text{pseudoreal}\\
%                                                     \end{cases}\\
%   \mu(k) = \langle \chi_R \cdot \chi_{R'} , \chi_{R_k} \rangle \quad tensor product decomposition\\
%  \end{align}
%  \\ \\ \textit{$A_4\times Z_3$}\\
%  tetrahedron; link to $T'$ just because its nice; smallest group with $\boldsymbol{3}$; PMNS matrix; $\theta_{13}$
%  seperation of leptons and neutrinos