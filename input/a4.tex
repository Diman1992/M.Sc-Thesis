
\subsection{$A_4\times Z_3$ and Discrete Groups}
\label{sec_A4GT}
The democratic pattern of the PMNS matrix \eqref{eq_pmns} shows that an addition of just a single $\FN$ flavour symmetry is not viable. Some efforts
were made to explain the lepton mixing with discrete groups but to cover both, the lepton and the quark sector, one has to try quite large groups
\cite{1605.03581}. We will use the smallest group containing a triplet, namely $A_4$ for the leptons. The quarks are trivially charged.
As a remark, its double cover $T'$ exhibits doublet representations which could be used for quark sector phenomenology \cite{Tprime}. 
\\ \\ $A_4$\textit{ properties}\\
\noindent The alternating group of order four can be visualised by a tetrahedron


...
%\textit{General Properties}\\


% finite amount of representations; characters; 
%  \begin{align}
%   \sum m_n n^2 = N_G \quad groupelements\\
%   \sum m_n = \#_{IRR} \quad C = Irred\,Reps\\
%   \varphi:\, G\rightarrow \text{GL}(V)\\
%   g\mapsto \varphi(g):\, V\rightarrow V\\
%   \chi_D(g) = \text{tr}D(g)\\
%   \sum_g \chi_\alpha(g)^*\chi_\beta(g) = N_G \delta_{\alpha\beta}\\
%   \sum_\alpha \chi_\alpha(g)^*\chi_\alpha(h) = \frac{N_G}{n_g} \delta_{C_g C_h} \stackrel{\Lambda}{=} \langle \chi^\mu, \chi^\nu \rangle = \delta^{\mu\nu}\\
%   \text{FS}(R) := \frac{1}{N_G} \sum_g \chi_R(g^2) =\begin{cases}
%                                                      1, & \text{real}\\
%                                                      0, & \text{complex}\\
%                                                      -1, & \text{pseudoreal}\\
%                                                     \end{cases}\\
%   \mu(k) = \langle \chi_R \cdot \chi_{R'} , \chi_{R_k} \rangle \quad tensor product decomposition\\
%  \end{align}
%  \\ \\ \textit{$A_4\times Z_3$}\\
%  tetrahedron; link to $T'$ just because its nice; smallest group with $\boldsymbol{3}$; PMNS matrix; $\theta_{13}$
%  seperation of leptons and neutrinos