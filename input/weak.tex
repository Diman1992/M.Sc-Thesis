\subsubsection{Weak Interaction}
Neutrino oscillation is a process where a neutrino of one generation changes its flavour while propagating through space described by the PMNS matrix. 
This suggests that 
flavour is not a conserved quantum number in nature unlike the electrical charge for example. As this phenomenon is still rather young (2001) and not implemented in the 
SM, we already have flavour violation (FV) therein, in the quark sector. There is a huge list of Mesons and Baryons (composites of one quark and one anti-quark,
or three quarks, respectively) decaying into others with different quark content enabled by the $W$-bosons via charged currents. 
\\ \\ \textit{Flavour Changing Charged Currents}\\
\noindent So it was thought, that the down-type quark mass
eigenstates $d'^i$ are superpositions of their interaction eigenstates $d^i$. Formerly started by Cabibbo and pursued by 
Kobayashi and Maskawa (1973) for CP-violation reasons (charge conjugation C, parity P), the CKM-matrix as an unitary $3\times 3$ matrix was invented,
which can be thought of as a rotation
matrix, rotating the weak eigenstates of the down-type quarks in the mass eigenstates by three Euler angles, $\theta_{12} = \theta_C$ called Cabibbo angle 
as well as $\theta_{23}$ and $\theta_{13}$.
\begin{equation}
 \begin{pmatrix}
  d' \\ s' \\ b'
 \end{pmatrix} = V_\text{CKM}  \begin{pmatrix}
  d \\ s \\ b
 \end{pmatrix},
\end{equation}
\noindent
where $V_\text{CKM}$ can be parameterised by $\lambda = \sin(\theta_C) \approx 0.2$ at leading order (LO) as
\begin{equation}
 V_\text{CKM} \approx \begin{pmatrix}
  1 & \lambda & \lambda^3\\
  -\lambda & 1 & \lambda^2\\
  -\lambda^3 & -\lambda^2 & 1
 \end{pmatrix}.
\end{equation}
The up-type quarks have no differences between their weak and mass eigenstates, but we could have played the same game with them leading to the 
same mixing, since $V_\text{CKM}$ is unitary. Looking at this matrix one can see that it is almost diagonal and hierachical which
means that the farther you leave the main diagonal the smaller become the magnitudes of the entries. It can be asked 
\\ \\ \textit{Flavour Changing Neutral Currents}\\
\noindent TO REVISE!
One problem of Cabibbo's theory is that a process $d'\bar {d'} \rightarrow Z$ would lead to an FCNC at tree level but is in fact highly suppressed. This
could be explained by the GIM-mechanism 


Now comes one of the basic questions motivating this thesis: Are these patterns random or is there rather an underlying broken family-symmetry which
could serve as a blindman's stick for the SM? We go with our guts and prefer the latter suggestion.
