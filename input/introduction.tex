After the prediction (1964) and the experimental verification (2012) of the
now called Higgs-boson, the Standard Model (SM) could be considered self-
contained. The forecasts it makes for processes described therein are proven
remarkably well by various experiments at least at the energy scales they
are currently operating. But there are still some physical phenomena and
properties which are not yet included. The Higgs mechanism is responsible
for the assignment of mass to the SM particles but the reason for their
absolute values is still notional. The fermions thereof can be put into
small groups and are considered to be interacting equally with the gauge bosons but it appears
that some break ranks. Besides the anomalies in the flavour sector there are also unsolved problems
in the field of cosmology. From observation it seems that there has to be a mass distribution of
invisible particles, say dark matter, that lead to the actual orbits of the stars in the galaxy.
There is of course even more, but the resolution of these three issues in one combined theory is
the task of this thesis.

\noindent We start with some notes on 
related topics in the SM and cosmology in section \ref{sec_guide}. With additional groups we want
to approach the right flavour patterns and ensure that our interactions behave appropriately. In \ref{sec_GT} the used symmetry groups and some of 
their properties are introduced. In section \ref{sec_modeloutline} the model
is sketched. Section \ref{sec_pheno} covers the analysis of several processes adressing these issues. 
Section \ref{sec_results} gathers up the results for the dark matter mass of given paramter constraints.
Lastly in section \ref{sec_conclusion} we will conclude. The appendices \ref{sec_appendix} and \ref{sec_appendixA4} present some of
the used loop functions and properties of the flavour group $A_4$, respectively.
% It still lacks for example a quantum field theoretical
% description for gravity or does not contain the experimentally shown masses
% of the neutrinos. Besides, its dynamics depend on 19 parameters, including
% the masses of the fermions, which are quite a lot and are wanted to decrease
% in number by finding relations among one another. Furthermore cosmology
% tells us that the SM is only covering about twenty per cent of the mass content
% of the universe where the rest is made of the nowadays called dark matter
% (DM). Since this thesis is about flavour and dark matter, we
% should have a closer look on the way things stand at the moment.
