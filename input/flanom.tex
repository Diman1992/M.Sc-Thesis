\subsubsection{Flavour Anomalies}
\label{sec_flAnom}
There are several observables in the SM which differ from the experimentally calculated expressions which request contributions from beyond the 
SM. We anticipate such contributions for three processes and therefore show how the experimentally compiled anomalous entities are compared with
the model expressions from section \ref{sec_pheno}.
\\ \\\textit{Anomalous magnetic moment of the muon}\\
\noindent
Every charged particle
may have a spin $s$ with an associated magnetic moment $ \mu = g \frac{e}{2m}s$ \cite{anomMom}.
The factor $g$ is called the gyromagnetic factor and is equal for all elementary particles of a kind. For fermions it is $g=2$. An anomalous
magnetic moment refers to a deviation value, expressed as $a=(g-2)/2$. Taking all deviations within the SM via loop effects into account, there
is still a discrepancy between SM and experiment 
\begin{align}
 \Delta a_\mu = a_\mu^\text{exp} - a_\mu^\text{SM} = 287(80)\cdot 10^{-11}.
\end{align}
Contributions to $a$ can be evaluated in the following way. An ingoing muon with momentum $p_1$ radiates a photon $A_\mu (q)$, leaving the muon in the final 
state with $p_2$. The matrix element for this process can be written as
\begin{align}
 \langle \mu_2|J^\mu(0)|\mu_1\rangle = \bar u_2 \left[F_D(q^2)\gamma^\mu + F_P(q^2)\frac{\ti \sigma^{\mu\nu}q_\nu}{2m}\right] u_1
 \label{eq_gordon}
\end{align}
which is the most general parametrisation one can write due to Ward identity and parity conservation in QED. The electric current is $J^\mu$ 
and $F_D$, $F_P$ are form factors from which the latter is used to calculate the deviation for the magnetic moment. Ideally, the model makes up
for the whole deviation. Hence,
\begin{align}
 \Delta a_\mu = F_P^\text{mod}(0).
\end{align}
\\ \\ \textit{Semileptonic four-fermion operators}\\
When hadrons are involved the methods of effective field theories (EFT) are used \cite{BurasEFT}. At their energy scale of order $\mathcal{O}$(1-10 GeV) processes
are expressed by an effective Hamiltonian which is a series of point-like vertices represented by local operators $O_i$ multiplied by effective 
couplings (Wilson coefficients) $C_i$. For $b\rightarrow s\bar\mu\mu$ transitions it can be written down as \cite{1411.3161}
\begin{align}
 \mathcal{H}^\text{eff} = -\frac{4G_F}{\sqrt{2}} V_{tb} V_{ts}^*\frac{e^2}{16\pi^2}\sum\limits_i (C_i(\mu)O_i(\mu) + C_i'(\mu)O_i'(\mu)) + \text{h.c.}
\end{align}
The crucial property of this is that this operator product expansion (OPE) allows a seperation of scales for calculating the process. At energies 
above a separation scale $\mu$ the perturbative calculation is encoded in the $C_i(\mu)$. Below this scale the matrix elements $\langle O_i(\mu)\rangle$
have to be calculated with non-perturbative methods. From the set of six-dimensional operators we consider NP effects in
\begin{align}
 O_9^{(')} &= (\bar s\gamma_\mu P_{L(R)}b)(\bar l\gamma^\mu l)\\
 O_{10}^{(')} &= (\bar s\gamma_\mu P_{L(R)}b)(\bar l\gamma^\mu\gamma_5 l).
\end{align}
Their Wilson coefficients ($C_i^\text{NP} = C_i - C_i^\text{SM}$) can be obtained by constructing a $\chi^2(\vec C^\text{NP})$ which contains the theory predictions for the observables,
the experimental central values and the corresponding (assumed to be Gaussian) uncertainties \cite{150306199}. For a left chiral model wherein the
NP states only couple to left-handed fermions the Wilson coefficients are thus derived as \cite{1608.07832}
\begin{align}
 C_9^\text{NP} = -C_{10}^\text{NP} \in [-0.81,-0.51]\quad \text{(at 1}\sigma), \label{eq_mumuBound1s}\\
 C_9^\text{NP} = -C_{10}^\text{NP} \in [-0.97,-0.37]\quad \text{(at 2}\sigma).
\end{align}
\\ \\ \textit{Four quark operators}\\
The observable relevant for $B_q$-mixing is $\Delta m_q$. By comparing its experimental value 
$\Delta m_s^\text{exp} = 1.1688(14) \cdot 10^{-11}$ GeV \cite{PDG} and the one from the SM 
$\Delta m_s^\text{SM} = 1.332(213)\cdot 10^{-11}$ GeV \cite{0612167} it appears that NP effect should have a destructive impact. 
With the SM Wilson coefficient $C_{B\bar B}^\text{SM} \simeq 8.2\cdot 10^{-5}$ TeV$^{-2}$ \cite{1608.07832} the determination of the NP contribution 
is estimated with 
\begin{align}
 \frac{\Delta m_s^\text{exp}}{\Delta m_s^\text{SM}}-1 = \frac{C^\text{NP}_{B\bar B}}{C^\text{SM}_{B\bar B}}
\end{align}
to be
\begin{align}
 C^\text{NP}_{BB} \in [-2.0,0.3] \cdot 10^{-5} \text{TeV}^{-2}.%\quad \text{(at 2}\sigma).
 \label{eq_mixBound}
\end{align}





