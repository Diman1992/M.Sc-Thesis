\subsubsection{Flavour Anomalies}
\label{sec_flAnom}
\textit{Anomalous magnetic moment of the muon}\\
\noindent
In general, every charged particle
may have a spin $\boldsymbol{s}$ with an associated magnetic moment \cite{anomMom}
\begin{align}
 \boldsymbol{\mu} = g \frac{e}{2m}\boldsymbol{s}.
\end{align}
The factor $g$ is called the gyromagnetic factor and is equal for all elementary particles of a kind. For fermions it is $g=2$. By saying, the magnetic 
moment is anomalous, one states a deviation from this value which is expressed as $a= (g-2)/2$. There are many contributions to this already within the SM
from QED or hadronic vacuum polarisation which cause most of the uncertainties \cite{160606861} but all of them are quantum loop effects. Even when all of them are taken into account, there is still a discrepancy
between theory and experiment 
\begin{align}
 \Delta a_\mu = a_\mu^\text{ex} - a_\mu^\text{SM} = 287(80)\cdot 10^{-11}.
\end{align}
Contributions to $a$ can be evaluated in the following way. An ingoing muon with momentum $p_1$ radiates a photon $A_\mu (q)$, leaving the muon in the final 
state with $p_2$. To first order in the external field, the scattering amplitude is 
\begin{align}
 M = -\ti e \langle \mu_2|J^\mu(x=0)|\mu_1\rangle A_\mu(q),
\end{align}
with the electic current $J^\mu(x)$. The most general parametrisation one can write thanks to Ward identity and parity conservation in QED is for on-shell
($p_i^2 = m_i^2$) external muons
\begin{align}
 \langle \mu_2|J^\mu(0)|\mu_1\rangle = \bar u_2 \left[F_D(q^2)\gamma^\mu + F_P(q^2)\frac{\ti \sigma^{\mu\nu}q_\nu}{2m}\right] u_1.
 \label{eq_gordon}
\end{align}
In the non-relativistic limit, the relation between the magnetic moment and these form factors $F_D$ and $F_P$ can be derived to be
\begin{align}
 \mu = \frac{e}{2m}\left(F_D(0) + F_P(0)\right)
\end{align}
\noindent
with $F_D(0)=1$ but $F_P(0)\neq0$.
\\ \\ \textit{Semileptonic four-fermion operators}\\
Wilson Coefficient \cite{150306199} \cite{1608.07832}
\begin{align}
 C_9^\text{NP} = -C_{10}^\text{NP} \in [-0.81,-0.51]\quad \text{(at 1}\sigma),\\
 C_9^\text{NP} = -C_{10}^\text{NP} \in [-0.97,-0.37]\quad \text{(at 2}\sigma),
 \label{eq_mumuBound}
\end{align}
\\ \\ \textit{Four quark operators}\\
Bag parameter \cite{1607.00299}
\begin{align}
 B_{B_q}(\mu) = \frac{3}{8f_{B_q}^2 m_B^2} \langle \bar B_q |O_q|B_q \rangle
\end{align}
Mass difference \cite{1102.0009}
\begin{align}
 \Delta m_q \propto f^2_{B_q} \hat{B}_{B_q}
\end{align}
Wilson Coefficient
\begin{align}
 \frac{\Delta m_s^\text{exp}}{\Delta m_s^\text{SM}}-1 = \frac{C_{B\bar B}}{C^\text{SM}_{B\bar B}}
\end{align}
with $\Delta m_s^\text{exp} = 1.1688(14) \cdot 10^{-11}$ GeV (PDG) \cite{1602.03560} and $\Delta m_s^\text{SM} = 1.332(213)\cdot 10^{-11}$ GeV \cite{0612167}%\cite{1102.0009}
and $C_{B\bar B}^\text{SM} \simeq 8.2\cdot 10^{-5}$ TeV$^{-2}$ 
\begin{align}
 C_{BB} \in [-2.0,0.3] \cdot 10^{-5} \text{TeV}^{-2}.%\quad \text{(at 2}\sigma).
 \label{eq_mixBound}
\end{align}
\\ \\ \textit{Majorana}\\
\begin{equation}
\begin{aligned}
 \frac{C^{-1}\left(\slashed{k}+m\right)}{(k^2-m^2}\\
 \frac{\left(\slashed{k}+m\right)C}{k^2-m^2}
\end{aligned}
\label{eq_majoProp}
\end{equation}
\begin{equation}
\begin{aligned}
 w^c = \gamma_0 C w^*
 \bar {w^c}= w^T C^{-1}
\end{aligned}
\label{eq_ChargeConj}
\end{equation}
\cite{Fierz}
\begin{align}
 \left(u_{L,R}\right)^c = v_{R,L}
\end{align}



