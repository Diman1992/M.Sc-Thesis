\subsubsection{Flavour Anomalies}
\label{sec_flAnom}
There are several observables in the SM which differ from the experimentally calculated expressions which request contributions from beyond the 
SM. We anticipate such contributions for three processes and therefore show how the experimentally compiled anomalous entities are compared with
the model expressions from section \ref{sec_pheno}.
\\ \\\textit{Anomalous magnetic moment of the muon}\\
\noindent
Every charged particle
may have a spin $s$ with an associated magnetic moment $ \mu = g \frac{e}{2m}s$ \cite{anomMom}.
The factor $g$ is called the gyromagnetic factor and is equal for all elementary particles of a kind. For fermions it is $g=2$. An anomalous
magnetic moment refers to a deviation value, expressed as $a=(g-2)/2$. Taking all deviations within the SM via loop effects into account, there
is still a discrepancy between SM and experiment 
\begin{align}
 \Delta a_\mu = a_\mu^\text{exp} - a_\mu^\text{SM} = 287(80)\cdot 10^{-11}.
\end{align}
Contributions to $a$ can be evaluated in the following way. An ingoing muon with momentum $p_1$ radiates a photon $A_\mu (q)$, leaving the muon in the final 
state with $p_2$. The matrix element for this process can be written as
\begin{align}
 \langle \mu_2|J^\mu(0)|\mu_1\rangle = \bar u_2 \left[F_D(q^2)\gamma^\mu + F_P(q^2)\frac{\ti \sigma^{\mu\nu}q_\nu}{2m}\right] u_1
 \label{eq_gordon}
\end{align}
which is the most general parametrisation one can write thanks to Ward identity and parity conservation in QED. $J^\mu$ is the electric current
and $F_D$, $F_P$ are form factors from which the latter is used to calculate the deviation for the magnetic moment. Ideally, the model makes up
for the whole deviation. Hence,
\begin{align}
 \Delta a_\mu = F_P^\text{mod}(0).
\end{align}
\\ \\ \textit{Semileptonic four-fermion operators}\\
Wilson Coefficient \cite{150306199} \cite{1608.07832}
\begin{align}
 C_9^\text{NP} = -C_{10}^\text{NP} \in [-0.81,-0.51]\quad \text{(at 1}\sigma),\\
 C_9^\text{NP} = -C_{10}^\text{NP} \in [-0.97,-0.37]\quad \text{(at 2}\sigma),
 \label{eq_mumuBound}
\end{align}
\\ \\ \textit{Four quark operators}\\
Bag parameter \cite{1607.00299}
\begin{align}
 B_{B_q}(\mu) = \frac{3}{8f_{B_q}^2 m_B^2} \langle \bar B_q |O_q|B_q \rangle
\end{align}
Mass difference \cite{1102.0009}
\begin{align}
 \Delta m_q \propto f^2_{B_q} \hat{B}_{B_q}
\end{align}
Wilson Coefficient
\begin{align}
 \frac{\Delta m_s^\text{exp}}{\Delta m_s^\text{SM}}-1 = \frac{C_{B\bar B}}{C^\text{SM}_{B\bar B}}
\end{align}
with $\Delta m_s^\text{exp} = 1.1688(14) \cdot 10^{-11}$ GeV \cite{PDG} \cite{1602.03560} and $\Delta m_s^\text{SM} = 1.332(213)\cdot 10^{-11}$ GeV \cite{0612167}%\cite{1102.0009}
and $C_{B\bar B}^\text{SM} \simeq 8.2\cdot 10^{-5}$ TeV$^{-2}$ 
\begin{align}
 C_{BB} \in [-2.0,0.3] \cdot 10^{-5} \text{TeV}^{-2}.%\quad \text{(at 2}\sigma).
 \label{eq_mixBound}
\end{align}




