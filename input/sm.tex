\subsection{Flavour in the SM}
% The Standard Model is a theory that contains particle fields, ordinary matter is composed of and messenger fields mediating interactions between them. 
% These fields can be separated by their quantum numbers indicating their couplings to each other. At first we have the twelve gauge vector boson fields, 
% eight gluon fields $G_a$ for strong interactions with coloured ($C$) particles and four electroweak boson fields $W_1$, $W_2$, $W_3$ and $B$ which couple
% to particles with weak isospin $T$ and hypercharge $Y_W$ respectively. Now we have 24 fermion fields, six leptons and six quarks each carrying one of three colours.
% The leptons can be subdivided into three electrically ($Q=T+Y_W$) charged and three uncharged ones and the quarks into three up-type ($T=\sfrac12$) and
% three down-type ($T=-\sfrac12$). Last but not least, the already mentioned scalar Higgs-field $\phi$ which develops a non vanishing vacuum expectation value
% (vev) $v$. This is the source for the spontanious breaking of the electroweak symmetry into the electromagnetic 
% (NOT)
The Standard Model is a gauge quantum field theory whose internal symmetry is the unitary product group $SU(3)_C\times SU(2)_L\times U(1)_{Y_W}$ representing
the quantum chromodynamics (QCD) whose charge is called color $C$ and the electroweak theory (GSW-Theory) whose charges are the weak isospin $T$ 
(hold by left handed particles) and the weak hypercharge $Y_W$. These quantum numbers (QN) are carried by a set of particle fields, ordinary matter is composed
of and messenger fields mediating these interactions between them. In particular twelve gauge vector boson fields, eight gluon fields $G_a$ for strong 
interactions and four electroweak boson fields from which $W_1$, $W_2$, $W_3$ couple to the weak isospin and $B$ to the weak hypercharge. Now we have 12
fermion fields, six leptons $e$, $\mu$, $\tau$ as well as their respective neutrinos $\nu$ and six quarks $u$, $d$, $s$, $c$, $b$, and $t$. Actually there
are even more since they are defined by their QN. So for each fermion there is a distinction drawn between left handed $f_L$ ($T=\sfrac12$) and right 
handed $f_R$ ($T=0$), although $\nu_R$ are not considered in the SM. Furthermore there are three different colours for each quark. Well finally the field 
$\phi$ of the already mentioned scalar
Higgs-boson holds a special role in the SM. It is known that the boson fields $W^\pm$ and $Z$, responsible for weak processes, have nonzero masses. But their
mass terms would break the gauge invariance of the lagrangian. So the Higgs-mechanism was considered wherein $\phi$ develops a non vanishing vacuum expectation
value $v$ (vev) which breaks the electroweak symmetry spontaniously down to the electromagnetic symmetry $U(1)_Q$ resulting in a still massless photon field 
$A$ and the three just named massive ones.
\subsubsection{Standard Model Interactions}
\textit{Yukawa Interaction} \\
\noindent Not only the bosons get their masses from this mechanism but the fermions as well - at least the electrically charged ($Q$) ones - which is represented by the
Yukawa (scalar-fermion interaction) term in the SM-lagrangian
\begin{equation}
 \mathcal{L}_{Y} = - y^u_{ij}\, \bar Q^i_L \, \phi^c\, u^j_R - y^d_{ij}\, \bar Q^i_L\, \phi\, d^j_R - y^e_{ij}\, \bar L^i_L\, \phi\, e^j_R + \text{h.c.}.
 \label{eq_yukawaSM}
\end{equation}
\noindent
$y^u$, $y^d$ and $y^e$ are $3\times 3$ (three generations) real, so called yukawa matrices and represent the 
coupling of the fermions to the Higgs. $\phi^c = \ti\sigma_2\phi^*$ is the charged conjugate Higgs. ``h.c.'' stands for hermetian conjugate so that it holds for the antiparticles as well. 
$Q_L$, $L_L$ and $\phi$ are doublets of the $SU(2)_L$, since they are built out of two fermion fields each holding a 
weak isospin of $T=\sfrac12$ with the 3rd component $T_3 = \pm\sfrac12$, e.g. $L^2_L = (\nu_{\mu\, L}, \mu_L)^T_{\textbf{2}}$ with the left handed muon-neutrino and the muon. 
Their counterparts $u_R$, $d_R$ and $e_R$ are singlets under the $SU(2)_L$ because they have no weak isospin, e.g. $e^2_R = (\mu_R)_{\textbf{1}}$, and hence don't take part
in the weak interaction mediated by the $W$-bosons. After 
the symmetry breaking and rotating the fermion fields in a basis where the yukawa matrices become diagonal, we can write down their mass terms 
\begin{equation}
 \mathcal{L}_m = -m^u_i \bar u'^i_L u'^i_R -m^d_i \bar d'^i_L d'^i_R -m^e_i \bar e'^i_L e'^i_R 
 \label{eq_massSM}
\end{equation}
\noindent
where $m^\alpha_i \sim y^\alpha_i \cdot v$ ($\alpha = u,d,e$). Here you can see that the SM does not distinguish between generations. 
It treats a left handed up quark the same way it treats a left handed charm or a left handed top. So if the eigenvalues of $m^\alpha$ would be degenerate, i.e. the masses would
be all the same, one would not have a method to differentiate them. 

% \vspace{-0.3cm}
You could ask
what happened with a term like $\nu_R$ in \eqref{eq_yukawaSM}. This would imply a particle which has neither a color charge, nor electrical charge, nor 
weak isospin and hence ($Q = T_3 + Y_W$) no hypercharge which means that the SM is totally blind to it. Furthermore it seemed obvious not to be
able to construct a neutrino mass since they were considered massless. But to answer the question we could add 
such a term somehow when it can be shown for example that righthanded neutrinos exist, which is likely due to the measurements of neutrino oscillations.
\\ \\ \textit{Weak Interaction}\\
\noindent Neutrino oscillation is a process where a neutrino of one generation changes its flavour while propagating through space described by the PMNS matrix. 
This suggests that 
flavour is not a conserved quantum number in nature unlike the electrical charge for example. As this phenomenon is still rather young (2001) and not implemented in the 
SM, we already have flavour violation (FV) therein, in the quark sector. There is a huge list of Mesons and Baryons (composites of one quark and one anti-quark,
or three quarks, respectively) decaying into others with different quark content enabled by the $W$-bosons via charged currents. 
\\ \\ \textit{Flavour Changing Charged Currents}\\
\noindent So it was thought, that the down-type quark mass
eigenstates $d'^i$ are superpositions of their interaction eigenstates $d^i$. Formerly started by Cabibbo and pursued by 
Kobayashi and Maskawa (1973) for CP-violation reasons (charge conjugation C, parity P), the CKM-matrix as an unitary $3\times 3$ matrix was invented,
which can be thought of as a rotation
matrix, rotating the weak eigenstates of the down-type quarks in the mass eigenstates by three Euler angles, $\theta_{12} = \theta_C$ called Cabibbo angle 
as well as $\theta_{23}$ and $\theta_{13}$.
\begin{equation}
 \begin{pmatrix}
  d' \\ s' \\ b'
 \end{pmatrix} = V_\text{CKM}  \begin{pmatrix}
  d \\ s \\ b
 \end{pmatrix},
\end{equation}
\noindent
where $V_\text{CKM}$ can be parameterised by $\lambda = \sin(\theta_C) \approx 0.2$ at leading order (LO) as
\begin{equation}
 V_\text{CKM} \approx \begin{pmatrix}
  1 & \lambda & \lambda^3\\
  -\lambda & 1 & \lambda^2\\
  -\lambda^3 & -\lambda^2 & 1
 \end{pmatrix}.
 \label{eq_ckm}
\end{equation}
The up-type quarks have no differences between their weak and mass eigenstates, but we could have played the same game with them leading to the 
same mixing, since $V_\text{CKM}$ is unitary. Looking at this matrix one can see that it is almost diagonal and hierachical which
means that the farther you leave the main diagonal the smaller become the magnitudes of the entries. It can be asked 
\\ \\ \textit{Flavour Changing Neutral Currents}\\
\noindent TO REVISE!
One problem of Cabibbo's theory is that a process $d'\bar {d'} \rightarrow Z$ would lead to an FCNC at tree level but is in fact highly suppressed. This
could be explained by the GIM-mechanism 
\begin{align}
 m_u\,:m_c\,m_t\,=\epsilon^8\,:\epsilon^4\,:1\\
 m_d\,:m_s\,m_b\,=\epsilon^4\,:\epsilon^2\,:1
 \label{eq_masshierarchy}
\end{align}
$m_i = y_i v$



Now comes one of the basic questions motivating this thesis: Are these patterns random or is there rather an underlying broken family-symmetry which
could serve as a blindman's stick for the SM? We go with our guts and prefer the latter suggestion.

