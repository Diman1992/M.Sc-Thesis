\begin{appendix}
 \section{Loop Functions}
\label{sec_appendix}
\textit{$B$-processes}\\
\noindent section \ref{sec_bsmumu}. From box diagrams with 4 external and 4 internal propagators of whom two are scalars and two are fermions.
The respective momentum integral reads
\begin{align}
 \int\limits_0^\infty \dx k \frac{k^5}{{D^\chi_k}^2D^l_k D^q_k} = \frac{K(x_q,x_l)}{2 m_\chi^2}
\end{align}
with the loop function \cite{Grip}
\begin{align}
 K(x)&=\frac{1-x+x^2\log(x)}{(x-1)^2} \stackrel{x\rightarrow 1}{=} \frac32, \\
 K(x,y) &= \frac{K(x)-K(y)}{x-y} > 0, \qquad\forall x,y > 0.
 \label{eq_boxloop}
\end{align}
In case of a Majorana fermion, there is also the contribution of crossed boxes wherein the momentum integral is 
\begin{align}
\int\limits_0^\infty \dx k \frac{k^3  m_\chi^2}{{D^\chi_k}^2D^l_k D^q_k} = \frac{G(x_q,x_l)}{2m_\chi^2}
 \label{eq_crossedBox}
\end{align}
which uses the loop function
\begin{align}
 G(x)&=\frac{1-x+x\log(x)}{(x-1)^2} \stackrel{x\rightarrow 1}{=} \frac12,\\
 G(x,y) &= \frac{G(x)-G(y)}{x-y} < 0,\qquad \forall\, x,y>0.
 \label{eq_boxcrossed}
\end{align}
Since there is just one scalar in the boxes for $B_s$-mixing, the needed loopfunction is obtained when $x$ and $y$ in \eqref{eq_boxloop} and 
\eqref{eq_boxcrossed} are the same, i.e. they are their respective derivatives
\begin{align}
 K'(x)&=\frac{-1+x^2-2x\log(x)}{(x-1)^3} \stackrel{x\rightarrow 1}{=} \frac13,\\
 G'(x)&=\frac{-2+2x-\log(x)-x\log(x)}{(x-1)^3} \stackrel{x\rightarrow 1}{=} -\frac16.
 \label{eq_mixloops}
\end{align}
\\ $(g-2)_\mu$\\
\noindent 
The momentum integrals in \eqref{eq_matg2} read as

\begin{align}
\begin{aligned}
 \int\frac{\dx k}{(2\pi)^4}\frac{ k^\rho k_\mu}{D^\chi_kD^l_{k-p_1}D^l_{k-p_2}} =&  f (p_1^\rho p_{2\mu} + p_2^\rho p_{1\mu}) +  d_1p_1^\rho p_{1\mu} +  d_2p_2^\rho p_{2\mu} + \bar x g^\rho_\mu\\
 \int\frac{\dx k}{(2\pi)^4}\frac{k_\mu}{D^\chi_kD^l_{k-p_1}D^l_{k-p_2}} =&  c_1 p_{1\mu} +  c_2 p_{2\mu}.
 \end{aligned}
 \label{eq_g2loops}
\end{align}
In case of negligable masses and an on-shell photon we have $c_1=c_2=:c$ and $d_1=d_2=f/2=:d$ which are \cite{Lavoura}
\begin{align}
 c &= \frac{\ti}{16\pi^2m_F^2} \frac{1-4x+3x^2-2x^2\log(x)}{4(x-1)^3}\stackrel{x\rightarrow 1}{=} -\frac16,\\
 d &= \frac{\ti}{16\pi^2m_F^2} \frac{-2+9x-18x+11x^3 - 6x^3\log(x)}{18(x-1)^4} \stackrel{x\rightarrow 1}{=} -\frac{1}{12}
 \label{eq_loopscd}
\end{align}
with the internal fermion mass $m_F$. When the photon is attached to the fermion we use $I(x) = -\ti 16\pi^2m^2_F(-c+\sfrac32 d)$. 
When its attached to the boson we use $\tilde{I}(x)=1/xI(1/x)$. $I$ reads
\begin{align}
 I(x) = \frac{1}{12(x-1)^4}\left(2+3x-6x^2+x^3+6x\log(x) \right) \stackrel{x\rightarrow1}{=} \frac{1}{24}.
 \label{eq_loopmuon}
\end{align}
\textit{Direct Detection}\\
\noindent In the singlet case \eqref{eq_ldsdLoop}, we have a short $f^s$ and a long $f^l$ distance contribution to the effective gluonic coupling. These two loop functions
are defined as \cite{1007.2601}
\begin{align}
 f^s &= M_q^2(B_0^{(1,4)} + B_1^{(1,4)}),\\
 f^l &= m_q^2(B_0^{(4,1)} + B_0^{(4,1)})
 \label{eq_singletloop}
\end{align}
with
\begin{align}
 B_0^{(n,m)} &= \int \frac{\dx^4 q}{\ti \pi^2}\frac{1}{\left((p+q)^2-m_q^2\right)^n\left(q^2-M^2_q\right)^m},\\
 p_\mu B_1^{(n,m)} &= \int \frac{\dx^4 q}{\ti \pi^2}\frac{q_\mu}{\left((p+q)^2-m_q^2\right)^n\left(q^2-M^2_q\right)^m}.
\end{align}
The calculation yields
\begin{align}
 f^s &=-\frac{(\Delta-6M_q^2m_q^2)(M_q^2+m_q^2-m_\chi^2)}{6\Delta^2M_q^2} - \frac{2M_q^2m_q^4}{\Delta^2}L,\\
 f^l &=-\frac{\Delta+12M_q^2m_q^2}{6\Delta^2} + \frac{M_q^2m_q^2(M_q^2+m_q^2-m_\chi^2}{\Delta^2}L
\end{align}
with
\begin{align}
 \Delta &= m_\chi^4-2m_\chi^2(M_q^2+m_q^2) + (M_q^2-m_q^2)^2,\\
 L&= \begin{cases} \frac{1}{\sqrt{|\Delta|}}\log\frac{M_q^2+m_q^2-m_\chi^2+\sqrt{|\Delta|}}{M_q^2+m_q^2-m_\chi^2-\sqrt{|\Delta|}} & \Delta>0\\ 
		  \frac{2}{\sqrt{|\Delta|}}\text{arctan}\frac{\sqrt{|\Delta|}}{M_q^2+m_q^2-m_\chi^2} &\Delta<0.   
     \end{cases}
\end{align}
They simplify if one expects negligable quark masses $m_q$ and assumes that $M_q \gg m_\chi$
\begin{align}
 f^s \stackrel{m_q=0}{=} - \frac{1}{6M_{\Phi_q}^2\left(M_{\Phi_q}^2-m_\chi^2\right)} \stackrel{M_{\Phi_q}\gg m_\chi}{=} -\frac{1}{6M_{\Phi_q}^4},\\
 f^l \stackrel{m_q=0}{=} - \frac{1}{6\left(M_{\Phi_q}^2-m_\chi^2\right)^2}  \stackrel{M_{\Phi_q}\gg m_\chi}{=} -\frac{1}{6M_{\Phi_q}^4}.
\end{align}

\section{$A_4$ Multiplication Rules}
\label{sec_appendixA4}
\begin{table}[t]
 \begin{tabular}{c|cccc}
 $A_4$ & $\chi^1$ & $\chi^{1'}$ & $\chi^{1''}$ & $\chi^3$ \\
 \hline
 $^1C_1$ & 1 & 1 & 1 & 3\\
 $^3C_2$ & 1 & 1 & 1& -1\\
 $^4C_3$ & 1 & $\omega$& $\omega^2$ & 0\\
 $^4C_4$ & 1 & $\omega^2$& $\omega$ & 0
 \end{tabular}
\caption{Character table for $A_4$. There are three singlet and one triplet irreducible representation. The figures in front of the conjugacy
classes $C_i$ denote their element count $N_i$ and $\omega=\exp(2\pi\ti/3)$.}
\label{tab_charactertable}
\end{table}
The alternating group of order four, the group of even permutations of four elements, 
can be visualised by a tetrahedron with three corners on a basis which can be rotated to each other
along the central axis through the spire by an angle $\omega=\exp(2\pi\ti/3)$ denoted by a rotation matrix $t$. The swapping of a corner on the ground
the spire can be expressed by a rotation along an axis crossing the edge connecting them and the opposite one. This transformation is denoted by 
$s$. These two transformations build the subgroups $G_T$, isomorphic to $Z_3$, and $G_S$, isomorphic to $Z_2$\cite{0512103}. 
The conjugacy classes out of the $N_G=12$ elements can now be constructed as $C_1(e)$, $C_2(s,t^2st,tst^2)$, $C_3(t,st,ts,sts)$ and 
$C_4(t^2,st^2,t^s,tst)=C_3^*$, thus $N_C=4$. With \eqref{eq_multi1} and \eqref{eq_multi2} three 1- and one 3-dimensional representation are derived.
Our interest in a tensor product decomposition is again the trivial singlet. As for continuous groups \eqref{eq_singletadjoint} a trivial singlet 
gets produced if a representation gets multiplied with its conjugate. The trivial singlet and the triplet are real representations and the two
non-trivial singlets are each others conjugates. The full decomposation can be calculated with \eqref{eq_multiplicity} and the characters in table
\ref{tab_charactertable}
\\ \\ \textit{Tensor product decomposations}\\
A triplet is denoted by $(x_1,x_2,x_3)$ and a singlet by $x$, with $x=a$ for the first and $x=b$ for the second representation.
\begin{align}
 \textbf{3}\times\textbf{3} = \textbf{1} +\textbf{1}' +\textbf{1}'' +\textbf{3}_s +\textbf{3}_a
\end{align}
with a symmetric and antisymmetric triplet. Their components are composed as
\begin{align}
 \textbf{1} \sim a_1b_1 +& a_2b_3 + a_3b_2,\qquad  \textbf{1}' \sim a_1b_2 + a_2b_1 + a_3b_3,\qquad  \textbf{1}'' \sim a_1b_3 + a_2b_2 + a_3b_1,\\
  &\textbf{3}_s \sim \frac13\begin{pmatrix}
            2a_1b_1-a_2b_3-a_3b_2\\
            2a_3b_3-a_1b_2-a_2b_1\\
            2a_2b_2-a_3b_1-a_1b_3\\
           \end{pmatrix}, \qquad  \textbf{3}_a\sim\frac12\begin{pmatrix}
				a_2b_3-a_3b_2\\
				a_1b_2-a_2b_1\\
				a_3b_1-a_1b_3\\
				\end{pmatrix}.
\end{align}
A singlet and a triplet always results in a triplet. The components are composed as
\begin{align}
  \textbf{1}\times \textbf{3} &=  \textbf{3} \sim (ab_1,ab_2,ab_3),\\
  \textbf{1}'\times \textbf{3} &=  \textbf{3} \sim (ab_3,ab_1,ab_2),\\
  \textbf{1}''\times \textbf{3} &=  \textbf{3} \sim (ab_2,ab_3,ab_1).
\end{align}
Two singlets always yield the component $ab$ and commute. The decomposed representations are
\begin{align}
 &\textbf{1}\times\textbf{1}=\textbf{1},\qquad \textbf{1}\times\textbf{1}'=\textbf{1}',\qquad \textbf{1}\times\textbf{1}''=\textbf{1}'',\\
 &\textbf{1}'\times\textbf{1}' = \textbf{1}'', \qquad \textbf{1}'\times\textbf{1}'' = \textbf{1},\\
 &\textbf{1}'' \times \textbf{1}'' = \textbf{1}'.
\end{align}





 
\end{appendix}