\begin{appendix}
 \section{Loop Functions}
\label{sec_appendix}
\textit{$B$-processes}\\
\noindent section \ref{sec_bsmumu}. From box diagrams with 4 external and 4 internal propagators of whom two are scalars and two are fermions.
The respective loop function is $K(x,y)$ and reads \cite{Grip}
\begin{align}
 K(x)&=\frac{1-x+x^2\log(x)}{(x-1)^2} \stackrel{x\rightarrow 1}{=} \frac32, \\
 K(x,y) &= \frac{K(x)-K(y)}{x-y} > 0, \qquad\forall x,y > 0.
 \label{eq_boxloop}
\end{align}
In case of a Majorana fermion, there is also the contribution of crossed boxes which uses $G(x,y)$
\begin{align}
 G(x)&=\frac{1-x+x\log(x)}{(x-1)^2} \stackrel{x\rightarrow 1}{=} \frac12,\\
 G(x,y) &= \frac{G(x)-G(y)}{x-y} < 0,\qquad \forall\, x,y>0.
 \label{eq_boxcrossed}
\end{align}
Since there is just one scalar in the boxes for $B_s$-mixing, the needed loopfunction is obtained when $x$ and $y$ in \eqref{eq_boxloop} and 
\eqref{eq_boxcrossed} are the same, i.e. they are their respective derivatives
\begin{align}
 K'(x)&=\frac{-1+x^2-2x\log(x)}{(x-1)^3} \stackrel{x\rightarrow 1}{=} \frac13,\\
 G'(x)&=\frac{-2+2x-\log(x)-x\log(x)}{(x-1)^3} \stackrel{x\rightarrow 1}{=} -\frac16.
 \label{eq_mixloops}
\end{align}
\\ $(g-2)_\mu$\\
\noindent 
The loop function $c$, $d$ and $f$ from \eqref{eq_g2loopscdf} are in case of negligable masses of external particles and $q^2=0$ \cite{Lavoura}
\begin{align}
 c &= \frac{\ti}{16\pi^2m_F^2} \frac{1-4x+3x^2-2x^2\log(x)}{4(x-1)^3}\stackrel{x\rightarrow 1}{=} -\frac16,\\
 d &= \frac{\ti}{16\pi^2m_F^2} \frac{-2+9x-18x+11x^3 - 6x^3\log(x)}{18(x-1)^4} \stackrel{x\rightarrow 1}{=} -\frac{1}{12}
 \label{eq_loopscd}
\end{align}
with the internal fermion mass $m_F$. When the photon is attached to the fermion we use $I(x) = -\ti 16\pi^2m^2_F(-c+\sfrac32 d)$. 
When its attached to the boson we use $\tilde{I}(x)=1/xI(1/x)$. $I$ reads
\begin{align}
 I(x) = \frac{1}{12(x-1)^4}\left(2+3x-6x^2+x^3+6x\log(x) \right) \stackrel{x\rightarrow1}{=} \frac{1}{24}.
 \label{eq_loopmuon}
\end{align}
\textit{Direct Detection}\\
\noindent In the singlet case \eqref{eq_ldsdLoop}, we have a short $f^s$ and a long $f^l$ distance contribution to the effective gluonic coupling. These two loop functions
are defined as \cite{1007.2601}
\begin{align}
 f^s &= M_q^2(B_0^{(1,4)} + B_1^{(1,4)}),\\
 f^l &= m_q^2(B_0^{(4,1)} + B_0^{(4,1)})
 \label{eq_singletloop}
\end{align}
with
\begin{align}
 B_0^{(n,m)} &= \int \frac{\dx^4 q}{\ti \pi^2}\frac{1}{\left((p+q)^2-m_q^2\right)^n\left(q^2-M^2_q\right)^m},\\
 p_\mu B_1^{(n,m)} &= \int \frac{\dx^4 q}{\ti \pi^2}\frac{q_\mu}{\left((p+q)^2-m_q^2\right)^n\left(q^2-M^2_q\right)^m}.
\end{align}
The calculation yields
\begin{align}
 f^s &=-\frac{(\Delta-6M_q^2m_q^2)(M_q^2+m_q^2-m_\chi^2)}{6\Delta^2M_q^2} - \frac{2M_q^2m_q^4}{\Delta^2}L,\\
 f^l &=-\frac{\Delta+12M_q^2m_q^2}{6\Delta^2} + \frac{M_q^2m_q^2(M_q^2+m_q^2-m_\chi^2}{\Delta^2}L
\end{align}
with
\begin{align}
 \Delta &= m_\chi^4-2m_\chi^2(M_q^2+m_q^2) + (M_q^2-m_q^2)^2,\\
 L&= \begin{cases} \frac{1}{\sqrt{|\Delta|}}\log\frac{M_q^2+m_q^2-m_\chi^2+\sqrt{|\Delta|}}{M_q^2+m_q^2-m_\chi^2-\sqrt{|\Delta|}} & \Delta>0\\ 
		  \frac{2}{\sqrt{|\Delta|}}\text{arctan}\frac{\sqrt{|\Delta|}}{M_q^2+m_q^2-m_\chi^2} &\Delta<0.   
     \end{cases}
\end{align}
They simplify if one expects negligable quark masses $m_q$ and assumes that $M_q \gg m_\chi$
\begin{align}
 f^s \stackrel{m_q=0}{=} - \frac{1}{6M_{\Phi_q}^2\left(M_{\Phi_q}^2-m_\chi^2\right)} \stackrel{M_{\Phi_q}\gg m_\chi}{=} -\frac{1}{6M_{\Phi_q}^4},\\
 f^l \stackrel{m_q=0}{=} - \frac{1}{6\left(M_{\Phi_q}^2-m_\chi^2\right)^2}  \stackrel{M_{\Phi_q}\gg m_\chi}{=} -\frac{1}{6M_{\Phi_q}^4}.
\end{align}

 
\end{appendix}