Phenomena on the largest scales in the universe are described by the Big Bang cosmology, parametrised
by the $\Lambda$-CDM model. It \cite{LambdaCDM} gives an explanation for, among other things, the energy density of the 
universe. The model's name refers to the sources adjusting the lack of energy considering only luminous matter. The $\Lambda$ points to the 
cosmological constant in Einstein's field equation
The CDM stands for cold dark matter, speaking particles moving slowly with respect to the speed of light (cold) and interacting very weakly with 
luminous matter (dark). In the following two hints for its existence are presented.
% The reasons for its existence and possible candidates satisfying the ensuing properties will be reviewed as well as the possiblities to test them.\\
\\ \textit{Rotational Curves}\\
\noindent The motion of stars is mainly influenced by their gravitational interactions \cite{LectDMLis}. From Newton's gravitation law, their circular 
velocity is derived as
$v(r)\propto\sqrt{M/r}$ with the radial distance $r$ from the centre of the galaxy and the enclosed mass $M$. For $r$ larger than the galactic 
disc, $M$ should remain constant and the velocity should drop $\propto r^{-1/2}$. But %(RubinFord,RobertsWhitehurst) 
it approximates a constant. Besides modified Newton dynamics \cite{11015122}\cite{160607790} there is strong evidence pointing at
a mass distribution even beyond the galactic
disc with $M(r)\propto r$ consisting of particles, or rather objects, which are not interacting electromagnetically, i.e. they are not visible. The rotation curves suggest a 
spherically symmetric ($\rho_\text{DM}\propto 1/r^2$) halo that also implies that this dark matter does not interact strongly with itself as the visible 
baryonic matter which collapses to the observed disc. To serve for this issue, the existence of DM is coherent but with this observation, its total 
amount in the 
universe cannot be extracted. It should be stressed that the mass distribution does not have to be created by just one particle species. Baryonic dark matter
and the SM neutrinos (may) also contribute. \\
\\ \textit{Cosmic Microwave Background}\\
\noindent The CMB is a major affirmation for the Big Bang theory. It emerged due to an effect called recombination. 
To this point, radiation, electrons and nucleons made up a plasma and the photons were scattering a lot off the actual free electrons. At some point it was 
energetically favoured for the electrons and nucleons to build light elements and the density of free electrons decreased which lead eventually to the decoupling
of the photons, meaning that they could travel freely without interacting. This residual amount of radiation has a wavelength in the microwave scale 
and is very isotropic and follows the spectrum of a black body \cite{DM-EvCaDo}. The observed anisotropics analysed by several CMB experiments 
(WMAP \cite{1212.5226}, ACBAR \cite{0303515}, CBI \cite{0205388}) lead to a discrepancy
between the abundance of baryonic and matter in total. We will now review the computation of the relic abundance and afterwards the detection type
of nucleon scattering.
