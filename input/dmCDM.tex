From the smallest scales, we are heading to the largest in the universe. The effects happening there are discribed by the Big Bang cosmology, parameterised
by the $\Lambda$-CDM model. It was developped in the late 1990's (1512.05356) to give an explanation for, among other things, the energy density of the 
universe. The model's name refers to the sources adjusting the lack of energy. The $\Lambda$ points to the cosmological constant in Einstein's field equation
in the context of general relativity which connects the curvature of space with the stress-energy tensor. If it's non-zero which is considered nowadays(CITE), 
then there is a non-vanishing vacuum energy that is a good candidate for dark energy (CITE), possibly responsible for the acceleration of the universe's expansion.
The CDM stands for cold dark matter, speaking particles moving slowly with respect to the speed of light (cold) and interacting very weakly with luminous
matter (dark). \\
\noindent The reasons for its existence and possible candidates satisfying the ensuing properties will be reviewed as well as the possiblities to test them.
\subsubsection{Evidence and Candidates}
The strongest evidence for DM is its gravitational influence on visible matter. On a galactic scale, in particular the rotational curves of the stars
within a galaxy (Dodelson, KolbTurner). \\
\\ \textit{A: Rotational Curves}\\
\noindent The motion of stars is mainly influenced by their gravitational interactions (1603.0379). From Newton's gravitation law, their circular velocity 
is derived as
$v(r)\propto\sqrt{M/r}$ with the radial distance $r$ from the center and the enclosed mass $M$. For $r$ larger than the the galactic disc, $M$ should remain
constant and the velocity should drop $\propto r^{-1/2}$. But appearently (RubinFord,RobertsWhitehurst) it approximates a constant. One explanation for this
is the lack of generality in Newton's law for very small accelerations which means it could by modified (MoND, 1606.07790, Milgrom), but is at least not able to 
explain the anomalous behaviour all by itself (1101.5122) and needs the help of the second possibility. Namely a mass distribution even beyond the galactic
disc with $M(r)\propto r$ consisting of particles, or rather objects, which are not interacting electromagnetically. The rotation curves suggest a 
spherically symmetric ($\rho_\text{DM}\propto 1/r^2$) halo that also implies that this dark matter does not interact strongly with itself as the visible 
baryonic matter which collapses to the disc. To serve for this issue, the existence of DM is coherent but with this observation,its total amount in the 
universe cannot be extracted\\
\\ \textit{B: Cosmic Micorwave Background}\\
\noindent The CMB is a major proof for the Big Bang theory. It has emerged due to an effect called Recombination at a temperature $T\propto 0.4$ eV (0110414CHECK)
To this point, Radiation and matter was in a thermal equilibrium. 
