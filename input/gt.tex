In anticipation of our model, we will now have a look on group theory. Groups are mathematical entities used to describe symmetries in nature
(\cite{pierre}\cite{georgi} for reviews). Groups are (in)finite if their amount of elements is (in)finite. They are discrete if every open subset
contains only one element, i.e. their elements are not arbitrarily close to each other. A continuous group does not have this property and therefore
contains an infinte amount of elements. If the elements commute, the group is abelian. 
Local group symmetries further depend on the point of spacetime to which they are applied and can be gauged if there are more degrees of freedom
than necessary to completely describe the phyical system and the redundant ones can be set freely. global ones,
whereas, are coordinate-independent and cannot be gauged. 
Some continuous groups also exhibit properties of differential geometry and are called Lie-groups. Even
though the SM gauge group is composed of Lie-groups we focus on the group theoretical aspects here. A vivid example is the rotational transformation
of an object in three dimensions. The continuous, non-abelian group associated with this symmetry is the speical orthogonal group 
$SO(3)$ of dimension 3. Its infinite elements can be represented by $3\times3$ matrices whose inverse is its 
transpose and whose determinant is 1.
% 
% A vivid example is the rotational transformation
% of an object in three dimensions. Such a transformation is induced by a rotation of an angle around an axis. This angle can be infinitely small
% (continuous) and the action of two non-trivial rotations is not equal to that of inversed order (non-abelian). 
% 
% 
% A sphere in three dimensions for example is rotationally symmetric around the origin i.e. a rotational transformation of any angle $\alpha$ would 
% leave the sphere invariant. The composition of two such rotations results in another rotation (closure) and two opposed transformations leave the 
% system unchanged (inverse element). Lengths of and angles between vectors are preserved (orthogonal). The group associated with this symmetry is 
% called $SO(3)$, the special orthogonal group of dimension 3. These infinite elements . 
% \\ \\ \textit{Representations and Kronecker product}\\
\noindent A representation $D$ of a group $G$ is a homomorphism which maps its elements $g\in G$ onto a vector space $V$, spanned by an orthonormal
set of $N$ (dimension of vector space) state vectors. When we speak of the dimension of the representation $d$, it is the dimension of the vector 
space. A fully reducible representation can be decomposed as a direct sum of irreducible ones if $V$ contains invariant subspaces while irreducible
representations on the other hand can be put together into larger ones. For two different irreducible representations $D_\alpha$ and $D_\beta$ 
we have two sets of Hilbert space vectors $|i\rangle_\alpha$ ($i,j=1,2,...,d_\alpha$) and $|s\rangle_\beta$ ($s,t=1,2,...,d_\beta$) with
\begin{align}
 |i\rangle_\alpha \rightarrow |i(g)\rangle_\alpha &= M^{(\alpha)}_{ij}(g)|j\rangle_\alpha\\
 |s\rangle_\beta \rightarrow |s(g)\rangle_\beta &= M^{(\beta)}_{st}(g)|t\rangle_\beta.
\end{align}
The product space is spanned by $|A\rangle := |i\rangle_\alpha |s\rangle_\beta$ and hence 
$|A\rangle \rightarrow M^{(\alpha)}_{ij}(g)M^{(\beta)}_{st}(g)|j\rangle_\alpha|t\rangle_\beta$. The new reducible representation $D_\alpha\times D_\beta$
is called a Kronecker product and can be expressed as a sum of irreducibles of the group which is called the Clebsch-Gordan series
\begin{align}
 D_\alpha \times D_\beta = \sum\limits_\gamma c(\alpha,\beta|\gamma) D_\gamma
 \label{eq_clebschseries}
\end{align}
with to be determined coefficients $c(\alpha,\beta|\gamma)$. After this brief survey on group and representation theory we will now introduce
our flavour symmetries giving rise to the flavour patterns in the SM, starting with the continuous one.


%into a4
% Irreducible
% representations have some interesting properties. Consider two inequivalent, finite dimensional, irreducible representations $D_1$ and $D_2$. 
% Schur's lemma states
% \begin{align}
%  D_1(g)A&=AD_2(g)\,\forall g\in G, \qquad \text{then }A=0,\\
%  D(g)A &= AD(g)\,\forall g\in G, \qquad \,\,\text{then }A\propto E,
% \end{align}
% with the idendity $E$. 

% 
% The action of $g$ in $V$ is assumed to be represented by $N\times N$ invertable matrices $M$ acting
% as 
% \begin{align}
%  |i\rangle \rightarrow |i(g)\rangle = M_{ij}|j\rangle.
%  \label{eq_represent}
% \end{align}
% 
% 
% Irreducible representations have some interesting properties. Consider two vectors $|a\rangle, |i\rangle \in V$ from 
% different representations $D^1$ and $D^2$, respectively, which are linearly related by a $d_\alpha\times d_\beta$ array $S$ as 
% $|a\rangle=S_{ai}|i\rangle$. Now two equations can be set up for $|a\rangle$, relating it to $|i\rangle$ at first and using \eqref{eq_represent}
% afterwards or vice versa. With the completeness $\sum_i  i\rangle\langle i = 1$
% 
% 
% 
% % symmetry breaking,representations
% 
% 
% 
