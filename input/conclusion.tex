The goal of this thesis was to determine whether a simple model with a flavour group and a minimal set of particles added to the Standard Model 
could explain some issues in flavour physics and cosmology. The patterns of fermion mixing and mass hierarchies are achieved with the the group
$\mathcal{G} = \FN\times A_4 \times Z_3$. Flavour anomalies encountered in $B$-meson and muonic processes are targeted by introducing two new complex 
scalar fields $\Phi_{l,q}$
and one new fermion field $\chi$ non trivially charged under the additional groups $U(1)_L\times U(1)_Q\times U(1)_\chi$. They ensure lepton- and baryon number
conservation and demand the NP states to couple to the SM pairwise. Since we demand that the new fermion carries no hypercharge its neutral state is 
assumed to be a Majorana fermion. The muon and third generation quarks obtain rather high couplings to the NP states which is needed to adress 
particularly the anomalous magnetic moment of the muon. Depending on the isospin representations of the new particles, the neutral component of the 
new fermion field obtains constraints on its mass. It is further assumed that the neutral component of $\chi$ is the lightest which makes it in
addition to its odd $U(1)_\chi$ charge stable on cosmological time scales. Therefore it can be a DM candidate which has to make up for the 
relic density in the universe and may not interact too strongly with nucleons due to exclusions bounds from direct detection searches
In case of $\chi$ being a singlet, there are only constraints from $b\rightarrow s\mu\mu$ and the annihilation cross section leaving its mass
in a region of 50-80 GeV. Two triplet cases are investigated. In the first, both scalars are isospin doublets (322), whereas, in the second one,
$\Phi_l$ is a quadruplet (342). The (322)-configuration obtaines an additional lower limit from $B_s$-mixing yielding in a DM mass region of
120-140 GeV. For the (342)-configuration there is also a limit from $\Delta a_\mu$ and , hence, covers all treated processes. The DM mass range
is about 40-50 GeV. While the direct detection cross section for the singlet is far below current bounds, the two triplet cases expose a cross section
which is barely below them and will be tested in the near future.\\
\noindent  Concerning dark matter phenomenology, only leading order
terms with analytical expressions are presented. It might be worth to cross-check the results with numerical calculations provided by 
\texttt{DarkSUSY} \cite{0406204} or \texttt{micrOMEGAs} \cite{1005.4133} among others. 
Besides improving the sensibility of direct detection detectors it is also interesting to follow up new constraints from the flavour 
sector. Rising Wilson coefficients according the amount may eventually exclude the favoured (342)-configuration. It appears that $B_s$-mixing bounds
prefer destructive NP contributions which can be indeed achieved in this setup but only if the new fermion is more massive than the scalars. The 
new fermion may, therefore, not be a viable DM candidate. 

% 
% One thing are the hierarchies of quark masses, quark mixing and neutrino mixing
% which are handled by a continuous $U(1)_\text{FN}$ under which the fermions are charged in such a way that particularly the quark sector gets
% its observed mass and mixing structure. The small discrete group $Z_3$ distinguishes the charged lepton from the neutrino sector so that another
% discrete group $A_4$ can take care of the patterns of neutrino oscillation and also give the muon a special role. This is needed to adress the 
% second issue, the 
% flavour anomalies encountered in $B$-meson processes where a muon is involved and the anomalous magnetic moment of itself $(g-2)_\mu$. These 
% processes are
% mediated by three new particles. At first, one colorless, only weakly charged fermion whose quantum numbers are chosen such that there exists one 
% electrically uncharged state in the isospin multiplet, meaning that it can be a Majorana fermion. Furthermore, this fermion is the lightest of the
% new particles and its stability is protected by an additional $U(1)_\chi$ under which they all are oddly charged while the Standard Model particles 
% are even. The other two particles are scalars, one carrying lepton number, the other baryon number so that no violations of these two quantum numbers
% are induced. With two new particles and one from the Standard Model, they mediate the just mentioned processes with a chiral interaction. 
% The third issue targeted here,
% is the assumed presence of dark matter in the universe which makes up a way bigger part of the mass distribution therein. A lot of hints point
% at a weakly interacting massive particle, just as our new lightest fermion state, to constitute the majority of this obscure matter. \\
% \noindent So the principal task was to examine the contributions to the flavour processes coming from new physics on the one hand and verify
% with the constraints obtained there whether the lightest new fermion state meets the conditions coming from its assumed density in space and
% its interaction with nuclei on earth on the other hand. We did this for two isospin representations of the fermion, either a singlet or a
% triplet while both carry no other Standard Model charge. Another article \cite{Grip}, where this model was invented, suggests a dark matter candidate having
% a high hypercharge. The advantage was that their charge assignment covers all flavour anomalies, especially the anomalous magnetic moment of the muon 
% but unfortunately the dark matter candidate had a too high interaction with nuclei so that it already should have been detected. Our modification 
% also explains the 
% $B$-anomalies but only a fraction of $\Delta a_\mu$. The advantage here, however, is that a Majorana particle as lightest particle is not yet
% excluded in this framework with a mass on the weak scale, leaving it to be a viable dark matter candidate. But even though both representations 
% can fit the $B$-anomalies and the relic density of dark matter, many models can achieve this. The main question is, whether direct detection 
% experiments will eventually rule them out or not. While the nucleon scattering cross section for singlet dark matter is still far below current bounds,
% the triplets will be examined in the near future.\\
