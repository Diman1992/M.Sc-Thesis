\thispagestyle{empty}
\section*{Acknowledgements}
acknowledgements
\newpage
\thispagestyle{empty}
\begin{center}
\section*{Eidesstattliche Versicherung}
\end{center}
\begin{spacing}{1,0}
%\vspace*{1cm}
\noindent
Ich versichere hiermit an Eides statt, dass ich die vorliegende Masterarbeit mit dem Titel \glqq{Majorana Dark Matter and its Role in B-anomalies}\grqq \ selbst\"andig und ohne unzulässige fremde Hilfe erbracht habe. Ich habe keine anderen als die angegebenen
Quellen und Hilfsmittel benutzt sowie w\"ortliche und sinngem\"a\ss e Zitate kenntlich gemacht.
Die Arbeit hat in gleicher oder \"ahnlicher Form noch keiner Pr\"ufungsbeh\"orde vorgelegen.
\vspace*{1cm}
\ \\
\ \\
\line(1,0){150} \hfill \line(1,0){150}\\
Ort, Datum \hfill Unterschrift \hspace*{3cm}
\vspace*{1.5cm}

\subsection*{Belehrung}
Wer vorsätzlich gegen eine die T\"auschung über Prüfungsleistungen betreffende Regelung einer Hochschulprüfungsordnung
verstößt handelt ordnungswidrig. Die Ordnungswidrigkeit kann mit einer Geldbu\ss e von bis zu 50.000,00 \euro{} geahndet werden. Zuständige Verwaltungsbehörde für die Verfolgung und Ahndung von Ordnungswidrigkeiten ist
der Kanzler/die Kanzlerin der Technischen Universit\"at Dortmund. Im Falle eines mehrfachen oder sonstigen schwerwiegenden Täuschungsversuches kann der Prüfling zudem exmatrikuliert werden (\S\ 63 Abs. 5 Hochschulgesetz - HG - ).\\
\ \\
Die Abgabe einer falschen Versicherung an Eides statt wird mit Freiheitsstrafe bis zu 3 Jahren oder mit Geldstrafe bestraft.\\
\ \\
Die Technische Universit\"at Dortmund wird ggf. elektronische Vergleichswerkzeuge (wie z.B. die Software ''turnitin'') zur \"Uberpr\"ufung von Ordnungswidrigkeiten in Pr\"ufungsverfahren nutzen.\\
\ \\
Die oben stehende Belehrung habe ich zur Kenntnis genommen.
\vspace*{1cm}
\ \\
\ \\
\line(1,0){150} \hfill \line(1,0){150}\\
Ort, Datum \hfill Unterschrift \hspace*{3cm}
\vspace*{\fill}
\end{spacing}
