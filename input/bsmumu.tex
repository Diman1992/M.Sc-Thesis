\subsection{$b\rightarrow s\bar\mu\mu$}
\label{sec_bsmumu}
\begin{figure}[t]
 \includegraphics[width=1\textwidth]{../pics/bsmumu.pdf}
 \caption{Feynman diagrams for $b\rightarrow s \bar\mu\mu$ transitions. In case of a Majorana particle being able to mediate such processes, crossed
 diagrams as (b) contribute (see text).}
 \label{pic_Bsmumu}
\end{figure}
Anomalies in decays induced by the quark level transition $b\rightarrow s\bar\mu\mu$ are first of all targeted. The parameter set for this 
semileptonic process includes all our model parameters
\begin{align}
 \left\{m, M_l, M_q, g_2^l\right\}.
\end{align}
Box diagrams of the kind as depicted in figure \ref{pic_Bsmumu}a are by the formalism of effective field theory since the typical hadronic energy 
scale is of $\mathcal{O}$(1 GeV) which is much lower than the weak scale $\mathcal{O}$(100 GeV) which is also the expected mass scale of our NP states. 
\\ \\ \noindent \textit{Matrix Element}\\
\noindent Within the OPE framework (see sec. \ref{sec_flAnom}) where the BSM fields are integrated out 
and the external masses and momenta are set to zero, the amplitude in the full theory is
\begin{align}
 \mathcal{M} =&\,g_i^{q*} g_j^{q} g_m^{l*} g_n^l\int \frac{\dx^4 k}{(2\pi)^4} \frac{k^\rho}{D^\chi_k}\frac{1}{D^q_k} \frac{1}{D^l_k} \frac{k^\sigma}{D^\chi_k} \nonumber\\
 \times&\left[\left(\bar L_L^n \gamma_\rho Q_L^j\right) \left(\bar Q_L^i \gamma_\sigma L_L^m\right) + \xi\left(\bar L_L^n \gamma_\rho\vec\tau Q_L^j\right) \left(\bar Q_L^i \gamma_\sigma \vec\tau L_L^m\right)\right].
 \label{eq_matElemBSmumu}
\end{align}
The denominators are $D_q^a = q^2-M_a^2$ for a particle $a$ with mass $M_a$ and momentum $q$. For these transitions involving four only left-handed
fermions, the two operators 
$O_{lq}^{(1)}:=\left(\bar Q_L^i \gamma_\rho Q_L^j\right)\left(\bar L_L^m \gamma^\rho L_L^n\right)$ and 
$O_{lq}^{(3)}:=\left(\bar Q_L^i\vec \tau \gamma_\rho Q_L^j\right)\left(\bar L_L^m\vec \tau \gamma^\rho L_L^n\right)$ are the only ones being
$SU(2)_L$ invariant \cite{1008.4884}. Their relative Wilson coefficient $\xi$ depends on the NP representations and its calculation is sketched
below. With the commutator relations of the projections
$P_{R,L}=\frac12(1\pm\gamma_5)$ the Lorentz structure is obtained. The numerators can be re-expressed as $k^\rho k^\sigma = k^2g^{\rho\sigma}/4$
under the integral \footnote[1]{This results from spherical coordinates where
every combination $\rho\neq\sigma$ would include at least one uneven angular function. When the spherical integral is performed, they would lead to zero.}.
Performing the non divergent momentum integral after going into spherical coordinates yields a loop function $K(x,y)$ (see \eqref{eq_boxloop}).
In the SM, the fermion currents do not combine quarks and leptons. To compare the NP contribution to this process, we have to rearrange the fermion currents,
so that we have the same structure. This will be done by Fierz identities.
\\ \\ \noindent \textit{Fierz Identities}\\
 A fermion quadrilinear, containing four spinors, is related with a sum of rearranged Dirac quadrilinears \cite{Fierz}
\begin{align}
  \left[\bar w_1\Gamma_J^{} w_2\right] \left[\bar w_3 \Gamma^J w_4 \right] =: e_J(1234) = \sum\limits_K M_{JK} e_K(1432)
\end{align}
with an (anti)spinor $w$ and $\Gamma_J$ as in \eqref{eq_Ctrafo}. The coefficients $M_{JK}$ are real numbers and can differ for other arrangements.
% \begin{align}
%  F_{IJ}=\frac14\begin{pmatrix}
%     1&1&\frac12&-1&1\\
%     4&-2&0&-2&-4\\
%     12&0&-2&0&12\\
%     -4&-2&0&-2&4\\
%     1&-1&\frac12&1&1\\
%  \end{pmatrix}_{IJ}.
% \end{align}
For our purpose the quadrilinear in \eqref{eq_matElemBSmumu} has to be rearranged. This case is represented by $J=V-A$ and it follows that this
Lorentz-structure is an invariant under this transformation so that the quadrilinear can be simply rearranged. For later cases two additional
transformations are pointed out 
\begin{subequations}
\begin{align}
 \left[\bar w_1 \gamma^\mu P_L w_2\right]\left[\bar w_3 \gamma^\mu P_L w_4\right] =& 1  \left[\bar w_3 \gamma_\mu P_L w_2\right]\left[\bar w_1 \gamma^\mu P_L w_4\right]\\
 \left[\bar w_1 P_R w_2\right]\left[\bar w_3 P_L w_4\right] =& \frac12  \left[\bar w_3 \gamma_\mu P_L w_2\right]\left[\bar w^c_4 \gamma^\mu P_L w_1^c\right]\\
 =& \frac12  \left[\bar w_3 \gamma_\mu P_L w_2\right]\left[\bar w_1 \gamma^\mu P_L w_4\right].
 \label{eq_fierzSPtoVA}
\end{align} 
\end{subequations}
\\ \\ \textit{Determining $\xi$}\\
\noindent While $O_{lq}^{(1)}$ only induces neutral currents, $O_{lq}^{(3)}$ enables charged ones as well. One possibility to compute $\xi$ is
to determine the isospin singlet (cf. sec. \ref{sec_FNGT}) in the decomposed tensor product of the respective isospin representations of the 
particles at each vertex. This is to be done for all four vertices per diagram which are multiplied and then added up for every diagram,
one can write down using these operators. Since charged currents cannot be induced by $\chi$ being a singlet, this calculation can be skipped
and $\xi$ has to be zero. Hence,
\begin{align}
  \mathcal{L}^{{122}}_\text{eff} \supset \frac{K(x_q,x_l)}{m_\chi^2}\frac{g_i^{q*} g_j^{q*} g_m^l g_n^l}{64\pi^2}  O_{lq}^{(1)}.
 \label{eq_LagBSmumuModA}
\end{align}
The superscript denotes the isosopin representations for $\chi$, $\Phi_l$ and $\Phi_q$, respectively. In cases of higher dimensional multiplets, 
processes of the aforementioned kind occur and thus we have three distinct types, with respect 
to their prefactor, namely $\bar d d\rightarrow \bar l l$, $\bar d d \rightarrow \bar\nu \nu$ and $d \bar u\rightarrow l\bar\nu$. The expression
for the singlet can be derived with Clebsch-Gordan coefficients for an $SU(2)$ which can be read off the tables from \cite{PDG}.
For a triplet ($\chi$) and two doublets ($\Phi$ and $D=Q,L$) we find
\begin{align}
{\textbf{1}} = \frac{1}{\sqrt{3}}\left(\chi_3\Phi_1D_1 - \frac{1}{\sqrt{2}}\chi_2\left(\Phi_2D_1+\Phi_1 D_2\right) + \chi_1\Phi_2D_2 \right).
 \label{eq_CG}
\end{align}
The indices represent the components of the respective multiplets. With \eqref{eq_CG} and \eqref{eq_modelLagrangian} a generic 
$\bar d d\rightarrow \bar l l$ process which also represents $ b  \rightarrow s\bar \mu \mu$ can be calculated and is examplarily depicted in
figure \ref{pic_CG}.
\begin{figure}[t]
 \includegraphics[width=\textwidth]{../pics/CG.pdf}
 \caption{$SU(2)_L$ singlet representation at each vertex of figure \ref{pic_Bsmumu} with the respective Clebsch-Gordan coefficients from \eqref{eq_CG}. 
 This is exemplary for $\bar d d\rightarrow \bar d d$ processes which get contributions from two different particle configurations.}
 \label{pic_CG}
\end{figure}
For the remaining two types the computation is similar. With these three results, multiplied by a 
phenomenological constant $\eta=\sqrt{2}\,^4 R_\chi$ (norm for a two doublet product at four vertices and $R_\chi$ is the dimension of the $\chi$ multiplet) 
we can extract the $\xi=\sfrac23$. Finally, we can write down the effective Lagrangian for the triplet case
\begin{align}
 \mathcal{L}^{{322}}_\text{eff} \supset \frac{K(x_q,x_l)}{m_\chi^2}\frac{g_i^{q*} g_j^{q*} g_m^l g_n^l}{64\pi^2}\left(O_{lq}^{(1)} + \frac23 O_{lq}^{(3)}\right).
 \label{eq_LagBSmumuModB}
\end{align}
There is further another possible assignment for a triplet DM in which the two scalars have different representations. Similar to the former, we
present the effective Lagrangian for a quadruplet $\Phi_l$ and a doublet $\Phi_q$
\begin{align}
 \mathcal{L}^{{342}}_\text{eff} \supset \frac{K(x_q,x_l)}{m_\chi^2}\frac{g_i^{q*} g_j^{q*} g_m^l g_n^l}{64\pi^2}\left(O_{lq}^{(1)} - \frac13 O_{lq}^{(3)}\right).
 \label{eq_LagBSmumuModC}
\end{align}
\\ \textit{Crossed Boxes}\\
\noindent As we assume our DM candidate $\chi^0$ to be Majorana, there are also contributions from crossed boxes (fig. \ref{pic_Bsmumu}b) for uncharged
currents like our process of interest $\bar d d\rightarrow \bar l l$. Compared to the former matrix element \eqref{eq_matElemBSmumu} there are
some differences for a crossed box diagram \cite{1411.6743}. Using the Feynman-rules for internal Majorana particles 
\eqref{eq_diracprop}-\eqref{eq_majpropoutwards} and the action of $C$ \eqref{eq_DirMaj}, 
the masses in the fermion propagators remain. The momentum integral similarly to the pure Dirac case yields
a similar (but not identical) loop function $G(x,y)$ (see \eqref{eq_boxcrossed}).
Concerning the $SU(2)_L$ invariance, the crossed box can require only the uncharged particle state for the internal lines. 
This results in a contribution which is one fifth of the bare Dirac case \footnote[2]{This term enters according to \cite{1608.07832} with an additional 
factor of 2}. The two terms with $K(x_q,x_l)$ and $G(x_q,x_l)$ have opposite signs so 
that they can cancel each other out and hence may reduce the impact of constraints from observables.\\
\\ \textit{New Physics Contributions}\\
\noindent Now we will gather up the information to get a first bound for our model. There are two operators in the effective Hamiltonian for $b\rightarrow s\mu\mu$
transitions, $O_9$ and $O_{10}$ (sec. \ref{sec_flAnom}). The expression for the Wilson coefficient $C_9 = -C_{10}$ \cite{1408.1627} is obtained by comparing the 
coefficients from the 4-Fermi operator in the respective effective Lagrangian and the effective Hamiltonian
\begin{align}
 C_9^{122} &= N \frac{g_2^{q*}g_3^q|g_2^l|^2}{m_\chi^2} \frac{1}{128\pi^2} \left(K(x_q,x_l) + 2G(x_q,x_l)\right)\label{eq_WilsonBsmumu122}\\
 C_9^{{322}} &= N \frac{g_2^{q*}g_3^q|g_2^l|^2}{m_\chi^2} \frac{5}{384\pi^2} \left(K(x_q,x_l) + 2\cdot\frac15 G(x_q,x_l)\right)\\
 C_9^{{342}} &= N \frac{g_2^{q*}g_3^q|g_2^l|^2}{m_\chi^2} \frac{1}{192\pi^2} \left(K(x_q,x_l) + 2\cdot\frac13 G(x_q,x_l)\right)
 \label{eq_WilsonBsmumu342}
\end{align}
with
\begin{align}
 N = \left(\frac{4G_F}{\sqrt{2}} V_{ts}^*V_{tb} \frac{\alpha_\text{EM}}{4\pi}\right)^{-1}.
\end{align}
With the formulae given here, anomalies in other semileptonic decays could also be examined, i.e. $b \rightarrow s\nu\nu$ \cite{1409.4557} or 
$b \rightarrow c\tau\nu$ \cite{1507.03233}. 
It is difficult to explain the latter at least with this kind of model, where these processes emerge at loop level and have to compete with the 
SM effect, which is at tree-level. The former can give bounds but since you cannot distinguish different neutrino flavours experimentally
and we are only coupling to muonic ones, the contribution is even lower. In the triplet case \eqref{eq_LagBSmumuModB} the prefactor
for the $\bar dd\rightarrow \bar\nu\nu$ type
are also small compared to the other ones enabling an even larger parameter space. So we expect $b\rightarrow s\bar \mu \mu$ 
to give the strongest constraints among the semileptonic processes.






% 
% \begin{align}
%  M =g_i^{q*} g_j^{q} g_m^{l*} g_n^l\int \frac{\dx^4 k}{(2\pi)^4} \frac{k^\rho}{D^\chi_k}\frac{1}{D^q_k} \frac{1}{D^l_k} \frac{k^\sigma}{D^\chi_k} \left(\bar L_L^n \gamma_\rho Q_L^j\right) \left(\bar Q_L^i \gamma_\sigma L_L^m\right)
%  \label{eq_matElemBSmumu}
% \end{align}
% with $k$ as the internal momentum and the denominators of the respective particle $a$ with momentum $q$ $D^a_q = q^2-M_a^2$ and the $\ti\epsilon$ terms in the denominators are already omitted
% due to Wick-rotation later on when the integral is performed. The reasons why $m$ in the fermion propagators is missing
% and why there is a Lorentz structure in the couplings, are connected. Since we have a model which couples only to left handed particles $\psi_L = P_L \psi$,
% the fermion currents look like $\bar \psi_L (\slashed{k}+m)\varphi_L = \bar \psi P_R (k^\mu \gamma_\mu+m) P_L\varphi$. Further
% $\{P_{L,R},\gamma^\mu\} = \gamma^\mu P_{R,L}$, $P_{L,R}^2 = P_{L,R}$ and $P_R P_L = 0$, so we are left over with the four fermion expression in 
% \eqref{eq_matElemBSmumu}. We can also reexpress $k^\rho k^\sigma = k^2 g^{\rho\sigma}/4$ under the integral. This results from spherical coordinates where
% every combination $\rho\neq\sigma$ would include at least one uneven angular function. When the spherical integral is performed, they would lead to zero. The 
% 4 in the denominator is the spacetime dimension. The spherical integral $\dx \Omega$ itself just gives a $2\pi^2$ and we are left with the momentum integral
% which is not problematic since it does not diverge
% \begin{align}
%  M \propto \frac{1}{32\pi^2} \int\limits_0^\infty \dx k \frac{k^5}{{D^\chi_k}^2D^l_k D^q_k} = \frac{K(x_q,x_l)}{64\pi^2 m_\chi^2}
% \end{align}
% with a loop function $K(x,y)$ \eqref{eq_boxloop}.
% In the SM, the fermion currents do not combine quarks and leptons. To compare the NP contribution to this process, we have to rearrange the fermion currents,
% so that we have the same structure. This will be done by the Fierz identities.
% \\ \\ \noindent \textit{Fierz Identities}\\
% They usually relate a Dirac quadrilinear, containing four spinors, with a sum of rearranged Dirac quadrilinears \cite{Fierz}
% \begin{align}
%   \left[\bar w_1\Gamma_J^{} w_2\right] \left[\bar w_3 \Gamma^J w_4 \right] =: e_J(1234) = \sum\limits_K M_{JK} e_K(1432)
% \end{align}
% with an (anti)spinor $w$ and $\Gamma_J$ as a set of Dirac matrices whereby $J=S,V,T,A,P$ representing $I,\gamma^\mu,\sigma^{\mu\nu},\gamma^\mu\gamma_5,\gamma_5$, 
% respectively. The coefficients $M_{JK}$ are real numbers and can differ for other arrangements.
% % \begin{align}
% %  F_{IJ}=\frac14\begin{pmatrix}
% %     1&1&\frac12&-1&1\\
% %     4&-2&0&-2&-4\\
% %     12&0&-2&0&12\\
% %     -4&-2&0&-2&4\\
% %     1&-1&\frac12&1&1\\
% %  \end{pmatrix}_{IJ}.
% % \end{align}
% \noindent For our purpose we want to transform the quadrilinear in \eqref{eq_matElemBSmumu}. By expanding there are four terms
% \begin{align}
%  e_V(1234)+e_A(1234)-\left(e'_V(1234) + e'_A(1234)\right).
%  \label{eq_fierz}
% \end{align}
% The primed quadrilinears represent a pseudoscalar combination of two parity partners, e.g. $e'_V = \left[\bar w\gamma^\mu w\right]\left[w\gamma_\mu\gamma_5w\right]$
% which are transformed with $M^\text{T}$. We can see that \eqref{eq_fierz} is an invariant for every ordering and so we can simply rearrange the quadrilinear.
% \\ \noindent For later cases \eqref{eq_crossedBox} we also point out that 
% \begin{subequations}
% \begin{align}
%  \left[\bar w_1 P_R w_2\right]\left[\bar w_3 P_L w_4\right] =& \frac12  \left[\bar w_3 \gamma_\mu P_L w_2\right]\left[\bar w^c_4 \gamma^\mu P_L w_1^c\right]\\
%  =& \frac12  \left[\bar w_3 \gamma_\mu P_L w_2\right]\left[\bar w_1 \gamma^\mu P_L w_4\right].
%  \label{eq_fierzSPtoVA}
% \end{align} 
% \end{subequations}
%  \noindent \textit{Additional $SU(2)_L$-invariant term}\\
% \noindent The operator $O_\delta=\left(\bar Q_L^i \gamma_\rho Q_L^j\right)\left(\bar L_L^m \gamma^\rho L_L^n\right)$ is not the only one which is a singlet
% in the isospin space but $O_\tau=\left(\bar Q_L^i\vec \tau \gamma_\rho Q_L^j\right)\left(\bar L_L^m\vec \tau \gamma^\rho L_L^n\right)$ as well which enables
% charged currents. The eventual operator is a superposition of these two. One possibility to compute the representation dependent, relative 
% prefactor $\xi$ of $O_\tau$ is to regard the particles at each vertex as isospin multiplets. Since we want the vertices to be 
% $SU(2)_L$-invariant, we 
% decompose the product and focus on the singlet (cf. sec. \ref{sec_FNGT}). This will be done for all four vertices per diagram which are multiplied and then added up for every diagram,
% one can write down using these operators. For the singlet, this calculus can be done equivalently but can be skipped because charged currents as 
% $b \bar u \rightarrow \mu \bar \nu$ are not induced, thus the prefactor of $O_\tau$ has to be zero and therefore 
% \begin{align}
%   \mathcal{L}^{\textbf{1}}_\text{eff} \supset \frac{K(x_q,x_l)}{m_\chi^2}\frac{g_i^{q*} g_j^{q*} g_m^l g_n^l}{64\pi^2} \times O_\delta.
%  \label{eq_LagBSmumuModA}
% \end{align}
% In the case of higher dimensional isospin multiplets, processes of the just mentioned kind occur and thus we have three distinct types, with respect to 
% their prefactor that can be checked by expanding $O_\delta + \xi O_\tau$, namely $\bar d d\rightarrow \bar l l$, $\bar d d \rightarrow \bar\nu \nu$ and 
% $d \bar u\rightarrow l\bar\nu$. \\
% \noindent To get the isospin singlet expression we can make use of the Clebsch-Gordan coefficients, known from spin arithmetics which also obey an $SU(2)$-group.
% They can be read off the tables from \cite{PDG}. For a triplet ($\chi$) and two doublets ($\Phi$ and $D=Q,L$) we get
% \begin{align}
%  \frac{1}{\sqrt{3}}\left(\chi_3\Phi_1D_1 - \frac{1}{\sqrt{2}}\chi_2\left(\Phi_2D_1+\Phi_1 D_2\right) + \chi_1\Phi_2D_2 \right).
%  \label{eq_CG}
% \end{align}
% The indices represent the components of the respective multiplets. One has to be careful which vertex one is looking at while using \eqref{eq_CG} in reference
% to the Lagrangian \eqref{eq_modelLagrangian}. An incoming $u$-quark occupies the first isospin component in the quark doublet $Q$ but an outgoing one the 
% second of $\bar Q$. Keeping this in mind we can look at each vertex of figure \ref{pic_Bsmumu} and examine the applicable coefficient. This is examplarily
% depicted in figure \ref{pic_CG} for a generic $\bar d d\rightarrow \bar l l$ process that also represents $ b  \rightarrow s\bar \mu \mu$.
% \begin{figure}[t]
%  \includegraphics[width=\textwidth]{../pics/CG.pdf}
%  \caption{$SU(2)_L$ singlet representation at each vertex of figure \ref{pic_Bsmumu} with the respective Clebsch-Gordan coefficients from \eqref{eq_CG}. 
%  This is exemplary for $\bar d d\rightarrow \bar d d$ processes which get contributions from two different particle configurations.}
%  \label{pic_CG}
% \end{figure}
% This gives an overall factor of $\sfrac{5}{36}$. The computation for the remaining two types is similar and yields $\sfrac{4}{36}$ for 
% $\bar d d\rightarrow \bar \bar \nu \nu$ and $\sfrac{1}{36}$ for $\bar u d \rightarrow \bar \nu l$. With these three fractions, multiplied by a 
% phenomenological constant $\eta=\sqrt{2}\,^4\cdot R_\chi$ (norm for two doublet product at four vertices and $R_\chi$ is the dimension of the $\chi$ multiplet) 
% and the respective relations from $O_\delta + \xi O_\tau$ for each prcoess type 
% we can extract the prefactor $\xi=\sfrac23$. Finally, we can write down the effective Lagrangian for the triplet case
% \begin{align}
%  \mathcal{L}^{\textbf{3}}_\text{eff} \supset \frac{K(x_q,x_l)}{m_\chi^2}\frac{g_i^{q*} g_j^{q*} g_m^l g_n^l}{64\pi^2}\times\left(O_\delta + \frac23 O_\tau\right).
%  \label{eq_LagBSmumuModB}
% \end{align}
% \\ \textit{Crossed Boxes}\\
% \noindent As we assume our DM candidate $\chi^0$ to be Majorana, there are also contributions from crossed boxes (fig. \ref{pic_Bsmumu}b) for uncharged
% currents like our process of interest $\bar d d\rightarrow \bar l l$. Compared to the former matrix element \eqref{eq_matElemBSmumu} we have some changes
% for the crossed box diagram \cite{1411.6743}. Using the Feynman rules for internal Majorana particles \eqref{eq_diracprop}-\eqref{eq_majpropoutwards} 
% and the action of $C$ \eqref{eq_DirMaj}, 
% we see that now the masses in the fermion propagators remain. The momentum integral similarly to the pure Dirac case yields
% \begin{align}
%  M_\text{crossed} \propto \int\limits_0^\infty \dx k \frac{k^3 \cdot m_\chi^2}{{D^\chi_k}^2D^l_k D^q_k} = \frac{G(x_q,x_l)}{2m_\chi^2}
%  \label{eq_crossedBox}
% \end{align}
% with a similar (but not identical) loop function $G(x,y)$ \eqref{eq_boxcrossed}.
% Concerning the $SU(2)_L$ invariance, the crossed box can require only the uncharged particle state for the internal lines. Therefore one gets only $\sfrac{1}{36}$
% which is one fifth of the bare Dirac contribution. The two terms with $K(x_q,x_l)$ and $G(x_q,x_l)$, respectively, have opposite signs so that they can cancel
% each other out and hence may reduce the impact of constraints from observables.\\
% \\ \textit{New Physics Contribution}\\
% \noindent Now we will gather up the information to get a first bound for our model. There are two operators in the effective Hamiltonian for $b\rightarrow s\mu\mu$
% transitions, $O_9$ and $O_{10}$ (sec. \ref{sec_flAnom}). The expression for the Wilson coefficient $C_9 = -C_{10}$ \cite{1408.1627} is obtained by comparing the 
% coefficients from the 4-fermi operator in the effective Lagrangians and the effective Hamiltonian
% \begin{align}
%  C_9^{\textbf{1}} &= N \frac{g_2^{q*}g_3^q|g_2^l|^2}{m^2} \frac{1}{128\pi^2} \left(K(x_q,x_l) + G(x_q,x_l)\right)\\
%  C_9^{\textbf{3}} &= N \frac{g_2^{q*}g_3^q|g_2^l|^2}{m^2} \frac{5}{384\pi^2} \left(K(x_q,x_l) + \frac15 G(x_q,x_l)\right)
%  \label{eq_WilsonBsmumu}
% \end{align}
% with
% \begin{align}
%  N = \left(\frac{4G_F}{\sqrt{2}} V_{ts}^*V_{tb} \frac{\alpha_\text{EM}}{4\pi}\right)^{-1}.
% \end{align}
% With the formulae given here, one could also examine anomalies in other semileptonic decays, i.e. $b \rightarrow s\nu\nu$ \cite{1409.4557} or $b \rightarrow c\tau\nu$ \cite{1507.03233}. 
% It is difficult to explain the latter at least with this kind of model, where these processes emerge at loop level and have to compete with the 
% SM effect, which is at tree-level. The former can give bounds in general but since you cannot distinguish different neutrino flavours experimentally.
% and we are only coupling to muonic ones, the contribution is even lower. In the triplet case \eqref{eq_LagBSmumuModB} the prefactors just calculated 
% of the kind $\bar dd\rightarrow \bar\nu\nu$
% are also small compared to the other ones enabling an even larger parameter space. So we expect $b\rightarrow s\bar \mu \mu$ among the semileptonic
% processes to give the strongest constraints.
%  
% 
% % Prefactors for other representations are listed in table below.
% 
% % \begin{table}[h]
% % \begin{tabular}{c|cccc}
% %  $(\chi,\Phi)$& (1,2) & (3,2) & (4,3) & (5,4)\\
% %  \hline
% %  $X$ & $0$ & $\sfrac23$ & $\sfrac59$ & $\sfrac12$
% %  \end{tabular}
% %  
% % \end{table}
