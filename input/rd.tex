This non-baryonic, dark matter has a distinct density nowadays. No matter what DM candidate there is, this density should be
yielded. For a single DM particle, a very popular idea to maintain this is that its cosmological abundance is set by thermal production in the 
early universe (see for example \cite{DM-EvCaDo}). In the early universe particles were in thermal equilibrium which means that their annihilation and
creation rate are equal. 
While the temperature $T$ lowers due to the expansion of the universe there is a point $(T_f)$ at which it drops below the particle mass $m$ so that 
DM cannot be further created 
and it becomes harder to find an annihilating partner, suppressed by $\sim \text{e}^{-m/T}$ \cite{LectDMLis}. Its number density $n$ 
remains basically the same, i.e. the particle freezes out, which is mainly dependent on its annihilation cross section.\\
The expansion rate is denoted by the Hubble parameter $H:=\dot{a}/a$ with scale factor of the universe $a$. The phase space distribution is 
governed by the Boltzman equation \cite{DM-EvCaDo} which can be written as 
\begin{align}
 \dot{n} + 3Hn = \langle \sigma v\rangle \left(n_\text{EQ}^2 - n^2\right)
 \label{eq_boltzmannN}
\end{align}
where $n_\text{EQ}\sim (mT)^{3/2}\e^{-m/T}$ is the equilibrium number density for non relativistic particles and $\langle\sigma v\rangle$ denotes 
the thermally averaged cross section times the relative velocity of the two annihilating DM particles which can be expanded for small velocities
\begin{align}
 \langle \sigma v \rangle = a + bv^2 + \mathcal{O}(v^4),
\end{align}
where $a$ and $b$ signify s- or p-wave annihilation, respectively, representing the angular orbital momentum of the initial two particle state,
similar to the atomic orbital model. We introduce the two dimensionless variables $Y=n/s$ (entropy desity $s$)and $x=m/T$ ($x_f\approx 20$) 
and rewrite \eqref{eq_boltzmannN} as
\begin{align}
 \frac{\dx Y}{\dx x} = \frac{s}{Hx}\langle \sigma v\rangle \left(Y_\text{EQ}^2 - Y^2\right)
 \label{eq_boltzmannY}
\end{align}
with d$x$/d$t = H(x) x = H(m)/x$. \eqref{eq_boltzmannY} can be solved numerically or analytically under some assumptions. When the annihilation
cross section does not depend on $x$ and the abundance at freeze out $Y_f$ is higher than today $Y_0$, then $Y_0=\frac{x_fH(m)}{s \langle \sigma v\rangle}$.
Multiplied by the particle mass and the current particle entropy density we compute the particle density $\rho_i=m_in_i$. 
We now express the fraction of this particular $\rho_i$ to the 
critical density $\rho_c \sim h^2$, being the density for a flat universe, as $\Omega_i = \rho_i / \rho_c$ and $h$ is the reduced Hubble constant 
\cite{1308.4150}. Ultimately, we can write down the density of DM today in terms of the annihilation cross section %\cite{1303.5076}\cite{1607.02475}
\begin{align}
 \Omega_\text{DM} h^2 \stackrel{\cite{1303.5076}}{=} 0.11805 \stackrel{\cite{1607.02475}}{\approx} \frac{10^{-26} \text{cm}^3 / \text{s}}{ \langle \sigma v \rangle_\text{Ann}}.
 \label{eq_boundAnn}
\end{align}