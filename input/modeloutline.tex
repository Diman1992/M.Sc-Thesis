\section{Model Outline}
\textit{The Underlying Model}\\
% \noindent  Now that we know the actors on the stage, we can build a model trying to reveal their behaviour altogether. 
\noindent We follow the model constructed
in \cite{Grip}. A renormalisable theory is built which explains the flavour anomalies associated with the $b$ and the $\mu$ at loop level. To achieve this, 
they introduce two new scalar fields $\Phi_l$ and $\Phi_q$ and a single fermion field $\chi$ which couple to SM-fermions via Yukawa interactions. 
The conditions to reduce the amount of possible charge assignments are:
\begin{enumerate}
 \item The conservation of baryon and lepton number and the prevention of additional sources of flavour violation should be ensured.
 \item Scalar couplings involving the Higgs as $\Phi H H$ or $\Phi H H H$, which could modify the observed Higgs phenomenology, should be prevented.
 \item $\Phi_q$ should be a scalar since it has weaker bounds on its mass.
 \item Only the BSM fields sould be charged non-trivially under an additional $U(1)$ symmetry which would lead to loop supression for all NP 
 flavour violating processes and would leave the lightest NP state (LP) stable. This LP should further be uncolored and electrically neutral 
 \item The $SU(2)_L$ representations for the BSM fields should have a dimension less than five.
\end{enumerate}
With these constraints some isospin representations of $\chi$ and $\Phi_l$ are rejected which could lead to unwanted renormalisable interactions, 
but it is stated that the LP would not fit the relic density and/or the direct detection bounds through a large $Z$-coupling, although the anomalies 
are well explained. \\ \\
\noindent \textit{Charge assignments}\\ \noindent
The addition of a flavour group $\mathcal{G} = U(1)_\text{FN}\times A_4 \times Z_3$ \cite{FerA4}\cite{VarzTotMod} prohibits
these unwanted interactions by charging the respective fields. The isospin representations themselves are therefore again enabled. Besides this,
the mass and mixing patterns of the SM fermions and the correct DM phenomeology are obtained. 
The downside herein is the surrender of renormalisability. NP is expected to couple chirally and global fits prefer the 4 fermion operator 
$\bar b_L \gamma^\nu s_L \mu_L \gamma_\nu \mu_L$. $SU(2)_L$ representations of NP yielding isospin invariance are considered and where the LP in
$\chi$ has vanishing hypercharge in order to reduce its coupling to the $Z$. This can be achieved by a singlet and a triplet representation while
the scalars are doublets.
Besides the focus on heavy quarks, anomalies including the muon as its anomalous magnetic moment and the discrepancy in  
$b\rightarrow s\bar\mu\mu$-like processes demand a special treatment in order to explain them.
To achieve this, $\Phi_l$ has to only couple to muons which can be ensured by assigning a suitable $A_4$ representation to it. 
% Anomalous tauonic transitions are also occuring but cannot be explained with this framework.
\begin{table}[t]
 \begin{tabular}{c|c|c|c}
%$SU(3)_C\times SU(2)_L\times U(1)_{Y_W}$
  Field & $\mathcal{G}_\text{SM}$ & $A_4 \times U(1)_\text{FN} \times Z_3$ & $U(1)_{B'}\times U(1)_{L'}\times U(1)_\chi$\\
  \hline
  $Q^i_L$ & (3,2,$\frac16$) & (1,$\Upsilon_{Q_i}$,$\omega$) & ($\frac13$,0,0)\\
  $U^i_R$ & (3,1,$\frac23$) & (1,$\Upsilon_{U_i}$,$\omega^2$)& ($\frac13$,0,0)\\
  $D^i_R$ & (3,1,$-\frac13$) & (1,$\Upsilon_{D_i}$,$\omega^2$)& ($\frac13$,0,0)\\
  $L^i_L$ & (1,2,$-\frac12$) & (3,0,$\omega$)& (0,1,0)\\
  $E^i_R$ & (1,1,$-1$) & ($1 {^(} {'} {^,} '' {^)} $,$\Upsilon_{E_i}$,$\omega^2$)& (0,1,0)\\
  $H$ & (1,2,$\frac12$) & (1,0,1)& (0,0,0)\\
  \hline
  $\chi$ & (1,1,0) & (1,0,$\omega$)& (0,0,1)\\ %bar chi has omega**2 (?) ->Z3invariance
 & (1,3,0) & (1,0,$\omega$)&(0,0,1)\\
  $\Phi_l$ & (1,2,$\frac12$) & ($1'$,0,1)& (0,-1,1)\\
  $\Phi_q$ & (3,2,-$\frac16$) & ($1$,0,1)& ($-\frac13$,0,1)\\
%   \hline
%   $\Phi_T$ & (1,1,0) & ($3$,0,1)& (0,0,0)\\
%   $\theta$ & (1,1,0) & (1,-1,0) & (0,0,0)
 \end{tabular}
\caption{Transformation rules for the SM and BSM fields. $i=1,2,3$ denotes a family index. The two rows for $\chi$ denote a singlet and a triplet, respectively. For the charges under 
$U(1)_\text{FN}$ and the representations of $E_R$ under $A_4$ see \eqref{eq_fnchargesQ} and table \ref{tab_a4charges}.}
\label{tab_models}
\end{table}
The accidental symmetries in the last column in table \ref{tab_models}
enforce stability of the proton and prevent contributions to other baryon and/or lepton number violating
processes. Additionally, the $U(1)_\chi$ stabilises the LP which is a crucial premise for a DM candidate. The $Z_3$ charges of the BSM fields are
chosen such interactions only with the fermionic SM doublets are valid.
When a fermionic multiplet only has $SU(2)_L\times U(1)_Y$ gauge interactions in the SM and its components have an identical tree-level mass, the charged 
ones become heavier than the neutral one due to quantum loop corrections of $\mathcal{O}$(100 MeV) \cite{Hisano}\cite{minMatter}. Chosing 
the hypercharge of $\chi$ to be 0 for both representations the DM particle can be a Majorana fermion. Eventually, 
the interaction Lagrangian for the new particles reads as
\begin{align}
 \mathcal{L} = g_i^q \bar \chi_R Q_L^i \Phi_q + g_i^l \bar \chi_R L_L^i \Phi_l + \text{h.c.}.
 \label{eq_modelLagrangian}
\end{align}
% The kinetic terms of the new particles are not shown here since their structure is not of interest in this thesis. 
$CP$-violating terms are not considered. The couplings $g^{q,l}$ are estimated from \eqref{eq_quarkyukawa} and \eqref{eq_muonmass}.