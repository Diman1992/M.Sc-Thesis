\section{Model Outline}
\textit{The Underlying Model}\\
\noindent Now that we know the actors on the stage, we can build a model trying to reveal their behaviour altogether. We will follow basically the model constructed
in \cite{Grip}. They build a renormalisable theory which generates the anomalies associated with the $b$ and the $\mu$ at loop level. To do so, they 
introduce two new scalar fields $\Phi_l$ and $\Phi_q$ and a single fermion field $\chi$ which couple to SM-fermions via Yukawa interactions. 
Their conditions to reduce the amount of possible charge assignments for them are 
\begin{enumerate}
 \item the preservation of the accidental symmetries as baryon and lepton number and the prevention of other sources of flavour violation.
 \item scalar couplings with the Higgs as $\Phi H H$ or $\Phi H H H$ which could modify the observed Higgs phenomenology should be prevented.
 \item the coloured particle, required by quark interaction, should be a scalar since it will have weaker bounds on its mass.
 \item a $U(1)$ symmetry only interacting non trivially on the BSM fields has the advantage that all NP flavour-violating processes are loop suppressed. And
 moreover the lightest NP state (LP) would be stable that should be uncolored and electrically neutral. 
 \item the dimension of $SU(2)_L$ of the NP irreducibles is less then five.
\end{enumerate}
With these constraints they reject some representations of $\chi$ and $\Phi_l$ which could lead to unwanted renormalisable interactions, but they state 
that their LP would not fit the relic density and/or the direct detection bounds through the $Z$-coupling, although the anomalies are 
well explained by the model. \\ \\
\noindent \textit{Charge assignment}\\ \noindent
The main feature of this thesis is the adding of a flavour group $\mathcal{G} = U(1)_\text{FN}\times A_4 \times Z_3$ \cite{FerA4}\cite{VarzTotMod}. 
It shall prohibit these dangerous interactions by charging the respective fields so that the representations themselves are again enabled. Besides this,
the flavour patterns and the correct DM phenomeology are obtained. 
The downside herein is the surrender of renormalisability. Since we expect that NP couples chirally and that the fits say that the 4 fermion operator 
$\bar b_L \gamma^\nu s_L \mu_L \gamma_\nu \mu_L$ is prefered, we consider $SU(2)_L$ representations of $\chi$ that yield invariance and wherein the lightest
state has a vanishing hypercharge in order to reduce its coupling to the $Z$. This can be achieved in many ways depending on what you want to get 
in the end. Besides the focus on heavy quarks,
our interest is a special role for the muon in order to explain its anomalous magnetic moment and the discrepancy in  
$b\rightarrow s\bar\mu\mu$-like processes. 
To do so, we want $\Phi_l$ to only couple to muons which can be ensured by representing it suitably under $A_4$. Anomalous tauonic transitions are 
also occuring but cannot be explained with this framework.
\begin{table}[t]
 \begin{tabular}{c|c|c|c}
%$SU(3)_C\times SU(2)_L\times U(1)_{Y_W}$
  Field & $\mathcal{G}_\text{SM}$ & $A_4 \times U(1)_\text{FN} \times Z_3$ & $U(1)_{B'}\times U(1)_{L'}\times U(1)_\chi$\\
  \hline
  $Q^i_L$ & (3,2,$\frac16$) & (1,$a^i$,$\omega$) & ($\frac13$,0,0)\\
  $U^i_R$ & (3,1,$\frac23$) & (1,$b^i$,$\omega^2$)& ($\frac13$,0,0)\\
  $D^i_R$ & (3,1,$-\frac13$) & (1,$c^i$,$\omega^2$)& ($\frac13$,0,0)\\
  $L^i_L$ & (1,2,$-\frac12$) & (3,0,$\omega$)& (0,1,0)\\
  $E^i_R$ & (1,1,$-1$) & ($1 {^(} {'} {^,} '' {^)} $,$d^i$,$\omega^2$)& (0,1,0)\\
  $H$ & (1,2,$\frac12$) & (1,0,1)& (0,0,0)\\
  \hline
  $\chi$ & (1,1,0) & (1,0,$\omega$)& (0,0,1)\\ %bar chi has omega**2 (?) ->Z3invariance
 & (1,3,0) & (1,0,$\omega$)&(0,0,1)\\
  $\Phi_l$ & (1,2,-$\frac12$) & ($1''$,0,1)& (0,-1,1)\\
  $\Phi_q$ & (3,2,-$\frac16$) & ($1$,0,1)& ($-\frac13$,0,1)\\
  \hline
  $\Phi_T$ & (1,1,0) & ($3$,0,1)& (0,0,0)\\
 \end{tabular}
\caption{Transformation rules for the SM and BSM fields. The two rows for $\chi$ denote a singlet and a triplet, respectively. For the charges under 
$U(1)_\text{FN}$ and the representations for $E_R$ under $A_4$ see sections \ref{sec_FNGT} and \ref{sec_A4GT}.}
\label{tab_models}
\end{table}
The accidental symmetries in the last column in table \ref{tab_models}
enforce stability of the proton and prevents contributions to other baryon and/or lepton number violating
processes. Additionally, the $U(1)_\chi$ stabilises the LP which is a crucial premise for a DM candidate. 
When a fermionic multiplet only has $SU(2)_L\times U(1)_Y$ gauge interactions in the SM and its components have a common tree-level mass, the charged 
ones become havier than the neutral one due to quantum loop corrections of $\mathcal{O}(100)$ MeV \cite{Hisano}\cite{minMatter}. This implies that 
we can chose the hypercharge of $\chi$ to be 0 for both representations in order to obtain a DM particle being a Majorana fermion. So eventually 
the interaction lagrangian for our new particles reads
\begin{align}
 \mathcal{L} = g_i^q \bar \chi_R Q_L^i \Phi_q + g_i^l \bar \chi_R L_L^i \Phi_l + \text{h.c.}.
 \label{eq_modelLagrangian}
\end{align}
The kinetic terms of the new particles are not shown here since their structure is not of interest in this thesis.