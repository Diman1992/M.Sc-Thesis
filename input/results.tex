We now want to gather up all the just calculated expressions for the anomalous quantities in the flavour sector and for the phenomenological ones 
in the dark sector and compare them to the boundaries we showed in sections \ref{sec_flAnom}, \ref{sec_RD} and \ref{sec_DMDetection}. The first
parameters we will talk about are the scalar masses.
\\ \\ \textit{Sparticle masses}\\
Models like this can be seen as minimal supersymmetric extensions to the SM.
The scalar sparticle masses are set right about the threshold of current limits (PDG)
\begin{align}
 M_l \geq 500 \, \text{GeV}
 M_q \geq 1100 \, \text{GeV}
\end{align}



% supersymmetry
The used parameters for the following plots are
\begin{align}
 |g^l_2| &= 2\\
 |g^q_2| &= 0.04\\
 |g^q_3| &= 1\\
 M_{\Phi_l} &= m_\chi + 200 GeV\\
 M_{\Phi_q} &= m_\chi + 800 GeV\\
 \label{eq_resParams}
\end{align}

\subsection{A - $T(\chi)=\boldsymbol{1}$}
% binolike
\textit{$B_s$ Processes}\\
\noindent \ref{pic_BsResA}
\begin{figure}[t]
 \includegraphics[width=1\textwidth]{../pics/BsModA.pdf}
 \caption{Paramter space plot for $B_s$ mixing and $B_s\rightarrow \mu\mu$.}
 \label{pic_BsResA}
\end{figure}
\\ \textit{Anomalous Magnetic Moment of the Muon}\\
\noindent \ref{pic_g-2A}
\begin{figure}[t]
 \includegraphics[width=1\textwidth]{../pics/g-2A.pdf}
 \caption{Paramter space plot for anomalous magnetic moment of the muon. The area of the blue line crossing the higher and lower bound is valid.}
 \label{pic_g-2A}
\end{figure}

\subsection{B - $T(\chi)=\boldsymbol{3}$}
%winolike
\textit{$B_s$ Processes}\\
\noindent \ref{pic_BsResB}
\begin{figure}[t]
 \includegraphics[width=1\textwidth]{../pics/BsModB.pdf}
 \caption{Paramter space plot for $B_s$ mixing and $B_s\rightarrow \mu\mu$.}
 \label{pic_BsResB}
\end{figure}
\\ \textit{Anomalous Magnetic Moment of the Muon}\\
\noindent \ref{pic_g-2B}
\begin{figure}[t]
 \includegraphics[width=1\textwidth]{../pics/g-2B.pdf}
 \caption{Paramter space plot for anomalous magnetic moment of the muon. The area of the blue line crossing the higher and lower bound is valid.}
 \label{pic_g-2B}
\end{figure}
\\ \textit{Direct Detection} \\
\noindent \ref{pic_ddB}
\begin{figure}[t]
 \includegraphics[width=1\textwidth]{../pics/ddB.pdf}
 \caption{Paramter space plot for nucleon scattering. The cross section for the triplet is barely below current bounds and will be tested next year.}
 \label{pic_ddB}
\end{figure}
\\ \textit{Annihilation}\\
With \eqref{eq_resParams} there are two masses for which $m_\chi$ as a thermal relic makes up for all DM in universe. They are 80 and 350 GeV. In
between the cross section is too high leaving the DM candidate over abundant. Its peak is around 120 GeV.