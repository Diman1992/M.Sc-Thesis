We now want to gather up all the just calculated expressions for the anomalous quantities in the flavour sector and for the phenomenological ones 
in the dark sector and compare them to the boundaries we showed in sections \ref{sec_flAnom}, \ref{sec_RD} and \ref{sec_DMDetection}. The first
parameters we will talk about are the scalar masses.
\\ \\ \textit{Superpartners}\\
Models like this can be seen as minimal supersymmetric (MSSM) or minimal dark matter (MDM) extensions to the SM. Each add a small bunch of new particles to 
the Standard Model and usally charge them under also new global or local symmetry group so that they only interact pairwise with SM particles. 
Supersymmetry is a theory that adds an additional particle for all SM ones (therefore ``superpartner'') with different statistics 
\cite{9709356} e.g. a scalar smuon for the muon, or a fermionic gluino for the gluon. The motivations for such a theory are manifold 
\cite{1302.6587}, such as the gauge hierarchy problem which targets the question of why there is alarge difference between the electroweak breaking 
scale of order $\mathcal{O}(100)$ GeV and the reduced Planck scale of order $\mathcal{O}(10^{18})$ GeV. This further leads to the question why the
higgs mass is what it is and does not suffer from large loop corrections. Another point is grand unification of the three gauge groups in the SM 
which assumes a larger gauge group eventually breaking into them. Within the MSSM the the gauge couplings get modified above the superpartner mass
scale so that they meet at the GUT scale of order $\mathcal{O}(10^{16})$ GeV. \\
\noindent For us the main motivation is that supersymmetry supplies a good DM
candidate, the neutralino which is a superposition of the superpartners of the uncharged electroweak gauge fields $W$ and $B$, the SM-Higgs field and an 
additional Higgs-field, namely two higgsinos, the bino and the neutral wino, respectively. Since they are each their own complex conjugate,
their superpartners have to be as well, meaning that the neutralino is a Majorana fermion. Particles and their superpartners have each the same 
representations under the gauge groups which fits well in our model considering that we have under $SU(2)_L$ either a singlet or a triplet $\chi$
which fit the bino or the wino, respectively, and our scalar messengers $\Phi$ are doublets just as the left handed SM-fermions. But despite these
QN, this model is not like the MSSM. The magnitude of the couplings is defined by the usual SM charges whereas the magnitude of our couplings is 
defined by the charges of our additional flavour group. So the current bounds for the sfermion masses are also not necessarily viable for us.\\
\noindent But nevertheless our scalars would have a similar decay signature as the slepton and the squark, respectively. The mass limits of $\Phi_q$
and $\Phi_l$ are condensed in \cite{Grip} to be $M_q \geq 720$ GeV \cite{1506.08616} and $M_l \geq 300$ GeV \cite{1403.5294}. This hierarchy is
actually favoured since a high $M_q$ lowers contributions to $B_s$-mixing and a low $M_l$ rises its to $g-2$ of the muon. We will not go 
below them in our analysis. 
\\ \\ \textit{$B_s$ anomalies}\\
\noindent We will start with the $B_s$ processes. Even though we stated that the magnitude of the fermionic couplings results from their charges
under the additional flavour group, the muonic one appears to be a bit peculiar. In figure \ref{pic_BsMumuReps} we can see the its dependency on
the DM mass to fit the anomaly from $B_s\rightarrow \mu\mu$ within one standard deviation for different isospin representations of our DM
candidate. 
\begin{figure}[t]
 \includegraphics[width=\textwidth]{../pics/BsmumuReps.pdf}
 \caption{Contour plot of $B_s\rightarrow \mu\mu$ \eqref{eq_WilsonBsmumu} for several $SU(2)_L$ representations of $\chi$. The regions above each 
 line is within the standard deviation \eqref{eq_mumuBound}.}
 \label{pic_BsMumuReps}
\end{figure}
This includes our two treated ones, the singlet and the triplet, as well as the quadruplet from \cite{Grip} and another possible Majorana
containing one, the quintuplet. With this set of lower bound on the scalar masses the muonic coupling for all of them is larger than one. We restrict
ourselves to couplings lower than three, where non-perturbativity would occur. An advantage of the odd representations (sec. \ref{sec_bsmumu}) is
that they give a destructive contribution due to their Majorana type leading to poles above the TeV-scale. Further, the singlet and the triplet
are the extremal ones in the lower mass region. With these two properties one can react to upcoming updates for the Wilson coefficients depending
on their direction so that there is at least no overproduction. It might also be interesting to try different representations for each scalar.
For a new fermion transforming as an isospin triplet, the leptonic scalar could transform as a quadruplet and the colored one as a doublet giving
a negative $\xi$ for $B_s\rightarrow \mu\mu$ loosening the bounds considerably, but leaving $\xi$ for the meson mixing unchanged.\\
\noindent The situation for our four quark operator is a bit different. While the semileptonic needs a NP contribution, the Wilson coefficient 
for $B_s$-mixing \eqref{eq_mixBound} suggests that NP should not be too influential at least not in an incremental way. That is where the Majorana
behaviour of the DM candidate can come in handy due to its destructive additional term. 
\begin{figure}[t]
 \includegraphics[width=\textwidth]{../pics/fa23BBReps.pdf}
 \caption{Parameter space plot for $B_s$-mixing \eqref{eq_WilsonMix} for three isospin representations of $\chi$ which each contain in principal a 
 Majorana field with the upper limit from \eqref{eq_mixBound}. For all three, the impact of the destructive term is shown as the deviation between
 an ``M'' and a ``D''. Regions below each line are allowed.}
 \label{pic_BsMixRepsMajorana}
\end{figure}
In figure \ref{pic_BsMixRepsMajorana} constraints from $B_s$-mixing for different representations as well as the influence of this term are shown.
We obviously do not have to worry about the singlet with our choice for $M_q$ and the quark couplings. In the triplet case, $m_\chi$ has to be at
least above the point of intersection shown there, to match the quark couplings and not to overproduce meson mixing.
\\ \\ \textit{Anomalous magnetic moment of the muon}\\
\noindent the muonic $(g-2)$ is really striking for this kind of model as demonstrated in \cite{Grip}. Their isospin representation for $\chi$ is
4 dimensional and has a maximum of hypercharge so that it still contains an electrically uncharged component. The scalars are triplets and 
are hypercharged appropriately. With this setup the highest electric charges possible with their condiditions are obtained which increases the 
contribution to $\Delta a_\mu$. For a 100 GeV'ish DM particle the muonic coupling would already be of order $\mathcal{O}(2)$ \footnote{Their 
formula differs a bit from ours, concerning the prefactor and the bosonic charge, but the argument stays the same.}.\\
\noindent For our DM candidate to be Majorana, it cannot have hypercharge which means that the overall electric charges are comparably small
after summing over the participating components, viz
\begin{align}
 Q_\chi &= \,0,\,\,\,\,Q_{\Phi_l} = \,1 \qquad \text{singlet}\\
 Q_\chi &= \,1,\,\,\,\,Q_{\Phi_l} = \,1 \qquad \text{triplet}\\
 Q_\chi &= 1.5,\,Q_{\Phi_l} = 1.5\,\,\, \quad \text{quintuplet}.
\end{align}
leaving neither the singlet nor the triplet to account for the anomalous magnetic moment of the muon with a perturbative coupling.
\\ \\ \textit{DM phenomenology}\\
\noindent bla



\subsection{A - $T(\chi)=\boldsymbol{1}$}
% binolike
\textit{$B_s$ Processes}\\
\noindent \ref{pic_BsResA}
\begin{figure}[t]
 \includegraphics[width=1\textwidth]{../pics/BsModA.pdf}
 \caption{Paramter space plot for $B_s$ mixing and $B_s\rightarrow \mu\mu$.}
 \label{pic_BsResA}
\end{figure}
\\ \textit{Anomalous Magnetic Moment of the Muon}\\
\noindent \ref{pic_g-2A}
\begin{figure}[t]
 \includegraphics[width=1\textwidth]{../pics/g-2A.pdf}
 \caption{Paramter space plot for anomalous magnetic moment of the muon. The area of the blue line crossing the higher and lower bound is valid.}
 \label{pic_g-2A}
\end{figure}

\subsection{B - $T(\chi)=\boldsymbol{3}$}
%winolike
\textit{$B_s$ Processes}\\
\noindent \ref{pic_BsResB}
\begin{figure}[t]
 \includegraphics[width=1\textwidth]{../pics/BsModB.pdf}
 \caption{Paramter space plot for $B_s$ mixing and $B_s\rightarrow \mu\mu$.}
 \label{pic_BsResB}
\end{figure}
\\ \textit{Anomalous Magnetic Moment of the Muon}\\
\noindent \ref{pic_g-2B}
\begin{figure}[t]
 \includegraphics[width=1\textwidth]{../pics/g-2B.pdf}
 \caption{Paramter space plot for anomalous magnetic moment of the muon. The area of the blue line crossing the higher and lower bound is valid.}
 \label{pic_g-2B}
\end{figure}
\\ \textit{Direct Detection} \\
\noindent \ref{pic_ddB}
\begin{figure}[t]
 \includegraphics[width=1\textwidth]{../pics/ddB.pdf}
 \caption{Paramter space plot for nucleon scattering. The cross section for the triplet is barely below current bounds and will be tested next year.}
 \label{pic_ddB}
\end{figure}
\\ \textit{Annihilation}\\
With \eqref{eq_resParams} there are two masses for which $m_\chi$ as a thermal relic makes up for all DM in universe. They are 80 and 350 GeV. In
between the cross section is too high leaving the DM candidate over abundant. Its peak is around 120 GeV.