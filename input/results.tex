We now want to gather up all the just calculated expressions for the anomalous quantities in the flavour sector and for the phenomenological ones 
in the dark sector and compare them to the boundaries we showed in sections \ref{sec_flAnom}, \ref{sec_RD} and \ref{sec_DMDetection}. The first
parameters we will talk about are the scalar masses.
\\ \\ \textit{Superpartner masses}\\
Models like this can be seen as minimal supersymmetric (MSSM) or minimal dark matter (MDM) extensions to the SM. Each add a small bunch of new particles to 
the Standard Model and usally charge them under also new global or local symmetry group so that they only interact pairwise with SM particles. 
Supersymmetry is a theory that adds an additional particle for all SM ones (therefore ``superpartner'') with different statistics 
\cite{9709356} e.g. a scalar smuon for the muon, or a fermionic gluino for the gluon. The motivations for such a theory are manifold 
\cite{1302.6587}, such as the gauge hierarchy problem which targets the question of why there is alarge difference between the electroweak breaking 
scale of order $\mathcal{O}(100)$ GeV and the reduced Planck scale of order $\mathcal{O}(10^{18})$ GeV. This further leads to the question why the
higgs mass is what it is and does not suffer from large loop corrections. Another point is grand unification of the three gauge groups in the SM 
which assumes a larger gauge group eventually breaking into them. Within the MSSM the the gauge couplings get modified above the superpartner mass
scale so that they meet at the GUT scale of order $\mathcal{O}(10^{16})$ GeV. \\
\noindent For us the main motivation is that supersymmetry supplies a good DM
candidate, the neutralino which is a superposition of the superpartners of the uncharged electroweak gauge fields $W$ and $B$, the SM-Higgs field and an 
additional Higgs-field, namely two higgsinos, the bino and the neutral wino, respectively. Since they are each their own complex conjugate,
their superpartners have to be as well, meaning that the neutralino is a Majorana fermion. Particles and their superpartners have each the same 
representations under the gauge groups which fits well in our model considering that we have under $SU(2)_L$ either a singlet or a triplet $\chi$
which fit the bino or the wino, respectively, and our scalar messengers $\Phi$ are doublets just as the left handed SM-fermions. But despite these
QN, this model is not like the MSSM. The magnitude of the couplings is defined by the usual SM charges whereas the magnitude of our couplings is 
defined by the charges of our additional flavour group. So the current bounds for the sfermion masses are also not necessarily viable for us.\\
\noindent But nevertheless our scalars would have a similar decay signature as the slepton and the squark, respectively. The mass limits of $\Phi_q$
and $\Phi_l$ are condensed in \cite{Grip} to be $M_q \geq 720$ GeV \cite{1506.08616} and $M_l \geq 300$ GeV \cite{1403.5294}. Within our framework,
there would also be a possible arrangement for the muon coupling and $m_\chi$ but either the contribution to $\Delta a_\mu$ would be to small or
the relic density would be too low
to be conservative,
we use the lower bounds from (PDG) which are set as
\begin{align}
 M_l \geq 500 \, \text{GeV}
 M_q \geq 1100 \, \text{GeV}
\end{align}



% supersymmetry
The used parameters for the following plots are
\begin{align}
 |g^l_2| &= 2\\
 |g^q_2| &= 0.04\\
 |g^q_3| &= 1\\
 M_{\Phi_l} &= m_\chi + 200 GeV\\
 M_{\Phi_q} &= m_\chi + 800 GeV\\
 \label{eq_resParams}
\end{align}

\subsection{A - $T(\chi)=\boldsymbol{1}$}
% binolike
\textit{$B_s$ Processes}\\
\noindent \ref{pic_BsResA}
\begin{figure}[t]
 \includegraphics[width=1\textwidth]{../pics/BsModA.pdf}
 \caption{Paramter space plot for $B_s$ mixing and $B_s\rightarrow \mu\mu$.}
 \label{pic_BsResA}
\end{figure}
\\ \textit{Anomalous Magnetic Moment of the Muon}\\
\noindent \ref{pic_g-2A}
\begin{figure}[t]
 \includegraphics[width=1\textwidth]{../pics/g-2A.pdf}
 \caption{Paramter space plot for anomalous magnetic moment of the muon. The area of the blue line crossing the higher and lower bound is valid.}
 \label{pic_g-2A}
\end{figure}

\subsection{B - $T(\chi)=\boldsymbol{3}$}
%winolike
\textit{$B_s$ Processes}\\
\noindent \ref{pic_BsResB}
\begin{figure}[t]
 \includegraphics[width=1\textwidth]{../pics/BsModB.pdf}
 \caption{Paramter space plot for $B_s$ mixing and $B_s\rightarrow \mu\mu$.}
 \label{pic_BsResB}
\end{figure}
\\ \textit{Anomalous Magnetic Moment of the Muon}\\
\noindent \ref{pic_g-2B}
\begin{figure}[t]
 \includegraphics[width=1\textwidth]{../pics/g-2B.pdf}
 \caption{Paramter space plot for anomalous magnetic moment of the muon. The area of the blue line crossing the higher and lower bound is valid.}
 \label{pic_g-2B}
\end{figure}
\\ \textit{Direct Detection} \\
\noindent \ref{pic_ddB}
\begin{figure}[t]
 \includegraphics[width=1\textwidth]{../pics/ddB.pdf}
 \caption{Paramter space plot for nucleon scattering. The cross section for the triplet is barely below current bounds and will be tested next year.}
 \label{pic_ddB}
\end{figure}
\\ \textit{Annihilation}\\
With \eqref{eq_resParams} there are two masses for which $m_\chi$ as a thermal relic makes up for all DM in universe. They are 80 and 350 GeV. In
between the cross section is too high leaving the DM candidate over abundant. Its peak is around 120 GeV.