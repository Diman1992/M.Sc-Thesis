All the just calculated expressions for the anomalous quantities in the flavour sector and for the phenomenological ones 
in the dark sector are collected and compared to the boundaries shown in sections \ref{sec_flAnom}, \ref{sec_RD} and \ref{sec_DMDetection}. The first
parameters focussed on are the scalar masses.
\\ \\ \textit{Superpartner masses}\\
Models like this can be seen as minimal supersymmetric (MSSM) or minimal dark matter (MDM) extensions to the SM. Each add a small bunch of new particles to 
the Standard Model and usally charge them under an also new global or local symmetry group so that they only interact pairwise with SM particles. 
Supersymmetry is a theory that adds a particle for all SM ones (therefore ``superpartner'') with different statistics 
\cite{9709356} e.g. scalar sfermions and fermionic bosinos. The motivations for such a theory are manifold 
\cite{1302.6587}, such as the gauge hierarchy problem or grand unification of the three gauge groups in the SM. 
For us the main motivation is that supersymmetry supplies a good DM
candidate, the neutralino which is a superposition of the superpartners of the uncharged electroweak gauge fields $W$ and $B$, the SM-Higgs field and 
an additional Higgs-field, namely two higgsinos, the bino and the neutral wino, respectively. Since they are each their own complex conjugate,
their superpartners have to be as well, meaning that the neutralino is a Majorana fermion. Particles and their superpartners have each the same 
representations under the gauge groups which fits well in our model considering that $\chi$ is either a singlet or a triplet under $SU(2)_L$ 
which fit the bino or the wino, respectively. Furthermore, the scalar messengers $\Phi_{l,q}$ are doublets like the left-handed SM-fermions. 
But despite these
QN, this model is not like the MSSM. The magnitude of the supersymmetry (SUSY) couplings is defined by the usual SM charges whereas the magnitude of 
the couplings of this model is 
defined by the charges of the additional flavour group. For the sfermion masses the current bounds are therefore not necessarily viable for us.\\
\noindent Nevertheless the scalars would have a similar decay signature as the slepton and the squark. The mass limits of $\Phi_q$
and $\Phi_l$ are condensed in \cite{Grip} to be $M_q \gtrsim 720$ GeV \cite{1506.08616} and $M_l \gtrsim 300$ GeV \cite{1403.5294}. This hierarchy is
actually favoured since a high $M_q$ lowers contributions to $B_s$-mixing and a low $M_l$ rises its to $g-2$ of the muon. We will not go 
below them in our analysis. 
\\ \\ \textit{$B_s$ anomalies}\\
\noindent We will start with the $B_s$ processes. Even though it was stated that the magnitude of the fermionic couplings results from their charges
under the additional flavour group, the muonic one appears to be peculiar. In figure \ref{pic_BsMumuReps} its dependency on
the DM mass is shown to fit the anomaly from $b\rightarrow s\bar\mu\mu$ within one standard deviation for different isospin representations of our DM
candidate. 
\begin{figure}[t]
 \includegraphics[width=\textwidth]{../pics/BsmumuReps.pdf}
 \caption{Contour plot for bounds of $b\rightarrow s\bar\mu\mu$  \eqref{eq_WilsonBsmumu122}-\eqref{eq_WilsonBsmumu342} for  $SU(2)_L$ 
 representations of $\chi$ used here and in \cite{Grip}, respectively.  The lines denote the upper boundaries one standard deviation 
 \eqref{eq_mumuBound1s}. The lower ones lying each above are not shown for lucidity reasons.}
 \label{pic_BsMumuReps}
\end{figure}
This includes our two treated ones, the singlet and the two triplet cases, as well as the quadruplet from \cite{Grip}. 
With this set of lower bounds on the scalar masses the muonic coupling for all of them is larger than one. We restrict
ourselves to couplings not larger than three, where non-perturbativity would occur. An advantage of the odd representations is
that they give a destructive contribution due to their Majorana type leading to poles above the TeV-scale. Further, the two triplet cases
are the extremal ones in the lower mass region. Hence, they are the most likely ones to adjust depending on the direction upcoming updates are 
going to lead the Wilson coefficients.\\
\noindent While the semileptonic process needs a NP contribution, the Wilson coefficient 
for $B_s$-mixing \eqref{eq_mixBound} suggests that NP should not be too influential at least not in an incremental way. That is where the Majorana
behaviour of the DM candidate can come in handy due to its destructive additional term. 
\begin{figure}[t]
 \includegraphics[width=\textwidth]{../pics/fa23BBReps.pdf}
 \caption{Parameter space plot for upper bounds \eqref{eq_mixBound} of $B_s$-mixing \eqref{eq_WilsonMix1},\eqref{eq_WilsonMix3} for the singlet and 
 the triplet $\chi$, respectively. The impact of crossed boxes is shown as a superscript $M$ signifies its inclusion, whereas, its absence is
 signified by a $D$ superscript. Regions below each line are allowed.}
 \label{pic_BsMixRepsMajorana}
\end{figure}
In figure \ref{pic_BsMixRepsMajorana} constraints from $B_s$-mixing for different representations as well as the influence of this term are shown.
With our choice for $M_q$ and the quark couplings, there is no constraint for the singlet. In the triplet case, $m_\chi$ has to be at
least above the point of intersection shown there, to match the quark couplings and not to overproduce meson mixing.
\\ \\ \textit{Anomalous magnetic moment of the muon}\\
\noindent the muonic $(g-2)$ is really striking for this kind of model as demonstrated in \cite{Grip}. Their isospin representation for $\chi$ is
4 dimensional and has a maximum of hypercharge so that it still contains an electrically uncharged component. The scalars are triplets and 
are hypercharged appropriately. With this setup the highest electric charges possible with their conditions are obtained which increases the 
contribution to $\Delta a_\mu$. For a 200 GeV DM particle the muonic coupling would already be of order $\mathcal{O}(2)$ \footnote[2]{The 
formula in \cite{Grip} differs a bit from ours: Their numerical prefactor is $1/6\pi^2$, whereas ours is $1/16\pi^2$ taken from \cite{Lavoura}.
The bosonic charge is 2, whereas it should be 3, considering the model configuration.}.\\
\noindent For a DM candidate to be Majorana, it cannot have hypercharge which means that the overall electric charges are comparably small
after summing over the participating components, viz
\begin{align}
 Q_\chi &= 0,\,Q_{\Phi_l} = 1 \qquad \text{(122)},\\
 Q_\chi &= 1,\,Q_{\Phi_l} = 1 \qquad \text{(322)},\\
 Q_\chi &= 0,\,Q_{\Phi_l} = 3 \qquad \text{(342)}.
\end{align}
leaving neither of the first two to account for the anomalous magnetic moment of the muon with a perturbative coupling. The three numbers denote
the isospin representations of $\chi$, $\Phi_l$ and $\Phi_q$, respectively. By demanding a Majorana DM candidate, its contribution to 
$\Delta a_\mu$ can only lie up to 20$\,\%$ (122) or 50$\,\%$ (322) for a reasonable 
mass scale. But the third configuration is viable.
\\ \\ \textit{Direct detection and annihilation}\\
\noindent One motivation for this setup is to gain non exclusion through direct detection. Since the fermion preferably interacts with third 
generation
quarks whose content in nuclei are low, the DM mostly scatters off gluons via heavy (s)quark loops giving the leading contribution. Unfortunately,
as already stated, the nucleon scattering cross section for our singlet suffers from the large coloured scalar mass levelling it off at order 
$\mathcal{O}(10^{-49}$ cm$^2$) which is far below current bounds. This leaves it viable as a DM candidate on the one side, but it consequently cannot
be detected this way in the near future. For the triplet, by contrast, there are also box diagrams with $W$-bosons and Higgs penguins. The 
contribution
from tree level scattering with light quarks is subleading and for both there is no tree level $Z$-echange which is predominantly responsible
for the direct detection exclusions for Dirac DM for wide mass ranges. But most importantly is that the cross section does not suffer from the 
heavy scalar mass which gives a larger cross section despite the corresponding processes are induced at two-loop level. The $\sigma_\text{SI}$
runs barely below the observed limit of current constraints from \cite{1607.02475} (see fig. \ref{pic_ddbounds}) but already within the $1-\sigma$
range of background-only trials. Hence, in the near future, \cite{1512.07501} suggests, the triplet can be examined with a minimum cross section
of order $\mathcal{O}(10^{-47}$ cm$^2$) for $m_\chi$ at 50 GeV and no model parameter could be tuned to avoid this.\\
% 
% is the subleading contribution from tree level scattering with light
% quarks and no contribution from tree level $Z$-exchange which is predominantly responsible for the direct detection exclusions for Dirac DM. 
% 
% \noindent But despite the fact that a fermion with no hypercharge cannot achieve to explain the whole amount of the anomaly, it is a preferred 
% 
% 
% its interactions with 
% nucleons are loop-suppressed leaving it to be not excluded by current measurements. 
% 
% 
% The major problem of a Dirac DM candidate is the high direct 
% detection rate due to $Z$ exchange \cite{1503.03382} excluding wide mass ranges. Unfortunately, as already stated, our singlet DM has a direct 
% detection rate far below current bounds which leaves it viable as a DM candidate on the one side, but consequently cannot be detected this way in
% the near future. The triplet, however, is also below current bounds but will be in reach in the near future as \cite{1512.07501} suggests which
% states a minimum cross section of order $\mathcal{O}(10^{-47})$ cm$^2$ for $m_\chi$ at 50 GeV. \\
\noindent Unlike direct detection, giving only upper bounds on the DM mass, the annihilation processes from section \ref{sec_annihilation} should
make up the required relic density \eqref{eq_boundAnn} if the DM candidate is assumed to be the only stable dark particle. The DM mass can therefore 
be fixed for a given parameter set.
%  \begin{figure}[t]
%   \includegraphics[width = \textwidth]{../pics/annTriplet.pdf}
%   \caption{Annihilation cross sections of the three most contributing channels for an exemplary muonic coupling for the triplet, $g_2^l=1.8$. 
%   Assuming the DM candidate is a thermal relic, the intersection point of the blue and purple line signify its applicable mass.}
%   \label{pic_annTriplet} %problematic with WW estimation
%  \end{figure}
For DM masses being indeed smaller than but almost equal to a mediator mass,
co-annihilation would occur ($m_\Phi = 1.2 m_\chi$), boosting the annihilation rate even further. For many models where the couplings are below unity
co-annihilation, among others, is a needed feature to obtain the right relic density. Mostly due to our muonic decay channel, the appropriate mass is 
below this region where this effect would enter. 
\\ \\ \textit{Muonic coupling and DM mass}\\
\noindent The influence of the calculated processes on the remaining two parameters $g_2^l$ and $m_\chi$ is examined and it is checked
which findings hold true for the most. \\
\noindent As said, only the 342-configuration can give an explanation for $\Delta a_\mu$ and the other two have no exclusions on mass areas for any 
region. Due to its strong destructive interference between the (un)crossed boxes mediating the $B_s$-mixing and our choice for $M_q$, this process 
also cannot give a constraint on $m_\chi$ in the singlet case which can already be seen in figure \ref{pic_BsMixRepsMajorana}, because it does 
not have an intersection point with the bound from our presumed quark couplings. This leaves us with two remaining processes shown in figure 
\ref{pic_SinRes}. 
\begin{figure}[t]
 \includegraphics[width=\textwidth]{../pics/contour122.pdf}
 \caption{Contour plot for the singlet with two constraining processes. The parameter tuple points on the annihilation line and in between the 
 1 $\sigma$ deviation of $b\rightarrow s\bar\mu\mu$ are considered good solutions. They are depicted by the solid
 part of the annihilation.}
 \label{pic_SinRes}
\end{figure}
This contour plot shows solutions for $g_2^l$ and $m_\chi$ meeting their conditions. The DM candidate accounting for  
$b\rightarrow s\bar\mu\mu$ in the 1$\sigma$ region and fulfilling the correct relic density have a mass of roughly 50-80 GeV.\\
\noindent With a maximal $\xi$ for the smaller isospin representations and hence a small lifting from crossed boxes, the triplet faces strong
constraints from $B_s$-mixing. But also due to this property, the muonic coupling does not have to be very high so that the 1$\sigma$ region from
$b\rightarrow s\bar\mu\mu$ is reached. Another point is the help from the $WW$ annihilation channel for smaller $g_2^l$. These three issues can be
seen in figure \ref{pic_TriRes}. A DM candidate accomplishing them all have a mass in the region of 120-140 GeV.
\begin{figure}[t]
 \includegraphics[width=\textwidth]{../pics/contour322.pdf}
 \caption{Contour plot for the (322)-configuration with three constraining processes. The parameter tuple points on the annihilation line in between the 
 1 $\sigma$ deviation of $b\rightarrow s\bar\mu\mu$ and in the area of $\Delta m_s$ are considered good solutions. They are depicted by the solid
 part of the annihilation.}
 \label{pic_TriRes}
\end{figure}
\\
\noindent The final triplet configuration with a quadruplet $\Phi_l$ and a doublet $\Phi_q$ is discussed. Higher dimensional representations for 
$\Phi_l$ result in larger electrical charges which rises the $\Delta a_\mu$ contribution. For $x_l>0$ the loop function in the bosonic term of
\eqref{eq_g-2} is also larger than the respective one in the fermionic term. For different representations for the two scalars, a negative $\xi$
is found which makes higher $g_2^l$ necessary compared to the other configurations which is a feature in this context since a large muonic coupling
is already required by its anomalous magnetic moment. Meson-mixing is unchanged compared to the other triplet case as well as the annihilation 
and direct detection cross sections. The contour plot for the (342) configuration is shown in figure \ref{pic_Tri342Res}.
\begin{figure}[t]
 \includegraphics[width=\textwidth]{../pics/contour342-770.pdf}
 \caption{Contour plot for the (342)-configuration with four constraining processes. The parameter tuple points on the annihilation line in between 
 the 1 $\sigma$ deviation of $b\rightarrow s\bar\mu\mu$ and in the area of $\Delta m_s$ are considered good solutions. They are depicted by the solid
 part of the annihilation. The mass of $\Phi_q$ is slightly lifted to 770 GeV.}
 \label{pic_Tri342Res}
\end{figure}
The DM candidate fitting all five given process has a mass ranging around 40-50 GeV which is the lowest of the considered ones. 




% \begin{align}
%  m_\chi \in [50.5, 78.3] \,\text{GeV}.
% \end{align}
% 
% 
% \begin{align}
%  m_\chi \in [120.4, 138.3] \, \text{GeV}
% \end{align}





% 
% \subsection{A - $T(\chi)=\boldsymbol{1}$}
% % binolike
% \textit{$B_s$ Processes}\\
% \noindent \ref{pic_BsResA}
% \begin{figure}[t]
%  \includegraphics[width=1\textwidth]{../pics/BsModA.pdf}
%  \caption{Paramter space plot for $B_s$ mixing and $B_s\rightarrow \mu\mu$.}
%  \label{pic_BsResA}
% \end{figure}
% \\ \textit{Anomalous Magnetic Moment of the Muon}\\
% \noindent \ref{pic_g-2A}
% \begin{figure}[t]
%  \includegraphics[width=1\textwidth]{../pics/g-2A.pdf}
%  \caption{Paramter space plot for anomalous magnetic moment of the muon. The area of the blue line crossing the higher and lower bound is valid.}
%  \label{pic_g-2A}
% \end{figure}
% 
% \subsection{B - $T(\chi)=\boldsymbol{3}$}
% %winolike
% \textit{$B_s$ Processes}\\
% \noindent \ref{pic_BsResB}
% \begin{figure}[t]
%  \includegraphics[width=1\textwidth]{../pics/BsModB.pdf}
%  \caption{Paramter space plot for $B_s$ mixing and $B_s\rightarrow \mu\mu$.}
%  \label{pic_BsResB}
% \end{figure}
% \\ \textit{Anomalous Magnetic Moment of the Muon}\\
% \noindent \ref{pic_g-2B}
% \begin{figure}[t]
%  \includegraphics[width=1\textwidth]{../pics/g-2B.pdf}
%  \caption{Paramter space plot for anomalous magnetic moment of the muon. The area of the blue line crossing the higher and lower bound is valid.}
%  \label{pic_g-2B}
% \end{figure}
% \\ \textit{Direct Detection} \\
% \noindent \ref{pic_ddB}
% \begin{figure}[t]
%  \includegraphics[width=1\textwidth]{../pics/ddB.pdf}
%  \caption{Paramter space plot for nucleon scattering. The cross section for the triplet is barely below current bounds and will be tested next year.}
%  \label{pic_ddB}
% \end{figure}
% \\ \textit{Annihilation}\\
% With \eqref{eq_resParams} there are two masses for which $m_\chi$ as a thermal relic makes up for all DM in universe. They are 80 and 350 GeV. In
% between the cross section is too high leaving the DM candidate over abundant. Its peak is around 120 GeV.