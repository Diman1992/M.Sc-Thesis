\subsection{$U(1)_\text{FN}$ and Continuous Groups}
\label{sec_FNGT}
The mass hierarchy and the patterns of the CKM matrix suggest the presence of a symmetry. A very popular symmetry group was proposed
by Froggatt and Nielsen (FN) in \cite{FN}. 
\\ \\ \textit{Froggatt-Nielsen mechanism}\\
\noindent Within their mechanism, a global $\FN$ symmetry with one massive scalar messenger, 
or ``flavon'', field $\theta$ get introduced whose charge under this new symmetry is set to $[\theta]_\text{FN} = \Upsilon_\theta=-1$. At a high scale $\Lambda$ the
flavon develops a vev $\langle \theta \rangle = \epsilon\Lambda$, breaking the new symmetry spontaneously. The quarks are charged as well in such 
a way that their mass hierarchies \eqref{eq_masshierarchy} and the CKM structure \eqref{eq_ckm} are obtained. In order to have $\FN$ invariant
interactions, a certain amount of flavon fields have to couple too which give rise to a description for the SM Yukawa couplings \eqref{eq_massSM}
\begin{align}
 y^u_{ij} = \epsilon^{\Upsilon_{Q_i} + \Upsilon_{u_j}},\qquad y^d_{ij} = \epsilon^{\Upsilon_{Q_i} + \Upsilon_{d_j}}
 \label{eq_quarkyukawa}
\end{align}
with the respective $\FN$ charges for quark doublets and singlets and omitted $\mathcal{O}(1)$ coefficients. They are rotated into the mass basis 
due to bidiagonalisation with each two unitary
matrices ${\left(V_{u,d}\right)^\dagger y^{u,d}U_{u,d}=:\hat{y}^{u,d}}$ which can approximated by ratios of the Yukawa couplings \cite{1501.07268}
\begin{align}
 \left(V_{u,d}\right)_{ij} = \epsilon^{\Upsilon_{Q_i} - \Upsilon_{Q_j}},\qquad\left(U_{u,d}\right)_{ij} = \epsilon^{\Upsilon_{u_i,d_i} - \Upsilon_{u_j,d_j}}
\end{align}
while $i\leq j$. Using these expressions, we can determine a set of $\FN$ charges which yields the right phenomenology and adopt one of \cite{0501071}
\begin{align}
 \Upsilon_{Q_i}=(4,2,0),\quad \Upsilon_{u_i}=(4,2,0),\quad \Upsilon_{d_i}=(1,0,0).
 \label{eq_fnchargesQ}
\end{align}
Another important group is $SU(2)_L$ associated with weak isospin symmetry because it is the only non-trivial one our DM candidate will be charged 
under. For invariance under this group, the decomposed tensor product \eqref{eq_clebschseries} of all the representations must contain the trivial
singlet. The concept of Young-tableaux is a possibility of determining the compliance of this requirement. 
\\ \\ \textit{Young-tableaux}\\
\noindent They are diagrams usually composed of left-aligned rows and columns of squares. For an $SU(N)$ representation we fill the tableau according
to
\ytableausetup{mathmode, boxsize=2em}\begingroup\makeatletter\def\f@size{5}\check@mathfonts\def\maketag@@@#1{\hbox{\m@th\large\normalfont#1}}
\begin{align*}
\begin{ytableau}
N & N+1 & N+2 & \none[\textstyle{\ldots}]\\
N-1 & N\cr
N-2 &\none[] & \none[\ddots]\cr
\none[\vdots]
\end{ytableau}
\end{align*}\endgroup
We begin with the dimension of the Young-tableau $d$ which is the dimension of the representation. It is the product of all fillings 
divided by a hook parameter $h$. This parameter counts the boxes to its right and directly below as well as the starting
box itself. A simple example is in $SU(3)$
\ytableausetup{smalltableaux}
\begin{align*}
\text{Filling: }\begin{ytableau}
 3&4\\
 2
\end{ytableau}\hspace{0.5cm}/\hspace{0.5cm} \text{hook lengths: }\begin{ytableau}
 3&1\\
 1
\end{ytableau}\hspace{0.3cm}\rightarrow\hspace{0.3cm} d = \frac{3\cdot4\cdot2}{3\cdot 1\cdot 1} = 8.
\end{align*}
This tableau represents the octuplet which is the adjoint representation of $SU(3)$. The standard representation \textbf{3} has dimension 3 and is 
depicted by just one box \,\ydiagram{1}\, and its conjugate representation $\bar{\textbf{3}}$ by \, \ydiagram{1,1}\,. A product 
\eqref{eq_clebschseries} of an irreducible
representation with its conjugate always decomposes to the sum of the adjoint and the singlet representation. We will now verify this by a 
pictographic illustration:
\begin{align}
 \bar{\textbf{3}} \times \textbf{3} = \ydiagram{1,1}\times\ydiagram{1} =\ydiagram{2,1} + \ydiagram{1,1,1} = \textbf{8} + \textbf{1}.
 \label{eq_singletadjoint}
\end{align}
The situation for $SU(2)$ is quite easy since each conjugate is the representation itself. Moreover, a product of two even or two odd dimensional
representations is decomposed to only odd dimensional ones, whereas a product of an odd and an even dimensional representation decomposes to only
even dimensional ones. We covered the most important issues concerning continuous groups and now we will discuss discrete flavour symmetries.



% \Yboxdim<3>



%  \begin{Young}
%   $a$&&\cr
%   &\cr
%  \end{Young}




% baryon composition of quarks as a $3\times3\times3$ in $SU(3)_C$ \cite{mesonreps}




%  \textit{Lie-Groups}\\
% \\ \textit{Clebsch Gordan coefficients} \\
% \\ \textit{Froggat Nielsen-formalism}